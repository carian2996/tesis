\title{tesis}
\documentclass[a4paper]{book}
 
\usepackage[utf8]{inputenc}
\usepackage[spanish]{babel}
\usepackage{graphicx}
\usepackage{amsmath}
\usepackage{amsthm}
\usepackage{amsfonts}
\usepackage{accents}
\usepackage{fancyhdr, blindtext}


\newcommand{\changefont}{%
    \fontsize{9}{11}\selectfont
}

\DeclareMathOperator*{\esssup}{esssup}
\newcommand{\ubar}[1]{\underaccent{\bar}{#1}}

% Estilos de teoremas
\newtheorem{theorem}{Teorema}[chapter]
\newtheorem{lemma}[theorem]{Lema}
\newtheorem{proposition}{Proposición}[chapter]
\newtheorem{corollary}[theorem]{Corolario}

% Estilos de definiciones
\theoremstyle{definition}
\newtheorem{definition}{Definición}[chapter]
\newtheorem{example}{Ejemplo}[chapter]
\theoremstyle{remark}
\newtheorem{remark}{Observación}

\fancyhf{}
\fancyhead[LE]{\slshape \rightmark} %section
\fancyfoot[C]{\changefont \thepage}
\fancyhead[RO]{\slshape \leftmark} % chapter
\fancyfoot[C]{\changefont \thepage}

\pagestyle{fancy}

\begin{document}

\frontmatter

\begin{titlepage}

\begin{minipage}{.3\textwidth}
  \flushleft
  \center{\includegraphics[scale=.09]{images/unam.pdf}}

  \vspace{20pt}

  \center{
    \rule{.5pt}{.6\textheight}
    \hspace{7pt}
    \rule{2pt}{.6\textheight}
    \hspace{7pt}
    \rule{.5pt}{.6\textheight}
  } \\

  \center{\includegraphics[scale=.40]{images/fesa.png}}
\end{minipage}
\begin{minipage}{.7\textwidth}
\flushright

\center{

  \center{
    \LARGE{U}\large{NIVERSIDAD} \LARGE{N}\large{ACIONAL} 
    \LARGE{A}\large{UTÓNOMA} \\[10pt]
    \large{DE} 
    \LARGE{M}\large{ÉXICO} 
  } \\
  \rule{\textwidth}{2pt}
  \\
  \hrulefill\\[1cm]
  
  \LARGE{F}\large{ACULTAD DE } \LARGE{E}\large{STUDIOS } \LARGE{S}\large{UPERIORES } \LARGE{A}\large{CATLÁN }\\[2cm]

  \Large{\textbf{Problemas de Paro Óptimo}}\\[2cm]

  \huge{
T \hspace{1cm} E \hspace{1cm} S \hspace{1cm} I \hspace{1cm} S  }\\[1cm]

  \large{QUE PARA OBTENER EL TÍTULO DE}\\
  \large{\textbf{Actuario}}\\[1cm]

  \large{PRESENTA}\\
  \large{\textbf{Ian Castillo Rosales}}\\[1cm]

  \large{DIRECTORES}\\
  \large{Dr. José Luis Ángel Pérez Garmendia} \\
  \large{Dr. Juan Carlos Pardo Millán} \\[1.5cm]
  
  \large{\textbf{Santa Cruz Acatlán, Noviembre 2016}}
}

\end{minipage}

\end{titlepage}

\thispagestyle{empty}
\title{Problemas de Paro Óptimo}
\author{Ian Castillo Rosales}
\date{}


\maketitle

\thispagestyle{empty}
\begin{flushright}
\textit{A mi padre, Jacobo...}
\end{flushright}

\newpage

\thispagestyle{empty}
\begin{flushright}

\flushright Mis más sinceros agradecimientos: \\

\flushright A mi madre y heroína de mi familia, \\ Azucena.

\flushright A mis amigos y asesores, \\ Juan Carlos y José Luis.

\flushright Al Centro de Investigación en Matemáticas, \\ Por la valiosa ayuda y confianza en la investigación mexicana.

\flushright A los miembros de mi honorable jurado: \\ 
 
\flushright Lic. José Lucio Sánchez Garrido \\ 
Mat. Miguel Ángel Chávez García \\ 
Dr. José Luis Ángel Pérez Garmendia \\ 
Mat. Gamaliel Alejandro Bautista Mendoza \\ 
Mat. Alejandro Martínez Ireneo \\ 
 
\flushright Por sus invaluables comentarios de este trabajo.

\flushright A mi hermano y mejor amigo Erick, \\ por todas sus muestras de apoyo durante mi tesis.

\flushright A Sandra, \\ por aguantar todas las pláticas matemáticas.

\flushright A todas las personas que conocí en mi estancia en el CIMAT, \\ por su solidaridad y apego.

\flushright Y a todos aquellos que me alentaron en la realización de este \\ proyecto, además de su valiosa amistad. En especial a \\ Brenda C, Brenda M, Dania, Mayra, Carlos, Gerardo y Jonhatan.

\end{flushright}

\tableofcontents

\chapter{Introducción}
Los objetivos de este trabajo consisten en analizar los problemas de paro óptimo a tiempo discreto y continuo, estudiar las posibles estrategias para su solución y las características de la función de ganancia que surgen en este tipo de problemas. \\

Los problemas de paro óptimo consisten en determinar el mejor momento para tomar una decisión y reaccionar ante una secuencia de observaciones que provienen de variables aleatorias, tratando de maximizar las ganancias esperadas o minimizar los costos esperados. \\ 

Dichos problemas surgen naturalmente en muchas áreas de las matemáticas aplicadas, como lo son: la Estadística (Inferencial o Bayesiana), las Matemáticas Financieras o la Investigación de Operaciones. Este trabajo es un referente introductorio a este tipo de problemas y está basado en el artículo de Andreas Kyprianou: \textit{A Hands-On Approach to Optimal Stopping} realizado durante la 6ta Escuela de Probabilidad y Procesos Estocásticos en el Centro de Investigaciones en Matemáticas, Guanajuato, Guanajuato. \\

Esta tesis es autocontenida, salvo posibles ocasiones en las que se requiera mencionar un resultado importante y no sea posible realizar una demostración apropiada debido al alcance del mismo, entonces se hará referencia a la bibliografía. \\

La estructura del trabajo consta de cinco capítulos. En el primer capítulo se estudia una de las herramientas básicas para el desarrollo de esta tesis, las martingalas. Introducimos los conceptos de martingala, tiempo de paro, así como importantes desigualdades que nos llevan a resultados de convergencia de martingalas. \\

En el segundo capítulo presentamos una definición rigurosa del problema de paro óptimo a tiempo discreto. Iniciamos con el tiempo discreto para desarrollar la idea general y así analizar fácilmente el problema a tiempo continuo. Se define el supremo esencial, que es de suma importancia para obtener la solución del problema. Para finalizar el capítulo, además de utilizar las martingalas para resolver el problema de paro óptimo, se presenta un enfoque con cadenas de Markov. \\

El movimiento Browniano se estudia en el tercer capítulo, pues es un componente importante del problema de paro que se analiza en esta tesis. Presentamos una construcción del movimiento Browniano para observar propiedades importantes de sus trayectorias y examinar la propiedad fuerte de Markov que éste posee. \\

El cuarto capítulo está dedicado a estudiar las martingalas continuas y la generalización de los resultados vistos en el primer capítulo. Aquí es donde vemos las características que adquieren los tiempos de paro cuando se trata de un tiempo continuo. Analizamos las principales desigualdades maximales en martingalas que dan paso a revisar el Teorema de Paro de Doob resultado que nos guiará a encontrar una estrategia óptima para resolver el problema a considerar en el capítulo final. Por último, presentamos aplicaciones de las martingalas al movimiento Browniano, las cuales nos ayudarán durante el análisis de los resultados finales. \\

Finalmente, consideramos en  el problema de paro óptimo a tiempo continuo un proceso basado en tres componentes: una parte lineal positiva, un movimiento Browniano y un proceso Poisson. Esta familia de procesos estocásticos es conocida como procesos de Lévy. Vemos las condiciones que se requieren para que se cumpla la optimización requerida por problema de paro. Analizamos al Problema de McKean, un problema de paro que modela al instrumento financiero conocido como Opción Americana y concluimos estudiando una propiedad que se presenta para este tipo de problema de paro, la condición de \textit{Smooth Fit}. \\ 

A lo largo del trabajo se tiene el supuesto de que el lector posee las nociones básicas en teoría de medida, para lo cual se puede consultar \cite{jacodprotter} o \cite{shiryaev}. 

\mainmatter 

\chapter{Martingalas}
En el presente capítulo presentamos la definición de martingala, así como algunas propiedades que surgen gracias a la estructura de estos procesos. Las martingalas son una de las herramientas más útiles de la teoría de probabilidad moderna. En particular, las martingalas son una referencia para el estudio de teoremas límite o convergencia, pues la teoría de martingalas proporciona pruebas más elegantes y estéticamente atractivas.

\section{Definición}
Consideremos un espacio de probabilidad conocido y fijo $(\Omega, \mathcal{F}, \mathbb{P})$, así como una sucesión de $\sigma$-algebras $(\mathcal{F}_n, n \geq 0)$, tales que $\mathcal{F}_n \subset \mathcal{F}_{n+1} \subset \mathcal{F}$ para todo $n \geq 0$.

\begin{definition}[Martingala]
\label{martingala}
	Decimos que $X = (X_n, n \geq 0)$ es una $(\mathcal{F}_n)$-\emph{martingala}, si
	\begin{enumerate}
		\item $\mathbb{E}[|X_n|] < \infty$ para cada $n$;
		\item $X_n$ es $\mathcal{F}_n$-medible, para cada $n$;
		\item $\mathbb{E}[X_n \mid \mathcal{F}_m] = X_m$ c.s., para todo $m \leq n$.
	\end{enumerate}
\end{definition}

Podemos observar que (2) es \emph{casi} una implicación de (3), la cual sostiene que $X_m$ es casi seguramente igual a una variable aleatoria $\mathcal{F}_m$ medible. \\

Recordemos que la Ley Fuerte de los Grandes Números afirma que \cite[p.~173]{jacodprotter}: si $(X_n, n \geq 0)$ es una sucesión de variables aleatorias las cuales son independientes e idénticamente distribuidas con $\mathbb{E}[X_n] = \mu$ y $\sigma_{X_n}^{2} < \infty$ para toda $n$, y si $S_n = \sum_{j=1}^{n} X_j$, entonces $\lim_{n \rightarrow \infty} \frac{S_n}{n} = \mu$, casi seguramente. Por otro lado, la Ley Cero Uno de Kolmogorov nos asegura que \cite[p.~381]{shiryaev}, si consideramos una sucesión de variables aleatorias $(X_n, n \geq 0)$ independientes entre si, el evento cola debe tener una probabilidad 0 o 1, es decir, debe ser una constante. Entonces, $\frac{S_n}{n} \rightarrow \mu$ cuando $n \rightarrow \infty$, donde $\mu$ es una variable aleatoria cuyo valor siempre es constante. \\

Es de poco interés el estudio de sucesiones convergentes a límites que sean constantes. Sin embargo, si escribimos la sucesión anterior de una manera distinta tenemos que
\begin{align*}
	\lim_{n \rightarrow \infty} \frac{S_n - n\mu}{n} = 0, \hspace{0.3cm} c.s.
\end{align*}

Veamos en el siguiente ejemplo que una propiedad clave de la sucesión $(S_n - n\mu, n \geq 0)$ es que, si $\mathcal{F}_n = \sigma ( S_k \mid k \leq n )$, entonces la sucesión es una martingala.

\begin{example}
	Sea $(X_n, n \geq 0)$ una sucesión de variables aleatorias independientes con $\mathbb{E}[|X_n|] < \infty$, para toda $n$. Para $n \geq 0$ sea $\mathcal{F}_n = \sigma ( S_k \mid k \leq n )$ y $S_n = \sum_{k=0}^n X_k$. Entonces tenemos que para toda $m \leq n$
	\begin{align}
		\mathbb{E}[S_n - n\mu \mid \mathcal{F}_m] & = \mathbb{E}[S_m + (S_n - S_m) - n\mu \mid \mathcal{F}_m] \nonumber \\
		& = (S_m - n\mu) + \mathbb{E}[S_n - S_m \mid \mathcal{F}_m] \nonumber \\
		& = (S_m - n\mu) + \mathbb{E}\left[\sum_{k = m+1}^{n} X_k \mid \mathcal{F}_m\right] \nonumber \\
		& = (S_m - n\mu) + \sum_{k = m+1}^{n} \mathbb{E}[X_k \mid \mathcal{F}_m] \nonumber \\
		& = (S_m - n\mu) + \sum_{k = m+1}^{n} \mathbb{E}[X_k]  \label{aaa} \\
		& = S_m - m\mu. \nonumber
	\end{align}
	Donde (\ref{aaa}) se cumple pues $\mathcal{F}_m$ es la información generada por las variables $X_0, X_1, \ldots, X_m$ y entonces $(X_k, k \geq m+1)$ es independiente de $\mathcal{F}_m$. Por lo tanto, $(S_n - n\mu, n \geq 0)$ es una \emph{martingala}.
\end{example}

Una propiedad importante de las martingalas es que poseen una esperanza cuyo valor siempre es constante.
\begin{proposition} \label{martin}
	Si $(X_n, n \geq 0)$ es una martingala, entonces $\mathbb{E}[X_n] = \mathbb{E}[X_0]$, para toda $n \geq 0$.
\end{proposition}
\begin{proof}
	Si $(X_n, n \geq 0)$ es una martingala entonces tenemos que $X_0 = \mathbb{E}[X_n \mid \mathcal{F}_0]$ para $n \geq 0$ por lo que
	\begin{align*}
		\mathbb{E}[X_0] & = \mathbb{E}[\mathbb{E}[X_n \mid \mathcal{F}_0]] = \mathbb{E}[X_n],
	\end{align*}
para toda $n \geq 0$.
\end{proof}

Por otro lado, en general no sucede que si un proceso tiene un valor esperado constante, este proceso resulte ser una martingala. Sin embargo, con las condiciones necesarias la afirmación anterior llega a ocurrir. Para poder mostrar el resultado requerimos de una definición que será vital para desarrollo de este capítulo.

\begin{definition}[Tiempo de Paro]
Una variable aleatoria $\tau: \Omega \rightarrow \bar{\mathbb{N}} = \mathbb{N} \cup \{ \infty \}$ es un tiempo de paro con respecto a ($\mathcal{F}_n$) si $\{ \tau \leq n \} \in \mathcal{F}_n$.
\end{definition}

Los tiempos de paro pueden ser pensados como aquel tiempo en el que ocurre algún evento de interés, con la convención de que el tiempo de paro toma el valor $+ \infty$ si el evento nunca ocurre. El término ``tiempo de paro" proviene de un concepto involucrado con los juegos de apuestas: dependiendo de ciertos eventos, un apostador puede dejar el juego en cualquier momento (un tiempo aleatorio), pero en el momento en que decide detenerse, esa decisión debe estar basada en los eventos que han ocurrido en el pasado y no en eventos futuros.

\begin{proposition} 
\label{espconst}
	Sea $\tau$ un tiempo de paro acotado por $c$, es decir, $\mathbb{P}(\tau \leq c) = 1$ y sea $(X_n, n \geq 0)$ una martingala, entonces  $\mathbb{E}[X_{\tau}] = \mathbb{E}[X_0]$.
\end{proposition}
\begin{proof}
	Observemos que a $X_{\tau}$ la podemos escribir como,
	\begin{align*}
		X_{\tau} = \sum_{n=0}^{\infty} X_n1_{ \{\tau = n\} }.
	\end{align*}
	Entonces, sea $[c]$ la parte entera de $c$ o el máximo entero no superior a $c$,
	\begin{align*}
		\mathbb{E}[X_{\tau}] & = \mathbb{E}\left[\sum_{n=0}^{\infty} X_n 1_{ \{\tau = n\} } \right] \\
		& = \mathbb{E}\left[\sum_{n=0}^{[c]} X_n  1_{ \{\tau = n\} }\right] \\
		& = \sum_{n=0}^{[c]} \mathbb{E}[X_n 1_{ \{\tau = n\} }] \\
		& = \sum_{n=0}^{[c]} \mathbb{E}[\mathbb{E}[X_[c] \mid \mathcal{F}_n] 1_{ \{\tau = n\} }].
	\end{align*}
	Notemos que podemos escribir a $\{ \tau = n\}$ como $\{ \tau \leq n\} \setminus \{ \tau \leq n-1\}$ donde ambos eventos pertenecen a $\mathcal{F}_n$ por ser $\tau$ un tiempo de paro, por lo tanto $\{ \tau = n\} \in \mathcal{F}_n$, entonces
	\begin{align*}
		 \mathbb{E}[X_{\tau}] & = \sum_{n=0}^{[c]} \mathbb{E}[\mathbb{E}[X_{[c]} 1_{ \{\tau = n\} } \mid \mathcal{F}_n]] \\
		 & = \sum_{n=0}^{[c]} \mathbb{E}[X_{[c]}  1_{ \{\tau = n\} }] \\
		 & = \mathbb{E}\left[\sum_{n=0}^{[c]} X_c 1_{ \{\tau = n\} }\right] \\
		 & = \mathbb{E}\left[X_{[c]}  \sum_{n=0}^{[c]} 1_{ \{\tau = n\} }\right] \\
		 & = \mathbb{E}[X_{[c]}] = \mathbb{E}[X_0].
	\end{align*}
\end{proof}

La siguiente definición muestra que para un tiempo de paro $\tau$, $\mathcal{F}_{\tau}$ puede ser interpretada la información disponible para algún momento aleatorio. Esta $\sigma$-algebra es conocida como \emph{$\sigma$-algebra parada} y será de utilidad más adelante.

\begin{definition}  \label{algebraaleatoria}
	Sea $\tau$ un tiempo de paro, entonces tenemos que la $\sigma$-algebra parada en $\tau$ esta definida como,
	\begin{align*}
	\mathcal{F}_{\tau} = \{ \Lambda \in \mathcal{F} \mid \Lambda \cap \{ \tau \leq n \} \in \mathcal{F}_n, \text{ para toda } n \}.
	\end{align*}
\end{definition}

Veamos que la definición anterior tiene sentido, al probar que $\mathcal{F}_{\tau}$ es, efectivamente, una $\sigma$-algebra.
\begin{proposition}
	Sea $\tau$ un tiempo de paro, entonces $\mathcal{F}_{\tau}$ es una $\sigma$-algebra.
\end{proposition}
\begin{proof}
	Como $\{ \tau \leq n \} = \Omega \cap \{ \tau \leq n \} \in \mathcal{F}_n$ para toda $n$, de acuerdo a la definición de $\mathcal{F}_{\tau}$ se tiene que $\Omega \in \mathcal{F}_{\tau}$. Si $A \in \mathcal{F}_{\tau}$, podemos escribir a $A^{c} \cap B$ como $B \setminus (A \cap B)$ entonces:
	\begin{align}
		A^{c} \cap \{ \tau \leq n \} = \{ \tau \leq n \} \setminus (A \cap \{ \tau \leq n \}). \label{aab}
	\end{align}
	Donde, a la derecha de (\ref{aab}), ambas partes pertenecen a $\mathcal{F}_{n}$, por lo tanto, $A^{c} \in \mathcal{F}_{\tau}$. Por último, si $(A_k, k \geq 0) \in \mathcal{F}_{\tau}$ entonces observemos que
	\begin{align*}
		\left(\bigcup_{k=0}^{\infty} A_k\right) \cap \{ \tau \leq n \} = \bigcup_{k=0}^{\infty} (A_k \cap \{ \tau \leq n \}).
	\end{align*}
	Se tiene entonces que $\bigcup_{k=0}^{\infty} A_k \in \mathcal{F}_{\tau}$. \\
	
	Como $\mathcal{F}_{\tau}$ contiene al conjunto $\Omega$, y al ser cerrado bajo complementos y uniones contables,  $\mathcal{F}_{\tau}$ es una $\sigma$-algebra.
\end{proof}

Verifiquemos dos sencillas proposiciones que serán de ayuda a la hora de presentar un resultado importante, el  Teorema de Paro Opcional de Doob.
\begin{proposition}
\label{inclu}
	Sean dos tiempos de paro, $\gamma$ y $\tau$, tales que $\gamma \leq \tau$, entonces $\mathcal{F}_{\gamma} \subset \mathcal{F}_{\tau}$.
\end{proposition}
\begin{proof}
	Si sabemos que $\gamma \leq \tau$, entonces $\{ \tau \leq n \} \subset \{ \gamma \leq n \}$, y por lo tanto $\{ \tau \leq n \} = \{ \gamma \leq n \} \cap \{ \tau \leq n \}$. Si $\Lambda \in \mathcal{F}_{\gamma}$ entonces
	\begin{align*}
		\Lambda \cap \{ \tau \leq n \} = \Lambda \cap \{ \gamma \leq n \} \cap \{ \tau \leq n \}.
	\end{align*}
	Donde $\Lambda \cap \{ \gamma \leq n \} \in \mathcal{F}_n$ por la definición de $\mathcal{F}_{\gamma}$, de igual manera $\{ \tau \leq n \} \in \mathcal{F}_n$ por ser $\tau$ un tiempo de paro. Por lo tanto, $\Lambda \cap \{ \tau \leq n \} \in \mathcal{F}_n$ para toda $n$, es decir, $\Lambda \in \mathcal{F}_{\tau}$.
\end{proof}

\begin{proposition}
\label{medible}
	Supongamos que $(X_n, n \geq 0)$ es una sucesión de variables aleatorias tales que para toda $n$ se tiene que $X_n$ es $\mathcal{F}_n$ medible. Consideremos un tiempo de paro $\tau$, entonces $X_{\tau}$ es $\mathcal{F}_{\tau}$ medible.
\end{proposition}
\begin{proof}
	Recordemos que podemos reescribir el término $X_{\tau}$ como
	\begin{align*}
		X_{\tau} = \sum_{n=0}^{\infty} X_n  1_{ \{\tau = n\} }.
	\end{align*}
	Si consideremos un boreliano $B$, es necesario mostrar que el evento $\{ X_{\tau} \in B \} \in \mathcal{F}_{\tau}$, es decir, $\{ X_{\tau} \in B \} \cap \{ \tau \leq n\} \in \mathcal{F}_{n}$ para toda $n$. Por lo tanto,
	\begin{align*}
		\{ X_{\tau} \in B \} \cap \{ \tau \leq n\} & = \bigcup_{i=0}^{n} \{ X_{\tau} \in B \} \cap \{ \tau = i\} \\
		& = \bigcup_{i=0}^{n} \{ X_{i} \in B \} \cap \{ \tau = i\}.
	\end{align*}
	Como $\{ X_{i} \in B \} \cap \{ \tau = i\} \in \mathcal{F}_i \subset \mathcal{F}_n$ para toda $i \leq n$, entonces tenemos que $\{ X_{\tau} \in B \} \cap \{ \tau \leq n\} \in \mathcal{F}_{n}$, por lo que $X_{\tau}$ es $\mathcal{F}_{\tau}$ medible.
\end{proof}

Los siguientes teoremas muestran un poderoso resultado, que liga los conceptos presentados hasta ahora. Veamos que la propiedad de martingala se mantiene aún cuando se consideren tiempos de paro, en lugar de tiempos conocidos.

\begin{theorem}[Teorema de Paro Opcional de Doob] 
\label{opcional}
	Sea $X = (X_n, n \geq 0)$ una martingala, considere dos tiempos de paro $\gamma$ y $\tau$, acotados por una constante $K$, con $\gamma \leq \tau$ c.s., entonces
	\begin{align*}
	\mathbb{E}[X_{\tau} \mid \mathcal{F}_{\gamma}] = X_{\gamma}, \hspace{0.3cm} \text{c.s.}
	\end{align*}
\end{theorem}
\begin{proof}
	Recordemos que la definición de esperanza condicional nos dice que \cite[p.~200]{jacodprotter}: si una variable $Y \in \mathcal{L}^2(\Omega, \mathcal{A}, \mathbb{P})$ y consideramos $\mathcal{G}$ una sub $\sigma$-algebra de $\mathcal{A}$, entonces existe un elemento $\mathbb{E}[Y \mid \mathcal{G}] \in \mathcal{L}^2(\Omega, \mathcal{A}, \mathbb{P})$ donde la siguiente condición se cumple
	\begin{align*}
		\mathbb{E}[YX] = \mathbb{E}[\mathbb{E}[Y \mid \mathcal{G}] X] \hspace{0.3cm} \text{para toda } X \in \mathcal{L}^2(\Omega, \mathcal{A}, \mathbb{P}).
	\end{align*}
	Al ser $\tau$ y $\gamma$ tiempos acotados por una constante $K \in \mathbb{N}$, tenemos que $|X_{\tau}| \leq \sum_{n=0}^{K} |X_n|$, por lo que $X_{\tau}$ es integrable y por la misma razón, $X_{\gamma}$ también es integrable. Por la Proposición \ref{medible} sabemos que $X_{\gamma}$  es $\mathcal{F}_{\gamma}$ medible. \\
	
	Basta probar entonces que $\mathbb{E}[X_{\tau} Z] = \mathbb{E}[X_{\gamma} Z]$ para toda variable aleatoria $\mathcal{F}_{\gamma}$ medible. Resulta equivalente a mostrar que si $\Lambda \in \mathcal{F}_{\gamma}$ entonces,
	\begin{align*}
		\mathbb{E}[X_{\tau} 1_{\Lambda}] = \mathbb{E}[X_{\gamma} 1_{\Lambda}].
	\end{align*}
	Puesto que podemos aproximar una variable aleatoria cualquiera a partir de variables aleatorias simples y usando el Teorema de Convergencia Dominada de Lebesgue el cual afirma que \cite[p.~187]{shiryaev}: si $\lim_{n \rightarrow \infty} X_n = X$ c.s. y $|X_n| \leq K$ c.s., entonces $\lim_{n \rightarrow \infty} \mathbb{E}[X_n \mid \mathcal{F}] = \mathbb{E}[X \mid \mathcal{F}]$. \\
	
\noindent Definamos un nuevo tiempo aleatorio $\phi$ con $\Lambda \in \mathcal{F}_{\gamma}$,
	\begin{align*}
		\phi(\omega) = \gamma(\omega)1_{\Lambda}(\omega) + \tau(\omega)1_{\Lambda^c}(\omega).
	\end{align*}
Veamos que $\{\phi \leq n\} = (\Lambda \cap \{\gamma \leq n\}) \cup (\Lambda^c \cap \{\tau \leq n\})$, entonces, sabemos que como $\Lambda \in \mathcal{F}_{\gamma}$ tenemos que $(\Lambda \cap \{\gamma \leq n\}) \in \mathcal{F}_n$. Por otro lado, como $\Lambda \in \mathcal{F}_{\gamma}$ entonces $\Lambda^c \in \mathcal{F}_{\gamma}$ y por la Proposición \ref{inclu}, tenemos que $\Lambda^c \in \mathcal{F}_{\tau}$ por lo que $(\Lambda^c \cap \{\tau \leq n\}) \in \mathcal{F}_n$. Por lo tanto, $\phi$ es un tiempo de paro.\\
	
Por el hecho anterior y la Proposición \ref{espconst} sabemos que $\mathbb{E}[X_{\phi}] =  \mathbb{E}[X_0] = \mathbb{E}[X_{\tau}]$. Por último, veamos que,
	\begin{align*}
		\mathbb{E}[X_{\phi}] & = \mathbb{E}[X_{\gamma}1_{\Lambda}] + \mathbb{E}[X_{\tau}1_{\Lambda^c}], \\
		\mathbb{E}[X_{\tau}] & = \mathbb{E}[X_{\tau}1_{\Lambda}] + \mathbb{E}[X_{\tau}1_{\Lambda^c}].
	\end{align*}
	Por lo que, al sustraer $\mathbb{E}[X_{\phi}]$ de $\mathbb{E}[X_{\tau}]$ obtenemos
	\begin{align*}
		\mathbb{E}[X_{\tau}] & - \mathbb{E}[X_{\phi}] = 0, \\
		(\mathbb{E}[X_{\tau}1_{\Lambda}] + \mathbb{E}[X_{\tau}1_{\Lambda^c}]) & - (\mathbb{E}[X_{\gamma}1_{\Lambda}] + \mathbb{E}[X_{\tau}1_{\Lambda^c}]) = 0, \\
		\mathbb{E}[X_{\tau}1_{\Lambda}] & - \mathbb{E}[X_{\gamma}1_{\Lambda}] = 0. \\
	\end{align*}
	Por lo tanto, $\mathbb{E}[X_{\tau}1_{\Lambda}] = \mathbb{E}[X_{\gamma}1_{\Lambda}]$ para todo $\Lambda \in \mathcal{F}_{\gamma}$.
\end{proof}

Con las conclusiones vistas hasta ahora podemos establecer un resultado parcial del recíproco de la Proposición \ref{martin}, el cual nos asegura que, dadas las condiciones adecuadas, podemos afirmar que un proceso con esperanza constante, resulta ser una martingala.

\begin{theorem}
\label{condmartin}
	Sea $(X_n, n \geq 0)$ una sucesión de variables aleatorias, tal que $X_n$ es $\mathcal{F}_n$-medible para toda $n$. Supongamos que $\mathbb{E}[|X_n|] < \infty$ para toda $n$ y que $\mathbb{E}[X_{\tau}] = \mathbb{E}[X_0]$ para todo tiempo de paro acotado $\tau$. Se tiene que, $(X_n, n \geq 0)$ es una martingala.
\end{theorem}
\begin{proof}
	Para verificar que $(X_n, n \geq 0)$ es una martingala debemos verificar que se cumpla que para todo tiempo $0 \leq m < n < \infty$
	\begin{align*}
		\mathbb{E}[X_n \mid \mathcal{F}_m] = X_m.
	\end{align*}
	Donde $\mathbb{E}[X_n \mid \mathcal{F}_m]$ representa el valor esperado del proceso al tiempo $n$ condicionado con respecto a la información a tiempo $m$. De la misma manera que en el Teorema \ref{opcional}, basta con verificar que $\mathbb{E}[X_n 1_{\Lambda}] = \mathbb{E}[X_m 1_{\Lambda}]$ para toda $\Lambda \in \mathcal{F}_m$.
	Sea $0 \leq m < n < \infty$ y $\Lambda \in \mathcal{F}_m$. Definamos el tiempo aleatorio $\tau$ como
	\begin{align*}
		\tau (\omega) =
		\begin{cases}
			n & \text{ si } \omega \in \Lambda, \\
			m & \text{ si } \omega \notin \Lambda.
		\end{cases}
	\end{align*}
	Podemos verificar fácilmente que $\tau$ es un tiempo de paro, ya que $\{\tau \leq n\} = (\Lambda \cap \{n = n\}) \cup (\Lambda^c \cap \{m \leq n\})$. Si $\omega \in \Lambda$ entonces $\{\tau = n\} \in \mathcal{F}_n$, además $\Lambda \in \mathcal{F}_m \subset \mathcal{F}_n$. Por otro lado, si $\omega \notin\Lambda$, entonces $\{m \leq n\} \in \mathcal{F}_m \subset \mathcal{F}_n$ y como $\Lambda \in \mathcal{F}_m$ entonces $\Lambda^c \in \mathcal{F}_m \subset \mathcal{F}_n$. \\
	
\noindent Al ser $\tau$ un tiempo de paro tenemos que
	\begin{align*}
		\mathbb{E}[X_0] = \mathbb{E}[X_{\tau}] = \mathbb{E}[X_m1_{\Lambda^c} + X_n1_{\Lambda}].
	\end{align*}
	Por otro lado
	\begin{align*}
		\mathbb{E}[X_0] = \mathbb{E}[X_m1_{\Lambda^c} + X_m1_{\Lambda}].
	\end{align*}
	Sustrayendo $\mathbb{E}[X_0]$ de $\mathbb{E}[X_{\tau}]$ tenemos
	\begin{align*}
		\mathbb{E}[X_m1_{\Lambda^c}] + \mathbb{E}[X_n1_{\Lambda}] & - (\mathbb{E}[X_m1_{\Lambda^c}] + \mathbb{E}[X_m1_{\Lambda}]) = 0, \\
		\mathbb{E}[X_n1_{\Lambda}] & -  \mathbb{E}[X_m1_{\Lambda}] = 0.
	\end{align*}
	Por lo tanto, $\mathbb{E}[X_n1_{\Lambda}] = \mathbb{E}[X_m1_{\Lambda}]$ para toda $\Lambda \in \mathcal{F}_m$, demostrando así que $\mathbb{E}[X_n \mid \mathcal{F}_m] = X_m$, es decir, $(X_n, n \geq 0)$ es una martingala. 
\end{proof}

\section{Supermartingalas y Submartingalas}
Para llegar a un segundo resultado importante en la teoría de martingalas, es necesario introducir un nuevo concepto asociado con nuestra Definición \ref{martingala}. Si reemplazamos la igualdad por una desigualdad en las condiciones dadas de la definición de martingala obtenemos el siguiente concepto. \\

Consideremos un espacio de probabilidad dado $(\Omega, \mathcal{F}, \mathbb{P})$ y una sucesión creciente de $\sigma$-algebras $(\mathcal{F}_n, n \geq 0)$.

\begin{definition}
\label{submartin}
	Una sucesión de variables $(X_n, n \geq 0)$ es llamada \emph{submartingala} si,
	\begin{enumerate}
		\item $\mathbb{E}[|X_n|] < \infty$ para cada $n$;
		\item $X_n$ es $\mathcal{F}_n$ medible, para cada $n$;
		\item $\mathbb{E}[X_n \mid \mathcal{F}_m] \geq X_m$ c.s., para todo $m \leq n$.
	\end{enumerate}
\end{definition}

Si $(X_n, n \geq 0)$ es una submartingala, entonces definimos al proceso $(- X_n, n \geq 0)$, para toda $n \geq 0$, es una \emph{supermartingala}. La sucesión $(X_n, n \geq 0)$ es una martingala si y solo si es una submartingala y al mismo tiempo una supermartingala. \\

Recordemos que la desigualdad de Jensen nos asegura lo siguiente \cite[p.~205]{jacodprotter}: si una función $\phi: \mathbb{R} \rightarrow \mathbb{R}$ es convexa, y las variables aleatorias $X$ y $\phi(X)$ son integrables, entonces para cualquier $\sigma$-algebra $\mathcal{F}$ se tiene que $\phi(\mathbb{E}[X \mid \mathcal{F}]) \leq \mathbb{E}[\phi(X) \mid \mathcal{F}]$. A través de este resultado podemos ligar las Definiciones \ref{martingala} y \ref{submartin} con la siguiente proposición.

\begin{proposition}
\label{convexa}
	Si $(X_n, n \geq 0)$ es una martingala, coni $\phi$ una función convexa y $\phi(X_n)$ es integrable para toda $n$, entonces $(\phi(X_n), n \geq 0)$ es una submartingala.
\end{proposition}
\begin{proof}
	Sea $m \leq n$, tenemos que $\mathbb{E}[X_n \mid \mathcal{F}_m] = X_m$, c.s., entonces al aplicar $\phi(\mathbb{E}[X_n \mid \mathcal{F}_m]) = \phi(X_m)$, c.s. y como $\phi$ es una función convexa podemos concluir de la desigualdad de Jensen
	\begin{align*}
		\mathbb{E}[\phi(X_n) \mid \mathcal{F}_m] \geq \phi(\mathbb{E}[X_n \mid \mathcal{F}_m]) = \phi(X_m).
	\end{align*}
	Por lo tanto, $(\phi(X_n), n \geq 0)$ es una submartingala.
\end{proof}

Consideremos la función $\phi(x) = |x|$, la cual es una función convexa. Por la Proposición \ref{convexa}, tenemos que si $(M_n, n \geq 0)$ es una martingala entonces $X_n = |M_n|$ para toda $n$ es una submartingala. \\

Recordemos que la propiedad de martingala se mantiene aún usando tiempos de paro acotados (Teorema \ref{opcional}), de igual manera, la propiedad de submartingala se mantiene bajo las mismas condiciones. \\

El siguiente teorema muestra la fuerte relación que existe entre las martingalas y submartingalas. El resultado asegura una descomposición única de un proceso submartingala como la suma de un proceso martingala y un proceso predecible.

\begin{theorem}[Descomposición de Doob]
\label{descdoob}
	Sea $(X_n, n \geq 0)$ es una submartingala. Entonces, se tiene la siguiente descomposición del proceso para cada $n \geq 1$
	\begin{align*}
	X_n = X_0 + M_n + A_n, \hspace{1cm} \text{con    } M_0 = A_0 = 0.
	\end{align*}
	donde $(M_n, n \geq 0)$ es una martingala y $(A_n, n \geq 0)$ satisface que $A_{n+1} \geq A_n$, c.s. y $A_{n+1}$ es $\mathcal{F}_n$-medible para toda $n$. Más aún, la descomposición es única.
\end{theorem}
\begin{proof}
	Como $(X_n, n \geq 0)$ es una submartingala se tiene que $\mathbb{E}[X_{k+1} \mid \mathcal{F}_k] \geq X_k$ entonces $\mathbb{E}[X_{k+1} \mid \mathcal{F}_k] - X_k \geq 0$. Sin embargo, $X_k$ es una variable aleatoria $\mathcal{F}_k$-medible, por lo tanto $\mathbb{E}[X_{k+1} - X_k \mid \mathcal{F}_k] \geq 0$.
	Definamos al proceso $A_0 = 0$ y
	\begin{align*}
	A_n = \sum_{k=1}^{n} \mathbb{E}[X_k - X_{k-1} \mid \mathcal{F}_{k-1}], \hspace{0.3cm} \text{con } n \geq 1.
	\end{align*}
	Por lo tanto, tenemos que $A_n \leq A_{n+1}$ c.s. y además $A_k$ es $\mathcal{F}_{k-1}$-medible. Mostremos la existencia de la martingala $(M_n, n \geq 0)$. \\
	
	Notemos que $X_{n-1}$ es una variable $\mathcal{F}_{n-1}$-medible, entonces $\mathbb{E}[X_n \mid \mathcal{F}_{n-1}] - X_{n-1} = \mathbb{E}[X_n - X_{n-1} \mid \mathcal{F}_{n-1}]$. Por otro lado, 
	\begin{align*}
	A_n - A_{n-1} & = \sum_{k=1}^{n} \mathbb{E}[X_k - X_{k-1} \mid \mathcal{F}_{k-1}] - \sum_{k=1}^{n-1} \mathbb{E}[X_k - X_{k-1} \mid \mathcal{F}_{k-1}] \\
	& = \mathbb{E}[X_n - X_{n-1} \mid \mathcal{F}_{n-1}]. 
	\end{align*}
	
	\noindent Teniendo en cuenta ambas igualdades obtenemos la siguiente igualdad
	\begin{align*}
		\mathbb{E}[X_n \mid \mathcal{F}_{n-1}] - X_{n-1} = A_n - A_{n-1}.
	\end{align*}
	Es decir
	\begin{align*}
		\mathbb{E}[X_n \mid \mathcal{F}_{n-1}] - A_n = X_{n-1} - A_{n-1}.
	\end{align*}
	Sin embargo, por definición sabemos que $A_n \in \mathcal{F}_{n-1}$, entonces 
	\begin{align*}
		\mathbb{E}[X_n - A_n \mid \mathcal{F}_{n-1}] = X_{n-1} - A_{n-1}.
	\end{align*}
	Por tanto, si definimos a $M_n = X_n - A_n$ para toda $n$, tenemos que $(M_n, n \geq 0)$ es una martingala. Aseguramos entonces la existencia de la descomposición. \\
	
	Para demostrar la unicidad de la descomposición, supongamos que para la submartingala $(X_n, n \geq 0)$ existen dos descomposiciones tales que
	\begin{align*}
		X_n &= X_0 + M_n + A_n, \hspace{0.3cm} n \geq 1, \\
		X_n &= X_0 + L_n + B_n, \hspace{0.3cm} n \geq 1.
	\end{align*}
	De la diferencia entre ambas descomposiciones obtenemos
	\begin{align*}
		M_n + A_n = L_n + B_n, \hspace{0.3cm} n \geq 1.
	\end{align*}
	Es decir, 
	\begin{align*}
		M_n - L_n = B_n - A_n, \hspace{0.3cm} n \geq 1.
	\end{align*}
	Recordemos que por su construcción, $B_n$ y $A_n$ son variables $\mathcal{F}_{n-1}$ medibles, entonces $B_n - A_n \in \mathcal{F}_{n-1}$, por lo tanto, $M_n - L_n \in \mathcal{F}_{n-1}$. Tenemos entonces que $M_n - L_n = \mathbb{E}[M_n - L_n \mid \mathcal{F}_{n-1}]$. Por otro lado, sabemos que $M_n$ y $L_n$ son martingalas, entonces
	\begin{align*}
		M_n - L_n = \mathbb{E}[M_n - L_n \mid \mathcal{F}_{n-1}] = M_{n-1} - L_{n-1} = B_{n-1} - A_{n-1}, \hspace{0.3cm} \text{c.s.}
	\end{align*}
	Siguiendo inductivamente este procedimiento llegamos a que
	\begin{align*}
		M_n - L_n = M_0 - L_0, \hspace{0.3cm} \text{c.s.}
	\end{align*}
	Por hipótesis sabemos que $M_0 = L_0 = 0$, y entonces podemos asegurar la unicidad de la descomposición ya que $M_n = L_n$, c.s. y ademas $B_n = A_n$, c.s.
\end{proof}

\begin{corollary}
	Sea $(X_n, n \geq 0)$ es una supermartingala. Entonces, existe una única descomposición del proceso para cada $n \geq 1$,
	\begin{align*}
		X_n = X_0 + M_n - A_n,
	\end{align*}
	con $M_0 = A_0 = 0$, $(M_n, n \geq 0)$ una martingala y un proceso $(A_n, n \geq 0)$ tal que $A_{n+1} \geq A_n$ c.s. y $A_{n+1}$ es $\mathcal{F}_n$-medible para toda $n$.
\end{corollary}
\begin{proof}
	Si definimos a $Y_n = - X_n$ entonces, el proceso $(Y_n, n \geq 0)$ es una submartingala. Por el Teorema \ref{descdoob}, tenemos la siguiente descomposición para el proceso
	\begin{align*}
		Y_n = Y_0 + L_n + B_n,
	\end{align*}
Entonces $X_n = X_0 - L_n - B_n$, al definir $M_n = - L_n$ y $A_n = B_n$ para toda $n \geq 1$.
\end{proof}

\section{Desigualdades Maximales}
A continuación mostramos las principales desigualdades para martingalas, las cuales serán una herramienta para el estudio de la convergencia de martingalas. Consideremos un espacio de probabilidad fijo y conocido $(\Omega, \mathcal{F}, \mathbb{P})$ y una sucesión de $\sigma$-algebras $(\mathcal{F}_n)_{n \geq 0}$ tales que $\mathcal{F}_n \subset \mathcal{F}_{n+1} \subset \mathcal{F}$ para toda $n$. Consideremos una sucesión de variables aleatorias integrables $(M_n, n \geq 0)$ y además, para cada $n$, se tiene que $M_n$ es $\mathcal{F}_n$-medible. Definamos
\begin{align*}
	M_n^{*} = \sup_{j \leq n} |M_j|.
\end{align*}
Es claro que $M_n^{*} \leq M_{n+1}^{*}$, entonces para toda $m \leq n$ se tiene que $M_m^{*} = \mathbb{E}[M_m^{*} \mid \mathcal{F}_m] \leq \mathbb{E}[M_{n}^{*} \mid \mathcal{F}_m]$ c.s., además 
\begin{align*}
	M_n^{*} = \sup_{j \leq n} |M_j| \leq \sum_{j=0}^n |M_j|.
\end{align*}
Por lo tanto, $\mathbb{E}[M_n^{*} ]\leq \mathbb{E}[\sum_{j=0}^n |M_j|] < \infty$. Con estas condiciones, aseguramos que el proceso $(M_n^{*}, n \geq 0)$ es una submartingala. Estamos interesados en conocer la probabilidad de que el valor $M_n^{*}$ supere una cantidad $\alpha$. \\

Recordemos que la \emph{Desigualdad de Markov} afirma que \cite[p.~29]{jacodprotter} para una variable aleatoria $X: \Omega \rightarrow \mathbb{R}$ se tiene que $\mathbb{P}(|X| \geq \alpha) \leq \mathbb{E}[|X|]/\alpha$ para toda $\alpha > 0$, entonces
\begin{align}
	\mathbb{P}(M_n^{*} \geq \alpha) \leq \frac{\mathbb{E}[M_n^{*}]}{\alpha}. \label{aac}
\end{align}
En el caso donde $(M_n, n \geq 0)$ no solo cumple con las condiciones mencionadas, sino que además satisface que es una martingala, podemos reemplazar $M_n^{*}$ por $|M_n|$ en el lado derecho de (\ref{aac}), dando como resultado el siguiente teorema.

\begin{theorem}[Primera Desigualdad de Martingalas de Doob] 
\label{primera}
	Sea $(M_n, n \geq 0)$ una martingala o una submartingala positiva. Entonces
	\begin{align*}
		\mathbb{P}(M_n^{*} \geq \alpha) \leq \frac{\mathbb{E}[|M_n|]}{\alpha}.
	\end{align*}
\end{theorem}
\begin{proof}
	Consideremos en primer lugar el siguiente tiempo de paro
	\begin{align*}
		\tau = \min \{j : |M_j| \geq \alpha\}.
	\end{align*}
	Veamos que sin importar que caso consideremos para $(M_n, n \geq 0)$ (martingala o submartingala positiva), $(|M_n|, n \geq 0)$ es una submartingala. \\ 
	
	Si $M$ una martingala, por la Proposición \ref{convexa} tenemos que $(|M_n|, n \geq 0)$ es una submartingala. Por otro lado, si $M$ es una submartingala positiva entonces resulta que $|M_n| = M_n$ para toda $n$. Por lo que $(|M_n|, n \geq 0)$ es una submartingala.
	% $M_m \leq \mathbb{E}[M_n \mid \mathcal{F}_m]$ c.s. para toda $m \leq n$, además, como $M$ es positiva para toda $n$ tenemos que $M_n > 0$, entonces, $|M_n|$ es creciente para toda $n$, ya que $\phi(x) = |x|$ es una función convexa y creciente en $\mathbb{R}_{+}$, por lo tanto
	\begin{align}
	|M_m| \leq  |\mathbb{E}[M_n \mid \mathcal{F}_m]|. \label{corrJC1}
	\end{align}
De la desigualdad de Jensen tenemos que 
	\begin{align}
	|\mathbb{E}[M_n \mid \mathcal{F}_m]| \leq \mathbb{E}[|M_n| \mid \mathcal{F}_m]. \label{corrJC2}
	\end{align}
De (\ref{corrJC1}) y (\ref{corrJC2}) tenemos que $(|M_n|, n \geq 0)$ es una submartingala. \\
	
	Observemos que el evento $\{ \tau \leq n, |M_{\tau}| \geq \alpha\}$ representa el primer momento en el que ocurre la condición $\{|M_{\tau}| \geq \alpha\}$ que resulta ser el mismo evento que $\{\sup_{j \leq n} |M_j| = M_n^{*} \geq \alpha\}$, entonces
	\begin{align}
		\mathbb{P}(M_n^{*} \geq \alpha) = \mathbb{P}(\tau \leq n, |M_{\tau}| \geq \alpha). \label{aad}
	\end{align}
Utilizando la desigualdad de Markov tenemos que 
	\begin{align}
		\mathbb{P}(\tau \leq n, |M_{\tau}| \geq \alpha) = \mathbb{P}(M_n^{*} \geq \alpha)  \leq \frac{\mathbb{E}[M_n^{*}]}{\alpha} = \mathbb{E} \left[\frac{|M_{\tau}|}{\alpha}  1_{\{\tau \leq n\}} \right]. \label{aae}
	\end{align}
Veamos por otro lado, en el conjunto $\{ \tau \leq n\}$ tenemos que $M_{\tau} = M_{\tau \wedge n}$. De (\ref{aad}) y (\ref{aae}) tenemos que
	\begin{align*}
		\mathbb{P}(M_n^{*} \geq \alpha) \leq \frac{1}{\alpha} \mathbb{E}[|M_{\tau}|  1_{\{\tau \leq n\}}] = \frac{1}{\alpha} \mathbb{E}[|M_{\tau \wedge n}|  1_{\{\tau \leq n\}}].
	\end{align*}
Además tenemos que 
	\begin{align*}
		\frac{1}{\alpha} \mathbb{E}[|M_{\tau \wedge n}|  1_{\{\tau \leq n\}}] \leq \frac{1}{\alpha} \mathbb{E}[|M_{\tau \wedge n}|].
	\end{align*}
Por lo tanto
	\begin{align}
		\mathbb{P}(M_n^{*} \geq \alpha) \leq \frac{1}{\alpha} \mathbb{E}[|M_{\tau}|  1_{\{\tau \leq n\}}] \leq \frac{\mathbb{E}[|M_{\tau \wedge n}|]}{\alpha}. \label{aaf}
	\end{align}
	Por último, con una prueba análoga de la Proposición \ref{espconst} se sabe que para un tiempo de paro $\tau$ acotado por $c \in \mathbb{N}$ y una submartingala $(X_n, n \geq 0)$ se tiene que $\mathbb{E}[X_{\tau}] \leq \mathbb{E}[X_{c}]$. Entonces, 
	%recordemos que de la Proposición \ref{espconst} y las observaciones de la Proposición \ref{convexa} tenemos que la propiedad de submartingala se mantiene aún usando tiempos aleatorios, y por lo tanto obtenemos la siguiente desigualdad para el tiempo de paro $(\tau \wedge n)$
	\begin{align}
		\frac{\mathbb{E}[|M_{\tau \wedge n}|]}{\alpha} \leq \frac{\mathbb{E}[|M_{n}|]}{\alpha} \label{aag}
	\end{align}
	De (\ref{aaf}) y (\ref{aag}) tenemos la desigualdad deseada
	\begin{align*}
		\mathbb{P}(M_n^{*} \geq \alpha) \leq \frac{\mathbb{E}[|M_{n}|]}{\alpha}.
	\end{align*}
\end{proof}

Será de mucha ayuda mostrar un sencillo pero valioso resultado que nos permitirá verificar, más adelante, nuestra segunda desigualdad de martingalas.

\begin{lemma}
\label{repint}
Sean $p \geq 1$ y $X \geq 0$ una variable aleatoria, tal que $\mathbb{E}[X^p] < \infty$. Entonces
	\begin{align*}
		\mathbb{E}[X^p] = \int_0^{\infty} p \lambda^{p-1} \mathbb{P}(X > \lambda) d\lambda.
	\end{align*}
\end{lemma}
\begin{proof}
De las propiedades de la esperanza tenemos que
	\begin{align*}
		\int_0^{\infty} p \lambda^{p-1} \mathbb{P}(X > \lambda) d\lambda = \int_0^{\infty} p \lambda^{p-1} \mathbb{E}[1_{\{X > \lambda\}}] d\lambda = \int_0^{\infty} p \lambda^{p-1} \int_{\Omega} 1_{\{X > \lambda\}} d\mathbb{P} d\lambda.
	\end{align*}
Del teorema de Fubini,
	\begin{align*}
		\int_0^{\infty} p \lambda^{p-1} \int_{\Omega} 1_{\{X > \lambda\}} d\mathbb{P} d\lambda & = \int_0^{\infty} \int_{\Omega} p \lambda^{p-1} 1_{\{X > \lambda\}} d\mathbb{P} d\lambda \\
		& = \int_{\Omega} \int_0^{\infty} p \lambda^{p-1} 1_{\{X > \lambda\}} d\lambda d\mathbb{P} \\
		& = \int_{\Omega} \int_0^{X} p \lambda^{p-1} d\lambda d\mathbb{P} \\
		& = \mathbb{E}\left[ \int_0^{X} p \lambda^{p-1} d\lambda \right].
	\end{align*}
	Por lo tanto, 
	\begin{align*}
		\int_0^{\infty} p \lambda^{p-1} \mathbb{P}(X > \lambda) d\lambda = \mathbb{E}[X^p].
	\end{align*}
\end{proof}

\begin{theorem}[Desigualdad de Doob en $L^p$ para Martingalas]
\label{lp}
	Sea $M = (M_n, n \geq 0)$ una martingala o una submartingala positiva. Sea $1 < p < \infty$, entonces existe una constante $c$ que depende solamente de $p$ tal que
	\begin{align*}
		\mathbb{E}[(M_n^{*})^p] \leq c\mathbb{E}[|M_n|^p].
	\end{align*}
\end{theorem}
\begin{proof}
	Consideremos el caso en que $M$ es una martingala. Como $\phi(x) = |x|$ es una función convexa entonces por la Proposición \ref{convexa} tenemos que $|M|$ es una submartingala. Consideremos además los siguiente procesos para una $n$ fija,
	\begin{align*}
		X_n & = M_n1_{\{|M_n| > \frac{\alpha}{2} \} }, \\
		Z_j & = \mathbb{E}[X_n \mid \mathcal{F}_j], \hspace{0.3cm} 0 \leq j \leq n.
	\end{align*}
	Notemos que para un valor $n$ conocido se tiene que $\mathbb{E}[Z_m \mid \mathcal{F}_k] = \mathbb{E}[\mathbb{E}[X_n \mid \mathcal{F}_m] \mid \mathcal{F}_k] = \mathbb{E}[X_n \mid \mathcal{F}_k] = Z_k$ para $k \leq m \leq n$, por lo tanto $Z_j$, $0 \leq j \leq n$ es una martingala. Además
	\begin{align}
		|M_j| & = |\mathbb{E}[M_n \mid \mathcal{F}_j]| \nonumber \\
		& = \left|\mathbb{E}\left[ M_n 1_{\{|M_n > \frac{\alpha}{2}|\}} + M_n 1_{\{|M_n \leq \frac{\alpha}{2}|\}} \mid \mathcal{F}_j \right] \right| \nonumber \\
		& = \left| \mathbb{E}\left[X_n + M_n 1_{\{|M_n \leq \frac{\alpha}{2}|\}} \mid \mathcal{F}_j \right] \right| \nonumber \\
		& \leq \left| \mathbb{E}\left[ X_n \mid \mathcal{F}_j \right] \right| + \frac{\alpha}{2} \nonumber \\
		& = |Z_j| + \frac{\alpha}{2}. \label{aah}
	\end{align}
	Entonces de (\ref{aah}) tenemos, 
	\begin{align*}
		M_n^{*} = \sup_{j \leq n} |M_j| \leq \sup_{j \leq n} |Z_j| + \frac{\alpha}{2} = Z_n^{*} + \frac{\alpha}{2}.
	\end{align*}
	Por lo tanto, 
	\begin{align}
		\mathbb{P}(M_n^{*} > \alpha) \leq \mathbb{P}(Z_n^{*} + \frac{\alpha}{2} > \alpha) =  \mathbb{P}(Z_n^{*} > \frac{\alpha}{2}). \label{aai}
	\end{align}
	Aplicando la Primera Desigualdad de Doob (Teorema \ref{primera}) a (\ref{aai}) 
	\begin{align*}
		\mathbb{P}(M_n^{*} > \alpha) & \leq \mathbb{P}(Z_n^{*} > \frac{\alpha}{2}) \\
		& \leq \frac{2}{\alpha} \mathbb{E}[|Z_n|] \leq \frac{2}{\alpha} \mathbb{E}[|X_n|] \ (\text{Jensen}) \\
		& = \frac{2}{\alpha} \mathbb{E}[|M_n|1_{\{|M_n| > \frac{\alpha}{2}\}}].
	\end{align*}
	Con este último resultado podemos aplicar a la igualdad del Lema \ref{repint} y verificar la desigualdad deseada
	\begin{align*}
		\mathbb{E}[(M_n^{*})^p] & = \int_0^{\infty} p \lambda^{p-1} \mathbb{P}(M_n^{*} > \lambda) d\lambda \\
		& \leq \int_0^{\infty} p \lambda^{p-1}  \frac{2}{\lambda} \mathbb{E}[|M_n|1_{\{|M_n| > \frac{\lambda}{2}\}}] d\lambda.
	\end{align*}
	Usando el Teorema de Fubini y realizando los cálculos de la integral obtenemos
	\begin{align*}
		& = \int_0^{\infty} 2p \lambda^{p-2} \mathbb{E}[|M_n|1_{\{2|M_n| > \lambda\}} ] d\lambda \\
		& = \mathbb{E}\left[ |M_n| \int_0^{2|Mn|} 2p\lambda^{p-2} d\lambda \right] \\
		& = \frac{2^p p}{p-1} \mathbb{E}[|M_n|^p].
	\end{align*}
	Tenemos entonces que
	\begin{align*}
	\mathbb{E}[(M_n^{*})^p] \leq c\mathbb{E}[|M_n|^p], \hspace{0.3cm} \text{con} \hspace{0.3cm} c \leq \frac{2^p p}{p-1}.
	\end{align*} 
	
Para el caso en que $M$ es una submartingala positiva se procede con la misma demostración, con el conocimiento de que $|M|$ es una submartingala (Teorema \ref{primera}).
\end{proof}

Con respecto al resultado anterior, hemos mostrado la relación existente entre la constante $c$ y el valor $p$, sin embargo, se puede verificar que $c^{\frac{1}{p}} = \frac{p}{p-1}$ y entonces podemos escribir la desigualdad en términos de normas $L^p$.

\begin{theorem}[Desigualdad de Doob en $L^p$ para Martingalas]
\label{lp2}
	Sea $M = (M_n, n \geq 0)$ una martingala o una submartingala positiva. Sea $1 < p < \infty$. Entonces
	\begin{align*}
		||M_n^{*}||_p \leq q ||M_n||_p.
	\end{align*}
donde $||M_n||_p = (\mathbb{E}[|M_n|^p])^{\frac{1}{p}}$ y $q = \frac{p}{p-1} $.
\end{theorem}

Esta desigualdad nos será de gran ayuda a la hora de demostrar uno de los resultados más importantes en teoría de martingalas, el Teorema de Convergencia en Martingalas. Pero antes de demostrar el Teorema \ref{lp2} verifiquemos los siguientes resultados.

\begin{lemma}
\label{lemdes2}
Sea $X = (X_n, n \geq 0)$ una submartingala y $Y = (Y_n, n \geq 0)$ una supermartingala. Entonces para todo $\lambda > 0$
	\begin{align}
		\lambda  \mathbb{P}(X_n^{*} > \lambda) & \leq \mathbb{E}[X_n^{+} 1_{\{X_n^{*} > \lambda\}}] \label{aaj}, \\
		\lambda  \mathbb{P}(Y_n^{*} > \lambda) & \leq \mathbb{E}[Y_0] - \mathbb{E}[Y_n 1_{\{Y_n^{*} \leq \lambda\}}]. \label{aak}
	\end{align}
\end{lemma}
\begin{proof}
Si definimos a $\tau$ de la siguiente manera
	\begin{align*}
	\tau = \inf \{k \leq n \mid X_k \geq \lambda\}
	\end{align*}
Con $\tau = n$, si $X_n^{*} > \lambda$
Entonces tenemos que, como $X$ es una martingala
	\begin{align*}
	0 \leq \mathbb{E}[X_0^{+}] & \leq \mathbb{E}[X_{\tau}^{+}] \\
	& = \mathbb{E}[X_{\tau}^{+} 1_{\{X_n^{*} \leq \lambda\}}] + \mathbb{E}[X_{\tau}^{+} 1_{\{X_n^{*} > \lambda\}}] \\
	& \leq \mathbb{E}[\lambda  1_{\{X_n^{*} \leq \lambda\}}] + \mathbb{E}[X_{\tau}^{+} 1_{\{X_n^{*} > \lambda\}}] \\
	& = \lambda  \mathbb{P}(X_n^{*} \leq \lambda) + \mathbb{E}[X_n^{+} 1_{\{X_n^{*} > \lambda\}}] \\
	& = \lambda  (1 - \mathbb{P}(X_n^{*} > \lambda)) + \mathbb{E}[X_n^{+} 1_{\{X_n^{*} > \lambda\}}] \\
	& \leq - \lambda \mathbb{P}(X_n^{*} > \lambda) + \mathbb{E}[X_n^{+} 1_{\{X_n^{*} > \lambda\}}].
	\end{align*}

Por lo tanto $\lambda  \mathbb{P}(X_n^{*} > \lambda) \leq \mathbb{E}[X_n^{+} 1_{\{X_n^{*} > \lambda\}}]$. \\

Consideremos ahora el caso donde $Y$ es una supermartingala. Definamos a $\tau$ como $\inf \{k \leq n \mid X_k \geq \lambda\}$, donde $\tau = n$ si $X_n^{*} < \lambda$
	\begin{align*}
	\mathbb{E}[Y_0] & \geq \mathbb{E}[Y_{\tau}] \\
	& = \mathbb{E}[Y_{\tau} 1_{\{Y_n^{*} > \lambda\}}] + \mathbb{E}[Y_{\tau} 1_{\{Y_n^{*} \leq \lambda\}}] \\
	& \geq \mathbb{E}[\lambda  1_{\{Y_n^{*} > \lambda\}}] + \mathbb{E}[Y_{\tau} 1_{\{Y_n^{*} \leq \lambda\}}] \\
	& = \lambda  \mathbb{P}(Y_n^{*} > \lambda) + \mathbb{E}[Y_n 1_{\{Y_n^{*} \leq \lambda\}}].
	\end{align*}
\end{proof}
Con el lema anterior podemos verificar el Teorema \ref{lp2}

\begin{proof}
Verifiquemos el caso para una submartingala positiva. Supongamos en primer lugar que
	\begin{align}
	||M_n^{*}||_p = \mathbb{E}[|M_n^{*}|^{\frac{1}{p}}]^p < \infty. \label{aal}
	\end{align}
Del Lema \ref{repint} tenemos que para toda $p > 1$
	\begin{align*}
	\mathbb{E}[M_n^{*}]^p & = \int_0^{\infty} p \lambda^{p-1} \mathbb{P}(M_n^{n} > \lambda) d\lambda.
	\end{align*}
Del Lema \ref{lemdes2} y como $M$ es una submartingala positiva entonces
	\begin{align*}
	\mathbb{E}[M_n^{*}]^p = \int_0^{\infty} p \lambda^{p-1} \mathbb{P}(M_n^{n} > \lambda) d\lambda  \leq \int_0^{\infty} p \lambda^{p-1} \left( \frac{1}{\lambda} \mathbb{E}[M_n 1_{\{M_n^{*} > \lambda\}}] \right) d\lambda.
	\end{align*}
Utilizando el teorema de Fubini obtenemos las siguientes igualdades
	\begin{align}
	\mathbb{E}[M_n^{*}]^p & \leq \int_0^{\infty} p \lambda^{p-1} \left( \frac{1}{\lambda} \mathbb{E}[M_n 1_{\{M_n^{*} > \lambda\}} ] \right) d\lambda \nonumber \\
	& = p  \int_0^{\infty} \lambda^{p-2} \mathbb{E}[M_n 1_{\{M_n^{*} > \lambda\}}] d\lambda \nonumber \\
	& = p \ \mathbb{E} \left[ M_n \int_0^{M_n^{*}} \lambda^{p-2} d\lambda \right] \nonumber \\
	& = \frac{p}{p-1} \mathbb{E}[M_n (M_n^{*})^{p-1}]. \label{aam}
	\end{align}
De la desigualdad de Hölder tenemos que
	\begin{align*}
	\mathbb{E}[M_n^{*}]^p & \leq \frac{p}{p-1} \mathbb{E}[M_n (M_n^{*})^{p-1}] \\
	& \leq q ||M_n||_p ||(M_n^{*})^{p-1}||_q \\
	& = q ||M_n||_p \mathbb{E}[(M_n^{*})^p]^{\frac{1}{q}}.
	\end{align*}
Por lo tanto
	\begin{align*}
	\frac{\mathbb{E}[M_n^{*}]^p}{\mathbb{E}[(M_n^{*})^p]^{\frac{1}{q}}} & \leq q ||M_n||_p, \\
	||M_n^{*}||_p & \leq q ||M_n||_p.
	\end{align*}
Por otro lado, si (\ref{aal}) no se cumple, entonces en (\ref{aam}) consideramos $(M_n^{*} \wedge C)$ en lugar de $M_n^{*}$ con $C$ como una constante, entonces
	\begin{align*}
		\mathbb{E}[M_n^{*} \wedge C]^p \leq q\mathbb{E}[M_n (M_n^{*} \wedge C)^{p-1}] \leq q ||M_n||_p \mathbb{E}[(M_n^{*} \wedge C)^p]^{\frac{1}{q}}.
	\end{align*}
Entonces, del hecho de que $\mathbb{E}[M_n^{*} \wedge C]^p \leq C^p < \infty$, tenemos
	\begin{align*}
		\mathbb{E}[M_n^{*} \wedge c]^p \leq q^p ||M_n||_p^n.
	\end{align*}
Por lo tanto, 
	\begin{align*}
		\mathbb{E}[M_n^{*}]^p = \lim_{C \rightarrow \infty} \mathbb{E}[M_n^{*} \wedge C]^p \leq q^p\mathbb{E}[M_n]^p
	\end{align*}
Por último, el caso en que $M$ es una martingala se reduce a la demostración anterior ya que $|M|^p$ con $p \geq 1$ es una submartingala positiva por la Proposición \ref{convexa}.
\end{proof}

Ahora introducimos el concepto de \emph{cruces ascendentes}, donde utilizaremos la notación de Doob. Sea $(X_n, n \geq 0)$ una submartingala, y sea $a < b$. El número de cruces ascendentes de un intervalo $[a, b]$ es el número de veces en que el proceso comienza por debajo del valor de $a$ y después de algunos pasos salta a algún valor por encima de $b$. Está noción puede ser expresada más adecuadamente usando tiempos de paro. Definamos
\begin{align*}
	\tau_0 = 0,
\end{align*}
inductivamente para $j \geq 0$:
	\begin{align*}
		\eta_{1} = \min \{k > \tau_0 : X_k \leq a \}, & \hspace{1cm} \tau_{1} = \min \{k > \eta_{1} : b \leq X_k \} \\
		\eta_{2} = \min \{k > \tau_1 : X_k \leq a \}, & \hspace{1cm} \tau_{2} = \min \{k > \eta_{2} : b \leq X_k \} \\
		& \vdots \\
		\eta_{j+1} = \min \{k > \tau_j : X_k \leq a \}, & \hspace{1cm} \tau_{j+1} = \min \{k > \eta_{j+1} : b \leq X_k \}
	\end{align*}
Bajo las siguientes condiciones de que el mínimo de un conjunto vacío es $+ \infty$ y el máximo es $0$, entonces podemos definir
	\begin{align}
		U_n^{[a, b]} = \max \{j : \tau_j \leq n\}, \label{aan}
	\end{align}
y $U_n^{[a, b]}$ es el número de cruces ascendentes de $[a, b]$ antes del tiempo $n$.

\begin{theorem}[Desigualdad de Cruces Ascendentes de Doob]
\label{cruces}
Sea $(X_n, n \geq 0)$ una submartingala, sea $a < b$ y sea $U_n^{[a, b]}$ el número de cruces ascendentes de $[a, b]$ antes del tiempo $n$ definida en (\ref{aan}). Entonces
	\begin{align*}
		\mathbb{E}[U_n^{[a, b]}] \leq \frac{1}{b-a} \mathbb{E}[(X_n - a)^{+}].
	\end{align*}
donde $(X_n - a)^{+} = \max (X_n - a, 0)$
\end{theorem}
\begin{proof}
Consideremos al proceso $Y_n = (X_n - a)^{+}$. Por su construcción, $\eta_{n+1} > n$, notemos que podemos descomponer a $Y_n$ como
	\begin{align}
	Y_n = Y_{\eta_1 \wedge n} + \sum_{i = 1}^n (Y_{\tau_i \wedge n} - Y_{\eta_i \wedge n}) + \sum_{i = 1}^n (Y_{\eta_{i+1} \wedge n} - Y_{\tau_i \wedge n}). \label{aao}
	\end{align}
Recordemos que cada cruce ascendente de $X_n$ entre los tiempos $0$ y $n$ corresponde a un entero $i$ tal que $\eta_i < \tau_i \leq n$. Por definición de $Y_n$ tenemos que $Y_{\eta_i} = 0$ y además $Y_{\tau_i \wedge n} = Y_{\tau_i} \geq b-a$. \\

Además por definición de los tiempos de paro $\eta_i$ y $\tau_i$ tenemos que para toda $i$, $Y_{\tau_i \wedge n} - Y_{\eta_i \wedge n} \geq 0$, entonces
	\begin{align}
	(b - a) U_n^{[a, b]} \leq \sum_{i = 1}^n (Y_{\tau_i \wedge n} - Y_{\eta_i \wedge n}). \label{aap}
	\end{align}
Realizando el correspondiente despeje en (\ref{aao}) y sustituyendo en (\ref{aap}) obtenemos
	\begin{align*}
	(b - a) U_n^{[a, b]} \leq Y_n - Y_{\eta_1 \wedge n} - \sum_{i = 1}^n (Y_{\eta_{i+1} \wedge n} - Y_{\tau_i \wedge n}).
	\end{align*}
Y como $Y_{\eta_1 \wedge n} \geq 0$, entonces
	\begin{align}
	(b - a) U_n^{[a, b]} \leq Y_n - \sum_{i = 1}^n (Y_{\eta_{i+1} \wedge n} - Y_{\tau_i \wedge n}). \label{aaq}
	\end{align}
Tomando la esperanza en ambos lados de (\ref{aaq}) y con el hecho de que $\eta_{i+1}$ y $\tau_i$ están acotadas por $n$ entonces y $(Y_n, n \geq 0)$ es una submartingala ya que $\phi(x) = (x-a)^{+}$ es una función convexa entonces, y además $Y_{\eta_{i+1}} \geq Y_{\tau_i}$, es decir, $\mathbb{E} \left[ \sum_{i = 1}^n (Y_{\eta_{i+1} \wedge n} - Y_{\tau_i \wedge n}) \right] \geq 0$
	\begin{align*}
	(b - a) \mathbb{E}[U_n^{[a, b]}] \leq \mathbb{E}[Y_n] - \mathbb{E} \left[ \sum_{i = 1}^n (Y_{\eta_{i+1} \wedge n} - Y_{\tau_i \wedge n}) \right] \leq \mathbb{E}[Y_n].
	\end{align*}
\end{proof}

\section{Teoremas de Convergencia de Martingalas}
En esta última sección presentamos resultados respecto a la convergencia de martingalas, así como el concepto de \emph{uniformemente integrables} para una colección de variables aleatorias, el cual tiene una relación fuerte con la convergencia de martingalas. Concluimos el primer capítulo con la prueba del Teorema del Límite Central de Martingalas análogo al conocido Teorema del Límite Central, pero mostrando las bondades de trabajar con las propiedades de las martingalas.

\begin{theorem}[Teorema de Convergencia de Martingalas]
\label{conver1}
Sea $(X_n, n \geq 0)$ una submartingala tal que $\sup_n \mathbb{E}[X_n^{+}] < \infty$. Entonces $\lim_{n \rightarrow \infty} X_n = X$ existe c.s. (y es finita c.s.). Más aún, $X \in L^1$.
\end{theorem}
\begin{proof}
Sea $U_n^{[a, b]}$ es número de cruces ascendentes de $[a, b]$ antes del tiempo $n$, como se definió en (\ref{aan}). Por su definición $U_n^{[a, b]}$ es no decreciente y acotado, por lo tanto $U(a, b) = \lim_{n \rightarrow \infty} U_n^{[a, b]}$ existe. Por el Teorema de Convergencia Monótona
	\begin{align*}
	\mathbb{E}[U(a, b)] = \mathbb{E}\left[\lim_{n \rightarrow \infty} U_n^{[a, b]}\right] = \lim_{n \rightarrow \infty} \mathbb{E}[U_n^{[a, b]}].
	\end{align*}
Por el Teorema \ref{cruces} tenemos que 
	\begin{align*}
	\mathbb{E}[U(a, b)] & = \lim_{n \rightarrow \infty} \mathbb{E}[U_n^{[a, b]}] \\
	& \leq \frac{1}{b - a} \sup_n \mathbb{E}[(X_n - a)^{+}].
	\end{align*}
Sabemos además, que para todo real $a, x$ se tiene $(x-a)^{+} \leq x^{+} + |a|$ entonces
	\begin{align*}
	\mathbb{E}[U(a, b)] & = \lim_{n \rightarrow \infty} \mathbb{E}[U_n^{[a, b]}] \\
	& \leq \frac{1}{b - a} \sup_n \mathbb{E}[(X_n - a)^{+}] \\
	& \leq \frac{1}{b - a} \left( \sup_n \mathbb{E}[X_n^{+}] + |a| \right) \leq \frac{c}{b-a} < \infty.
	\end{align*}
Donde $c$ es alguna constante y es finita por las condiciones estipuladas. Más aún, como $\mathbb{E}[U(a, b)] < \infty$ entonces $\mathbb{P}(U(a, b) < \infty) = 1$, es decir, el proceso $X_n$ realiza un número finito de cruces ascendentes casi seguramente.

Veamos que el conjunto de puntos muestrales donde el límite de $X_n$ no existe tiene probabilidad cero. Para esto, consideremos para todo $a < b$
	\begin{align*}
	\Lambda_{a, b} = \left\{ \omega : \liminf_n X_n(\omega) \leq a < b \leq  \limsup_n X_n(\omega) \right\}
	\end{align*}
Como $X_n$ realiza un número finito de cruces ascendentes entonces $\mathbb{P}(\Lambda_{a, b}) = 0$, además si $\Lambda = \bigcup_{a, b \in \mathbb{Q}} \Lambda_{a, b}$ entonces $\mathbb{P}(\Lambda) = 0$. \\

Por otro lado, tenemos que 
	\begin{align*}
	\Lambda & = \bigcup_{{a, b} \in \mathbb{Q}} \Lambda_{a, b} \\
	& = \bigcup_{{a, b} \in \mathbb{Q}} \left\{ \omega : \liminf_n X_n(\omega) \leq a < b \leq  \limsup_n X_n(\omega) \right\} \\
	& = \left\{ \omega : \liminf_n X_n(\omega) < \limsup_n X_n(\omega) \right\}.
	\end{align*}
Por lo tanto, $\mathbb{P}(\liminf_n X_n < \limsup_n X_n ) = 0$. Entonces $\mathbb{P}(\liminf_n X_n = \limsup_n X_n \}) = 1$ concluyendo que $\lim_n X_n$ existe casi seguramente. \\

Veamos que $X = \lim_n X_n \in L^1$. Como $X_n$ es un submartingala entonces tenemos que $\mathbb{E}[X_0] \leq \mathbb{E}[X_n]$, por lo tanto
	\begin{align}
	\mathbb{E}[|X_n|] & = \mathbb{E}[X_n^{+}] + \mathbb{E}[X_n^{-}] \nonumber \\
	& = 2\mathbb{E}[X_n^{+}] - \mathbb{E}[X_n] \nonumber \\
	& \leq 2\mathbb{E}[X_n^{+}] - \mathbb{E}[X_0]. \label{aar}
	\end{align}
Entonces usando el Lema de Fatou \cite[p.~205]{jacodprotter} y nuestra desigualdad (\ref{aar}), además de que por hipótesis $\sup_n \mathbb{E}[X_n^{+}] < \infty$,
	\begin{align*}
	\mathbb{E}\left[\lim_n |X_n|\right] \leq \liminf_{n \rightarrow \infty} \mathbb{E}[|X_n|] \leq 2 \sup_n \mathbb{E}[X_n^{+}] - \mathbb{E}[X_0] < \infty,
	\end{align*}
lo que implica que $X_n$ converge a un límite $X$ finito y además $\mathbb{E}[X] < \infty$, es decir $X \in L^1$.
\end{proof}

\begin{corollary}
Si $X_n$ es una supermartingala no negativa, o una martingala acotada por arriba o acotada por debajo, entonces $\lim_{n \rightarrow \infty} X_n = X$ existe c.s., y $X \in L^1$.
\end{corollary}
\begin{proof}
Si $X_n$ es una supermartingala no negativa tenemos que $(-X_n, n \geq 0)$ es una submartingala acotada por arriba por 0 y se aplica el Teorema \ref{conver1}. \\

Si $(X_n, n \geq 0)$ es una martingala acotada por debajo, entonces $X_n \geq -c$ casi seguramente, para toda $n$ y alguna constante $c > 0$. Definamos $Y_n = X_n + c$, entonces, $Y_n$ es una martingala no negativa y resulta ser una supermartingala no negativa, por lo que tenemos el caso anterior, donde aplicamos de nuevo el Teorema \ref{conver1}. \\

Si $(X_n, n \geq 0)$ es una martingala acotada por arriba, se tiene que $(-X_n, n \geq 0)$ es una martingala acotada por debajo.
\end{proof}

En general la convergencia $L^1$ del proceso $X_n$ a $X$ no se da. Para conseguir ese tipo de convergencia necesitamos hipótesis más fuertes, por lo que debemos introducir una nueva definición.

\begin{definition}
Un conjunto $\mathcal{H}$ de $L^1$ se dice que es una colección de variables aleatorias uniformemente integrables si
	\begin{align*}
	\lim_{c \rightarrow \infty} \sup_{X \in \mathcal{H}} \mathbb{E}[|X| 1_{ \{|X| \geq c\} }] = 0.
	\end{align*}
\end{definition}

Con los siguientes resultados podemos asegurar la condición de integrabilidad a partir de dos condiciones.

\begin{proposition}
Sea $\mathcal{H}$ una clase de variables aleatorias
	\begin{enumerate}
	\item Si $\sup_{X \in \mathcal{H}} \mathbb{E}[|X|^p] < \infty$ para alguna $p > 1$, entonces $\mathcal{H}$ es uniformemente integrable.
	\item Si existe una variable aleatoria $Y$ tal que $|X| < Y$ casi seguramente para toda $X \in \mathcal{H}$ y $\mathbb{E}[Y] < \infty$, entonces $\mathcal{H}$ es uniformemente integrable.
	\end{enumerate}
\end{proposition}

\begin{proof}
	\begin{enumerate}
	\item Como $\sup_{X \in \mathcal{H}} \mathbb{E}[|X|^p] < \infty$, consideremos a $K$ como la constante tal que $\sup_{X \in \mathcal{H}} \mathbb{E}[|X|^p] < K < \infty$. Si $0 < c \leq x$, entonces como $p > 1$ se tiene que $x^{1-p} \leq c^{1-p}$, si multiplicamos por $x^p$ obtenemos $x \leq c^{1-p}x^p$, por lo tanto
		\begin{align*}
		|X| 1_{\{|X| > c\}} \leq c^{1-p}  |X|^p 1_{\{|X| > c\}}.
		\end{align*}
	Y entonces,
		\begin{align*}
		\mathbb{E}\left[ |X| 1_{\{|X| > c\}} \right] \leq c^{1-p}  \mathbb{E} \left[ |X|^p 1_{\{|X| > c\}} \right] \leq \frac{k}{c^{p-1}},
		\end{align*}
	si tomamos el supremo sobre toda $X \in \mathcal{H}$ y hacemos que $n \rightarrow \infty$, tenemos que
		\begin{align*}
		\lim_{c \rightarrow \infty} \sup_{X \in \mathcal{H}} \mathbb{E} \left[ |X| 1_{ \{|X| > c\} } \right] \leq \lim_{c \rightarrow \infty} \frac{k}{c^{p-1}} = 0.
		\end{align*}
		
	\item Como $|X| \leq Y$ casi seguramente, para toda $X \in \mathcal{H}$, se tiene
		\begin{align*}
		|X| 1_{ \{|X| > c\} } \leq Y 1_{ \{Y > c\} }.
		\end{align*}
	Tomando el supremo sobre toda $X \in \mathcal{H}$ y el límite en ambos lados de la desigualdad tenemos
		\begin{align*}
		\lim_{c \rightarrow \infty} \sup_{X \in \mathcal{H}} \mathbb{E} \left[ |X| 1_{ \{|X| > c\} } \right] \leq \lim_{c \rightarrow \infty} \mathbb{E}\left[Y 1_{ \{Y > c\} }\right].
		\end{align*}
	Por otro lado, tenemos que $\lim_{c \rightarrow \infty} Y 1_{ \{Y > c\} } = 0$ c.s. y por el Teorema de Convergencia Dominada de Lebesgue obtenemos
		\begin{align*}
		\lim_{c \rightarrow \infty} \sup_{X \in \mathcal{H}} \mathbb{E} \left[ |X| 1_{ \{|X| > c\} } \right] & \leq \lim_{c \rightarrow \infty} \mathbb{E} \left[ Y 1_{ \{Y > c\} } \right] = \mathbb{E} \left[ \lim_{c \rightarrow \infty} Y 1_{ \{Y > c\} } \right] = 0.
		\end{align*}
	\end{enumerate}
\end{proof}

El siguiente resultado es una versión más fuerte del Teorema \ref{conver1} para el caso de martingalas.

\begin{theorem}
	\begin{enumerate}
	\item Sea $(M_n, n \geq 0)$ una martingala y supongamos que $(M_n, n \geq 0)$ es una colección de variables aleatorias uniformemente integrables. Entonces
	\begin{align*}
	\lim_{n \rightarrow \infty} M_n = M_{\infty} \text{ existe c.s. }
	\end{align*}
	$M_{\infty}$ está en $L^1$, y $M_n$ converge a $M_{\infty}$ en $L^1$. Más aún, la propiedad de martingala se mantiene para $M_{\infty}$, es decir, $M_n = \mathbb{E} \left[ M_{\infty} \mid \mathcal{F}_n \right]$.
	
	\item Sea $Y \in L^1$ y considere la martingala $M_n = \mathbb{E} \left[ Y \mid \mathcal{F}_n \right]$. Entonces $(M_n, n \geq 1)$ es una colección de variables aleatorias uniformemente integrables.
	\end{enumerate}
\end{theorem}
\begin{proof}
\begin{enumerate}
\item De la definición de uniformemente integrable para variables aleatorias, tenemos que, para toda $\epsilon > 0$ existe una constante $c > 0$ tal que $\sup_{X \in \mathcal{H}} \mathbb{E} [ 1_{ \{|X| > c\} } |X| ] \leq \epsilon$. Entonces
	\begin{align*}
	\mathbb{E}[|M_n|] = \mathbb{E} \left[ |M_n|1_{ \{|M_n| > c\} } \right] + \mathbb{E} \left[ |M_n|1_{ \{|M_n| \leq c\} } \right] \leq \epsilon + c.
	\end{align*}
Por lo tanto $(M_n)_{n \geq 0}$ está en acotado en $L^1$, es por esta razón que se tiene $\sup_n \mathbb{E}[M_n^{+}] < \infty$. Utilizando el Teorema \ref{conver1} 
	\begin{align*}
	\lim_{n \rightarrow \infty} M_n = M_{\infty} \text{ existe c.s. y } M_{\infty} \text{ está en } L^1.
	\end{align*}
Para mostrar la convergencia del proceso a la variables aleatoria $M_{\infty}$ en $L^1$ veamos que para una constante $N$ suficientemente grande $\mathbb{E}[|M_n - M_{\infty}|] < \epsilon$ para toda $n \geq N$. \\

Recordemos que una función cumple la condición de Lipschitz si \cite[p.~169]{bartle} dados dos espacios métricos $(X, d_X)$ y $(Y, d_Y)$ existe una constante $K$ tal que
	\begin{align*}
	\frac{d_Y(f(x) - f(y))}{d_X(x - y)} \leq K.
	\end{align*}
Si definimos 
	\begin{align*}
	f_c(x) = 
	\begin{cases}
	c, & \text{ si } c > x; \\
	x, & \text{ si } |x| \leq c; \\
	-c, & \text{ si } x < -c.
	\end{cases}
	\end{align*}
Tenemos que $f_c(x)$ es Lipschitz. Por la integrabilidad uniforme sabemos que existe $c$ suficientemente grande tal que para cualquier $\epsilon > 0$ dada:
	\begin{align}
	\mathbb{E}[ |f_c(M_n) & - M_n|] < \frac{\epsilon}{3}, \text{ para toda } n; \label{aas} \\
	\mathbb{E}[ |f_c(M_{\infty}) & - M_{\infty}|] < \frac{\epsilon}{3}. \label{aat}
	\end{align}
	
Como $\lim_n M_n = M_{\infty}$ y por la condición de $f_c$ tenemos que $\lim_n f_c(M_n) = f_c(M_{\infty})$. Por el Teorema de Convergencia Dominada de Lebesgue \cite[p.~52]{jacodprotter} tenemos que para cualquier $n \geq N$, para $N$ suficientemente grande
	\begin{align}
	\mathbb{E} [ | f_c(M_n) & - f_c(M_{\infty}) |] < \frac{\epsilon}{3}. \label{aau}
	\end{align}

Combinando los resultados (\ref{aas}), (\ref{aat}) y (\ref{aau}) obtenemos la desigualdad requerida para asegurar que $M_n \xrightarrow{L^1} M_{\infty}$. Para demostrar que $\mathbb{E}\left[M_{\infty} \mid \mathcal{F}_n\right] = M_n$ c.s. veamos que para cualquier conjunto $\mathcal{F}_n$-medible se tiene $\mathbb{E}[M_n 1_{\Lambda}] = \mathbb{E}[M_{\infty} 1_{\Lambda}]$. Consideremos $\Lambda \in \mathcal{F}_m$ y $n \geq m$. Entonces por la propiedad de martingala tenemos que
	\begin{align*}
	\mathbb{E}[M_n 1_{\Lambda}] = \mathbb{E}[M_m 1_{\Lambda}].
	\end{align*}
Sin embargo, 
	\begin{align*}
	|\mathbb{E}[M_m 1_{\Lambda}] - \mathbb{E}[M_{\infty} 1_{\Lambda}]| & \leq |\mathbb{E}[(M_m - M_{\infty}) 1_{\Lambda}]| \\
	& \leq \mathbb{E}[|(M_m - M_{\infty}) 1_{\Lambda}|] \\
	& \leq \mathbb{E}[|M_m - M_{\infty}|],
	\end{align*}
Donde $\mathbb{E}[|M_n - M_{\infty}|] \rightarrow 0$ cuando $n \rightarrow 0$, por lo tanto, $\mathbb{E}[M_{\infty} \mid \mathcal{F}_m] = M_m$ c.s.

\item Sabemos $(M_n, n \geq 0)$ es una martingala. Para una variable $Y \in L^1$, con $c > 0$ tenemos que
	\begin{align*}
	M_n 1_{\{|M_n > c|\}} = \mathbb{E}\left[Y1_{\{|M_n > c|\}} \mid \mathcal{F}_n\right],
	\end{align*}
ya que el evento $\{|M_n| > c\}$ pertenece a $\mathcal{F}_n$. Por lo tanto, para cualquier constante $d > 0$ se tiene
	\begin{align*}
	\mathbb{E}\left[|M_n| 1_{\{|M_n > c|\}}\right] & = \mathbb{E}\left[|Y|1_{\{|M_n > c|\}}\right] \\
	& \leq \mathbb{E}\left[|Y|1_{\{|Y > d|\}} \right] + d\mathbb{P}(|M_n| > c) \\
	& \leq \mathbb{E}\left[|Y|1_{\{|Y > d|\}} \right] + \frac{d}{c}\mathbb{E}[|M_n|].
	\end{align*}
Si tomamos un $\epsilon$ arbitrario y escogemos a la constante $d$ tal que 
	\begin{align*}
	\mathbb{E}\left[|Y|1_{\{|Y > d|\}}\right] < \frac{\epsilon}{2}.
	\end{align*}
Y $c$ como la constante tal que 
	\begin{align*}
	\frac{d}{c} \mathbb{E}[|M_n|] < \frac{\epsilon}{2},
	\end{align*}
	entonces tenemos que $\mathbb{E}[|M_n| 1_{\{|M_n > c|\}}] < \epsilon$ para todo $n$.
\end{enumerate}
\end{proof}

La propiedad de martingala estudiada hasta el momento contempla números enteros positivos, pero también podemos considerar el conjunto de índices $-\mathbb{N}$: los enteros negativos. Una martingala reversible es una sucesión $(X_{-n}, n \in \mathbb{N})$ de variables aleatorias integrables si, $X_{-n}$ es $\mathcal{F}_{-n}$ medible y satisface
	\begin{align}
	\mathbb{E}[X_{-n} \mid \mathcal{F}_{-m}] = M_{-m}, \hspace{0.3cm} \text{ c.s } \label{aav}
	\end{align}
donde $0 \leq n < m$.

\begin{theorem}[Teorema de Convergencia de Martingalas Reversibles]
\label{conver2}
Sea $(X_{-n}, \mathcal{F}_{-n})_{n \in \mathbb{N}}$ una martingala reversible, y sea $\mathcal{F}_{-\infty} = \cap_{n = 0}^{\infty} \mathcal{F}_{-n}$. Entonces la sequencia $(X_{-n})$ converge c.s. y en $L^1$ al límite $X$ c.s., cuando $n \rightarrow +\infty$ (en particular $X$ es c.s. finita e integrable).
\end{theorem}
\begin{proof}
Sea $U_{-n}^{[a, b]}$ es el número de cruces ascendentes de $(X_{-n}, n \geq 0)$ de $[a, b]$ entre el tiempo $-n$ y $1$. Entonces $U_{-n}^{[a, b]}$ es creciente mientras $n$ crece, además consideremos $U^{-}(a, b) = \lim_n U_{-n}^{[a, b]}$, el cual existe por ser $U_{-n}^{[a, b]}$ una sucesión creciente y acotada. Por el Teorema de Convergencia Monótona
	\begin{align*}
	\mathbb{E}[U^{-}(a, b)] & = \lim_{n \rightarrow \infty} \mathbb{E}[U_{-n}^{[a, b]}] \leq \frac{1}{b-a} \mathbb{E}[(-X_0 - a)^{+}] < \infty.
	\end{align*}
Entonces $\mathbb{P}(U^{-}(a, b) < \infty) = 1$. El mismo argumento de los cruces ascendentes que se utilizó en la prueba del Teorema \ref{conver1} implica que el $\lim_{n \rightarrow \infty} X_{-n} = X$ existe c.s. \\

Sea $\phi(x) = x^{+} = (x \vee 0)$, la cual es una función convexa y creciente, entonces $\phi(X_{-n})$ es integrable para toda $n$. De la desigualdad de Jensen y (\ref{aav}) se infiere que 
	\begin{align*}
	X_{-n}^{+} \leq \mathbb{E}[X_0^{+} \mid \mathcal{F}_{-n}].
	\end{align*}
Por lo tanto
	\begin{align*}
	\mathbb{E}[X_{-n}^{+}] \leq \mathbb{E}[X_0^{+}].
	\end{align*}
Del Lema de Fatou y por el hecho de que $X_{-n}^{+} \geq 1$ y $X_{-n}^{+} \rightarrow X^{+}$ c.s. se tiene
	\begin{align*}
	\mathbb{E}[X^{+}] \leq \liminf_n \mathbb{E}[X_{-n}^{+}] \leq \mathbb{E}[X_0^{+}] < \infty.
	\end{align*}
Implicando que $X^{+} \in L^1$ y por el mismo argumento aplicado a la martingala $(-X_n)$ se muestra que $X^{-} \in L^1$, entonces $X \in L^1$. \\

La convergencia en $L^1$ es una implicación del Teorema \ref{conver1}, pues se demostró que, si $X_{-n} \rightarrow X$ c.s., si $X \in L^1$ y además $(X_{-n})$ es uniformemente integrable, entonces $X_{-n} \rightarrow X$ en $L^1$.
\end{proof}

Los Teoremas de Convergencia de Martingalas probados hasta ahora (Teoremas \ref{conver1} y \ref{conver2}) son resultados de convergencia fuerte: todas las variables aleatorias están definidas en el mismo espacio de probabilidad y convergen fuertemente a variables aleatorias del mismo espacio, casi seguramente y en $L^1$. \\ 

Es posible desarrollar un resultado de convergencia débil, para una clase de martingalas que no satisfacen las condiciones mencionadas en el Teorema \ref{conver1}. El límite resulta una distribución normal  y tal teorema es conocido como \emph{Teorema del Límite Central para Martingalas}.

\begin{theorem}[Teorema del Límite Central para Martingalas]
\label{conver3}
Sea $(X_n, n \geq 1)$ una sucesión de variables aleatorias que satisface
	\begin{enumerate}
	\item $\mathbb{E}[X_n \mid \mathcal{F}_{n-1}] = 0$.
	\item $\mathbb{E}[X_n^2 \mid \mathcal{F}_{n-1}] = 1$.
	\item $\mathbb{E}[|X_n|^3 \mid \mathcal{F}_{n-1}] \leq K < \infty$.
	\end{enumerate}
Sea $S_n = \sum_{i=1}^n X_i$ con $S_0 = 0$. Entonces $\lim_{n \rightarrow \infty} \frac{1}{\sqrt{n}} S_n = Z$, donde $Z$ es una variable aleatoria con distribución $N(0, 1)$, además la convergencia es en distribución.
\end{theorem}
\begin{proof}
Para verificar la convergencia en distribución de la proposición haremos uso de las funciones características. Para $u \in \mathbb{R}$, recordemos que $\varphi_X (u) = \mathbb{E}[e^{iuX}]$ es la función característica de $X$. Definamos la siguiente función  como
	\begin{align*}
	\varphi_{n, j} (u) = \mathbb{E}\left[e^{iu \frac{1}{\sqrt{n}} X_j} \bigg| \mathcal{F}_{j-1}\right].
	\end{align*}
Por el Teorema de Taylor \cite[p.~358]{apostol} tenemos
	\begin{align}
	e^{iu \frac{1}{\sqrt{n}} X_j} = 1 + iu \frac{1}{\sqrt{n}} X_j - \frac{u^2}{2n} X_j^2 - \frac{iu^3}{6n^{\frac{3}{2}}} \bar{X}_j^3, \label{aaw}
	\end{align}
donde $\bar{X}_j$ es un valor (aleatorio) que se encuentra entre $1$ y $X_j$. Al tomar la esperanza condicional de ambos lados de (\ref{aaw}) obtenemos
	\begin{align*}
	\varphi_{n, j} (u) = 1 + iu \frac{1}{\sqrt{n}} \mathbb{E}[X_j \mid \mathcal{F}_{j-1}] - \frac{u^2}{2n} \mathbb{E}[X_j^2 \mid \mathcal{F}_{j-1}] - \frac{iu^3}{6n^{\frac{3}{2}}} \mathbb{E}[\bar{X}_j^3 \mid \mathcal{F}_{j-1}].
	\end{align*}
Haciendo uso de las hipótesis (1) y (2) tenemos,
	\begin{align}
	\varphi_{n, j} (u) = 1 - \frac{u^2}{2n} - \frac{iu^3}{6n^{\frac{3}{2}}} \mathbb{E}[\bar{X}_j^3 \mid \mathcal{F}_{j-1}]. \label{aax}
	\end{align}
Además como $|\bar{X}_j| \leq |X_j|$ y $S_p = \sum_{i=1}^p X_i$, para $1 \leq p \leq n$ tenemos
	\begin{align}
	\mathbb{E}\left[e^{iu \frac{1}{\sqrt{n}} S_p} \right] & = \mathbb{E}\left[e^{iu \frac{1}{\sqrt{n}} (S_{p-1} + X_p) } \right] \label{aay} \\
	& = \mathbb{E}\left[e^{iu \frac{1}{\sqrt{n}} S_{p-1}} e^{iu \frac{1}{\sqrt{n}} X_p} \right] \nonumber \\ 
	& = \mathbb{E}\left[e^{iu \frac{1}{\sqrt{n}} S_{p-1}} \mathbb{E}\left[e^{iu \frac{1}{\sqrt{n}} X_p} \bigg| \mathcal{F}_{p-1} \right]\right] \nonumber \\ 
	& = \mathbb{E}\left[e^{iu \frac{1}{\sqrt{n}} S_{p-1}} \varphi_{n, p} (u) \right] \nonumber.
	\end{align}
Usando (\ref{aay}) y (\ref{aax}) tenemos
	\begin{align*}
	\mathbb{E}\left[e^{iu \frac{1}{\sqrt{n}} S_p} \right] = \mathbb{E}\left[e^{iu \frac{1}{\sqrt{n}} S_{p-1}} \left[ 1 - \frac{u^2}{2n} - \frac{iu^3}{6n^{\frac{3}{2}}} \mathbb{E}[\bar{X}_j^3 \mid \mathcal{F}_{j-1}] \right]\right].
	\end{align*}
Desarrollando la ecuación,
	\begin{align}
	\mathbb{E}\left[e^{iu \frac{1}{\sqrt{n}} S_p} - \left(1 - \frac{u^2}{2n}\right)e^{iu \frac{1}{\sqrt{n}} S_{p-1}}\right] = \mathbb{E}\left[e^{iu \frac{1}{\sqrt{n}} S_{p-1}} \frac{iu^3}{6n^{\frac{3}{2}}} \mathbb{E}[\bar{X}_j^3 \mid \mathcal{F}_{j-1}] \right]. \label{aaz}
	\end{align}
Tomando el módulo en ambos lados de (\ref{aaz}) y usando la hipótesis (3)
	\begin{align*}
	\left| \mathbb{E} \left[e^{iu \frac{1}{\sqrt{n}} S_p} - \left(1 - \frac{u^2}{2n}\right) e^{iu \frac{1}{\sqrt{n}} S_{p-1}} \right]\right| & \leq \mathbb{E} \left[ | e^{iu \frac{1}{\sqrt{n}} S_{p1}} | \frac{|u|^3}{6n^{\frac{3}{2}}} \mathbb{E}[| \bar{X}_j^3 | \mid \mathcal{F}_{j-1}] \right] \\
	& \leq K \frac{|u|^3}{6n^{\frac{3}{2}}}.
	\end{align*} 
Si fijamos el valor de $u \in \mathbb{R}$ y hacemos tender a $n \rightarrow \infty$, eventualmente $n \geq u^2/2$ y es por esto que para una $n$ suficientemente grande tenemos que $0 \leq 1 - \frac{u^2}{2} \leq 1$. Si multiplicamos el lado izquierdo de desigualdad anterior por $(1 - u^2/2)^{n-p}$ para una $n$ suficientemente grande
	\begin{align*}
	\left| \left(1 - \frac{u^2}{2}\right)^{n-p} \mathbb{E} \left[e^{iu \frac{1}{\sqrt{n}} S_p} - \left(1 - \frac{u^2}{2n}\right)^{n-p+1}\right]\mathbb{E}\left[e^{iu \frac{1}{\sqrt{n}} S_{p-1}} \right] \right| \leq K \frac{|u|^3}{6n^{\frac{3}{2}}}.
	\end{align*}
Para finalizar, al usar la propiedad telescópica de sumas (finitas) observamos que 
	\begin{align*}
	& \mathbb{E} \left[e^{iu \frac{1}{\sqrt{n}} S_n} \right] - \left(1 - \frac{u^2}{2n}\right)^{n} \\ 
	& = \sum_{p=1}^{n} \left(1 - \frac{u^2}{2n}\right)^{n-p} \mathbb{E} \left[e^{iu \frac{1}{\sqrt{n}} S_p} \right] - \left(1 - \frac{u^2}{2n}\right)^{n-(p-1)} \mathbb{E} \left[e^{iu \frac{1}{\sqrt{n}} S_{p-1}} \right].
	\end{align*}
Usando la desigualdad del triángulo y de la última desigualdad tenemos que (siempre para $n \geq 2/u^2$)
	\begin{align}
	\left| \mathbb{E} \left[e^{iu \frac{1}{\sqrt{n}} S_n} \right] - \left(1 - \frac{u^2}{2n}\right)^{n} \right| \leq n K \frac{|u|^3}{6n^{\frac{3}{2}}} = K \frac{|u|^3}{6\sqrt{n}}. \label{aba}
	\end{align}
Como el lado derecho de (\ref{aba}) tiende a $0$ y además
	\begin{align}
	\lim_{n \rightarrow \infty} \left(1 - \frac{u^2}{2n}\right)^{n} = e^{-\frac{u^2}{2}}.
	\end{align}
Usando la regla de L'Hôpital \cite[p.~215]{bartle}, tenemos
	\begin{align}
	\lim_{n \rightarrow \infty}\mathbb{E} \left[e^{iu \frac{S_n}{\sqrt{n}}} \right] = e^{-\frac{u^2}{2}}.
	\end{align}

Por el Teorema de Continuidad de Lévy \cite[p.~166]{jacodprotter}, el cual nos permite relacionar la convergencia puntal (de las funciones características) con la convergencia en distribución, tenemos que $S_n/\sqrt{n}$ converge en distribución  $Z$, donde la función característica de $Z$ es $e^{-(u^2)/2}$, la cual es una función característica de una variable aleatoria con distribución $N(0, 1)$.
\end{proof}

La proposición anterior se establece de forma similar en el Teorema del Límite Central \cite[p.~181]{jacodprotter} para variables independientes e idénticamente distribuidas $X_n$, con sumas parciales $S_n$. La condición $(1)$ implica que $(S_n)$ es una martingala, pues $X_n = S_n - S_{n-1}$. 

% Por otro lado, un martingala arbitraria $(S_n)$ es una sucesión de sumas parciales asociada con las variables aleatorias $X_n = S_n - S_{n-1}$ y esto también satisface $(1)$.

% Si $S_n$ es la martingala del Teorema \ref{conver1}, sabemos que no puede existir una convergencia fuerte ya que si tenemos $\lim_n S_n = S$ c.s. con $S$ en $L^1$ entonces podríamos tener $\lim_n S_n/\sqrt{n} = 0$ c.s., y la convergencia débil de $S_n/\sqrt{n}$ a una variable aleatoria normal no sería posible. \\

% Lo que no hace posible tener una convergencia fuerte en martingala es el comportamiento de las varianzas condicionales de los incrementos de la martingala $X_n$ (hipótesis (2) del Teorema \ref{conver3}).




\chapter{Problema de Paro Óptimo: Tiempo Discreto}
El propósito principal del presente capítulo es mostrar algunos resultados básicos de la teoría general de paro óptimo. Se estudiarán dos enfoques para resolver el problema de paro óptimo, una aproximación con martingalas y otra con cadenas de Markov, ambos a tiempo discreto. 

% El estudio del problema de paro óptimo desde un punto de vista discreto, nos proveerá de mucha intuición para comprender el caso general a tiempo continuo.

\section{Enfoque con Martingalas}
Interpretaremos al proceso $G_n$ como la ganancia obtenida si la observación de $G$ es parada al tiempo $n$. Supongamos que nuestro proceso es adaptado a la filtración $(\mathcal{F}_n, n \geq 1)$ en el sentido de que, para toda $n$, la observación $G_n$ es $\mathcal{F}_n$-medible. \\

Considere a la $\sigma$-álgebra $\mathcal{F}_n$ como la información del proceso $G$, disponible a tiempo n. Se desea tomar una decisión con respecto al proceso, en este caso, detenerse o no; esta resolución deberá estar basada solamente en $\mathcal{F}_n$, es decir, no es posible realizar suposición alguna con respecto a $G$ a un tiempo mayor a $n$. \\
% Podemos interpretar a $\mathcal{F}_n$ como la información disponible a tiempo $n$. Para cualquier decisión tomada con respecto al tiempo óptimo de paro, ésta deberá ser basada solamente en la información disponible hasta ese momento (ninguna suposición o anticipación está permitida). 

Definamos nuestro problema de una manera más formal. Sea una sucesión adaptada de variables aleatorias no negativas $G = (G_n, n \geq 1)$ definidas en un espacio de probabilidad filtrado $(\Omega, \mathcal{F}, (\mathcal{F}_n)_{n \geq 1}, \mathbb{P})$. Sea $\mathcal{M}_\tau$ el conjunto de tiempos de paro respecto a $(\mathcal{F}_n, n \geq 1)$ finitos casi seguramente, los cuales son mayores a $\tau$ (el cual puede ser un tiempo de paro). Estamos interesados en resolver el siguiente problema de paro óptimo
	\begin{align}
	V_n = \sup_{\tau \in \mathcal{M}_n} \mathbb{E}[G_\tau]. \label{abb}
	\end{align}
Notemos que el problema de paro óptimo involucra dos tareas: mostrar el tiempo de paro que optimiza la esperanza (si es que existe) y calcular el valor óptimo para $V_n$. \\

Para asegurar la existencia de $\mathbb{E}[G_\tau]$ en (\ref{abb}) necesitamos agregar un supuesto adicional
	\begin{align}
	\mathbb{E} \left[ \sup_n G_n \right] < \infty. \label{abc}
	\end{align}
Este supuesto nos asegura, uniformemente, que todas las esperanzas son finitas.  \\

Para poder tomar la decisión de detenerse o no, se debería considerar dos aspectos; la ganancia si se decide detenerse y los retornos esperados al momento de la decisión en caso de que no se desee detenerse, en otras palabras, supongamos que no hemos detenido el proceso al momento $n$, entonces, deberíamos observar todas las cantidades $\mathbb{E}[G_\tau]$ para aquellos tiempos de paro $\tau \in \mathcal{M}_n$, los cuales pueden ser vistos como los potenciales retornos al no habernos detenido a tiempo $n$, y compararlos con el retorno $G_n$, el cual representa la ganancia que obtendríamos si nos detenemos en ese momento. \\

Sería deseable poder observar el valor de $\sup_{\tau \in \mathcal{M}_n} \mathbb{E}[G_\tau \mid \mathcal{F}_n]$, sin embargo, una dificultad surge al tomar el supremo de las esperanzas condicionales en (\ref{abc}) sobre un conjunto no numerable de valores $\tau$ en $\mathcal{M}_n$, el valor $\sup_{\tau \in \mathcal{M}_n} \mathbb{E}[G_\tau \mid \mathcal{F}_n]$ no está definido como una función medible pues cada objeto en el supremo es una variable aleatoria. \\

\subsection{Supremo Esencial}

Para poder resolver este problema, es necesario introducir el concepto de supremo esencial. El siguiente teorema expresa la definición y funcionalidad del supremo esencial de variables aleatorias.

\begin{theorem}[Supremo Esencial]
Sea $(Z_\alpha : \alpha \in I)$ una colección de variables aleatorias definidas en un espacio de probabilidad $(\Omega, \mathcal{F}, \mathbb{P})$, donde $I$ es un conjunto arbitrario de índices. Entonces existe un conjunto numerable $J \subset I$ tal que la variable aleatoria $Z^{*} : \Omega \rightarrow \bar{\mathbb{R}} : = \mathbb{R} \cup \{- \infty, \infty\}$ definida por
	\begin{align*}
	Z^{*} = \sup_{\alpha \in J} Z_\alpha,
	\end{align*}
satisface lo siguiente
	\begin{enumerate}
	\item $\mathbb{P}(Z_\alpha \leq Z^{*}) = 1, \hspace{0.5cm} \forall \alpha \in I$,
	\item Si $Y : \Omega \rightarrow \bar{\mathbb{R}}$ es otra variable aleatoria que satisface (1), entonces
	\begin{align*}
	\mathbb{P}(Z^{*} \leq Y) = 1,
	\end{align*}
	\end{enumerate}
\end{theorem}

La variable $Z^{*}$ es conocida como el \emph{supremo esencial} de $\{Z_\alpha : \alpha \in I\}$, y la representamos como $\esssup_{\alpha \in I} Z_\alpha$. Está determinada únicamente por las propiedades $(1)$ y $(2)$ excepto para un conjunto $\mathbb{P}-$nulo.

\begin{proof}
Sin perdida de generalidad supongamos que $Z^{*} : \Omega \rightarrow [-1, 1]$, es decir
	\begin{align*}
	|Z_\alpha| \leq 1, \hspace{0.5cm} \forall \alpha \in I.
	\end{align*}
De otra forma, se puede utilizar la biyección $f(x) = \frac{2}{\pi} \arctan(x)$ que mapea al conjunto $\mathbb{R}$ al intervalo $[-1, 1]$. \\

\noindent Denotemos a $\mathcal{C}$ como la familia de conjuntos numerables de $I$
	\begin{align*}
	\mathcal{C} = \{C \subset I \mid C \text{ es numerable}\}.
	\end{align*}
% Consideremos la siguiente notación, para cualquier conjunto $C \in \mathcal{C}$
%	\begin{align*}
%	\ell_C = \sup_{\alpha \in C} Z_\alpha.
%	\end{align*}
Si tomamos una sucesión creciente $(C_n : n \geq 1)$ en $\mathcal{C}$ de tal manera que
	\begin{align*}
	a = \sup_{C \in \mathcal{C}} \mathbb{E} \left[ \sup_{\alpha \in C} Z_{\alpha} \right] = \sup_{n \geq 1} \mathbb{E} \left[ \sup_{\alpha \in C_n} Z_{\alpha} \right] \in [-1, 1]
	\end{align*}
Entonces, $\cup_{n \geq 1} C_n$ es un conjunto numerable, pues es la unión de conjuntos numerables. Por lo tanto, podemos definir la variable aleatoria
	\begin{align*}
	Z^{*} = \sup_{\alpha \in J} Z_\alpha,
	\end{align*}
con $J = \cup_n C_n$. Verifiquemos ahora que nuestra variable aleatoria cumple con las propiedades $(1)$ y $(2)$. \\

\noindent $(1)$ Si consideramos un índice arbitrario $\alpha$ tal que $\alpha \in J$, entonces es claro que la definición de $Z^{*} = \sup_{\alpha \in J} Z_\alpha$ cumple con las propiedades $(1)$ y $(2)$ pues $J$ es un conjunto numerable de $I$. Por otra parte, si se considera un índice $\beta \in I \setminus J$ tal que
	\begin{align}
	\mathbb{P}(Z_\beta > Z^{*}) > 0, \label{abd}
	\end{align}
Entonces, para los índices $\{ \beta : Z^{*} < Z_\beta\}$ se tiene que $\mathbb{E}[Z^{*}, Z^{*} < Z_\beta] < \mathbb{E}[Z_\beta, Z^{*} < Z_\beta]$. Del Teorema de Convergencia Monótona tenemos que 
	\begin{align*}
	a = \sup_{n \geq 1} \mathbb{E} \left[ \sup_{\alpha \in C_n} Z_{\alpha},  Z^{*} < Z_\beta \right] & = \lim_n \mathbb{E}\left[ \sup_{\alpha \in C_n} Z_{\alpha},  Z^{*} < Z_\beta \right] \\
    & = \mathbb{E}\left[ \sup_{\alpha \in J} Z_\alpha,  Z^{*} < Z_\beta \right] \\ 
    & = \mathbb{E}[Z^{*},  Z^{*} < Z_\beta] \\
    & < \mathbb{E}[Z_\beta,  Z^{*} < Z_\beta]
	\end{align*}
Además, se tiene que $\mathbb{E}[\max (Z_\beta, Z^{*}),  Z^{*} < Z_\beta] = \mathbb{E}[Z_\beta,  Z^{*} < Z_\beta]$, combinando estos resultados obtenemos
	\begin{align}
	a < \mathbb{E}[\max (Z_\beta, Z^{*}),  Z^{*} < Z_\beta]. \label{abe}
	\end{align}
Por otro lado, como $J \cup \{\beta\}$ es un conjunto que pertenece a $\mathcal{C}$ tenemos 
	\begin{align}
	\mathbb{E} \left[ \max (Z_\beta, Z^{*}),  Z^{*} < Z_\beta \right] & = \mathbb{E} \left[ \sup_{\alpha \in J \cup \{\beta\}} Z_\alpha,  Z^{*} < Z_\beta \right] \nonumber \\
    & \leq \sup_{C \in \mathcal{C}} \mathbb{E} \left[ \sup_{\alpha \in C} Z_{\alpha} \right] = a. \label{abf}
	\end{align}
De (\ref{abe}) y (\ref{abf}) tenemos
	\begin{align*}
	a < \mathbb{E} \left[ \max (Z_\beta, Z^{*}),  Z^{*} < Z_\beta \right] \leq a.
	\end{align*}
La contradicción surge al suponer (\ref{abd}), por lo tanto $\mathbb{P}(Z_\alpha \leq Z^{*}) = 1$ para toda $\alpha \in I$. \\

\noindent $(2)$ Para mostrar la segunda condición, consideremos una variable aleatoria $Y$ que satisface $(1)$, es decir, para toda $\alpha \in I$ se tiene que
	\begin{align}
	\mathbb{P}(Z_\alpha \leq Y) = 1. \label{abg}
	\end{align}
En particular, la condición (\ref{abg}) se cumple para toda $\alpha \in J$, donde $J$ es un conjunto numerable, por lo tanto
	\begin{align*}
	Z^{*} = \sup_{\alpha \in J} Z_\alpha \leq Y, \ \mathbb{P}\text{-c.s. } 
	\end{align*}
Es decir, $\mathbb{P}(Z^{*} \leq Y) = 1$ se cumple.
\end{proof}

El siguiente corolario muestra que bajo ciertas condiciones, podemos tomar variables aleatorias $Z_{\alpha_{n}}$, con índices $\alpha_{n}$ que hagan converger a la sucesión a la variable $Z^{*}$

\begin{corollary}
\label{coroesssup}
Si la familia $(Z_\alpha : \alpha \in I)$ cumple que para cualquier $\alpha$, $\beta \in I$ existe un $\gamma \in I$ tal que $Z_\alpha \vee Z_\beta \leq Z_\gamma$ c.s., entonces el conjunto numerable $J = \{\alpha_0, \alpha_1, \alpha_2, \cdots\}$ puede ser escogido de tal manera que
	\begin{align*}
	Z^{*} = \lim_{n \uparrow \infty} Z_{\alpha_n}, \hspace{0.3cm} \mathbb{P}\text{-c.s.}
	\end{align*}
donde la sucesión $( Z_{\alpha_n})_{n \geq 0}$ es creciente casi seguramente.
\end{corollary}
\begin{proof}
Supongamos que $J = \{\alpha_0, \alpha_1, \alpha_, \cdots\}$ es el conjunto numerable del teorema anterior, entonces podemos reemplazar a $J$ por una nueva sucesión
	\begin{align*}
	J^{*} = \{\alpha_0^{*}, \alpha_1^{*}, \alpha_2^{*}, \cdots\}.
	\end{align*}
donde $\alpha_0^{*} = \alpha_0$ y los elementos restantes estén determinados de manera inductiva como
	\begin{align*}
	\alpha_1^{*} \text{ tal que } Z_{\alpha_1^{*}} & \geq \max(Z_{\alpha_0^{*}}, Z_{\alpha_0}) \text{ c.s. } \\
	\alpha_2^{*} \text{ tal que } Z_{\alpha_2^{*}} & \geq \max(Z_{\alpha_1^{*}}, Z_{\alpha_1}) \text{ c.s. } \\
	& \vdots \\
	\alpha_{n+1}^{*} \text{ tal que } Z_{\alpha_{n+1}^{*}} & \geq \max(Z_{\alpha_n^{*}}, Z_{\alpha_n}) \text{ c.s. } \\
	\end{align*}
Como $Z_{\alpha_{n+1}^{*}} \geq Z_{\alpha_{n}}$ casi seguramente y como la sucesión $(Z_{\alpha_n^{*}}, n \geq 0)$ es una sucesión creciente cuyos elementos están acotados casi seguramente por $Z^{*}$ tenemos que
	\begin{align*}
	Z^{*} \geq \lim_n Z_{\alpha_n^{*}} = \sup_{\alpha \in J^{*}} Z_\alpha \geq \sup_{\alpha \in J} Z_\alpha = Z^{*}, \hspace{0.3cm} \mathbb{P}\text{-c.s.}
	\end{align*}
Por lo tanto $Z^{*} = \lim_{n \uparrow \infty} Z_{\alpha_n^{*}}$ casi seguramente.
\end{proof}

\subsection{Solución al problema de paro óptimo}
Con el concepto del supremo esencial podemos mostrar como el proceso
	\begin{align}
	S_n = \esssup_{\tau \in \mathcal{M}_n} \mathbb{E}[G_\tau \mid \mathcal{F}_n] \label{snell}.
	\end{align}
puede generar una estrategia de paro óptima, al ser comparado con el proceso $G_n$. Recordemos que $G_n$ puede ser visto como la ganancia obtenida si se detiene a tiempo $n$; por lo que, $S_n$ puede entenderse como la mayor ganancia esperada si uno no se detiene a tiempo $n$. \\
% (interpretado como la mejor ganancia que uno puede esperar al no detenerse a tiempo $n$) puede ser usado en comparación con $G_n$ (la ganancia obtenida si se detiene a tiempo $n$) para definir una estrategia de paro óptimo. \\

Consideremos el siguiente caso, si $\tau = n$ es un tiempo de paro en $\mathcal{M}_n$ entonces, $S_n \geq G_n$ casi seguramente. La estrategia de paro óptimo consiste en detenerse al primer momento $n$ donde $S_n = G_n$, pues se tendría el caso en que la mayor ganancia esperada después del tiempo $n$ dada la información disponible resulta ser el valor que se obtiene al detenerse a tiempo $n$. 

En otro caso, es decir, cuando se tiene que $S_n > G_n$, se puede interpretar que, la máxima ganancia esperada después del tiempo $n$, dada la información disponible es mayor a la ganancia que obtendríamos en caso de detenernos a tiempo $n$, lo cual, no es óptimo; por lo que no detenerse aparentaría tener un valor ``mayor". \\

El proceso $(S_n : n \geq 1)$ es llamado \emph{Envoltura de Snell} -nombrada así por el matemático Laurie Snell \cite[p.~4]{kyprianou}- pues ``envuelve"  la ganancia obtenida por el proceso $(G_n : n \geq 1)$.

%en otro escenario (es decir, cuando ocurra que $S_n > G_n$) la decisión de no detenerse aparentaría tener un valor ``mayor". 

\begin{theorem}
\label{probmartin}
Sea $n \in \{1, 2, \cdots\}$ fija, supongamos que se cumple
	\begin{align*}
	\mathbb{E} \left[ \sup_n G_n \right] < \infty.
	\end{align*}
Sea $\tau_n = \inf \{k \geq n : S_k = G_k\}$, suponiendo que $\mathbb{P}(\tau_n < \infty) = 1$. Consideremos el problema de paro óptimo definido en (\ref{abb})
	\begin{align*}
	V_n = \sup_{\tau \in \mathcal{M}_n} \mathbb{E}[G_\tau]
	\end{align*}
Entonces
\begin{enumerate}
\item $S_n = \max (G_n, \mathbb{E}[S_{n+1} \mid \mathcal{F}_n])$;
\item El proceso parado $(S_{k \wedge \tau_n} : k \geq n)$ es una martingala;
\item $V_n = \mathbb{E}[S_n]$ y el tiempo de paro $\tau_n$ son óptimos para (\ref{abb});
\item El proceso $(S_k : k \geq n)$ es la mínima supermartingala que domina a $(G_k : k \geq n)$.
\end{enumerate}
\end{theorem}

\begin{proof}
$(1)$ Para probar nuestra igualdad veamos que se cumplen dos condiciones
	\begin{align}
	S_n & \leq \max (G_n, \mathbb{E}[S_{n+1} \mid \mathcal{F}_n]);  \label{corrJC} \\
	S_n & \geq \max (G_n, \mathbb{E}[S_{n+1} \mid \mathcal{F}_n]). \label{corrJC2}
	\end{align}
Para mostrar que la primera desigualdad se cumple, definamos el siguiente concepto
	\begin{align*}
	\bar{\tau} = \max(\tau, n+1), \hspace{0.5cm} \tau \in \mathcal{M}_n.
	\end{align*}
Sin importar que condición llegue a pasar, tenemos que $\bar{\tau} \in \mathcal{M}_{n+1}$. Ahora, consideremos el evento $\{\tau \geq n+1\}$, el cual puede ser visto como $\{\tau > n\}$, que pertenece a $\mathcal{F}_n$, por lo tanto, $\{\tau \geq n+1\} \in \mathcal{F}_n$. Con los resultados anteriores podemos observar que
% Por otra parte, basta con tener la información al momento $n$ para determinar si el evento $\{ \tau \geq n+1\}$ ha ocurrido o no, por lo que $\{ \tau \geq n+1\} \in \mathcal{F}_n$. Entonces
	\begin{align*}
	\mathbb{E}[G_\tau \mid \mathcal{F}_n] & = \mathbb{E}[G_n 1_{\{\tau = n\}} \mid \mathcal{F}_n] + \mathbb{E}[G_{\bar{\tau}} 1_{\{\tau \geq n+1\}} \mid \mathcal{F}_n] \\
	& = G_n 1_{\{\tau = n\}} + 1_{\{\tau \geq n+1\}}  \mathbb{E}[G_{\bar{\tau}} \mid \mathcal{F}_n] \\
	& = G_n 1_{\{\tau = n\}} + 1_{\{\tau \geq n+1\}}  \mathbb{E}[ \mathbb{E} [G_{\bar{\tau}} \mid \mathcal{F}_{n+1}] \mid \mathcal{F}_n].
	\end{align*}
Además, sabemos que $S_{n+1} = \esssup_{\tau \in \mathcal{M}_{n+1}} \mathbb{E}[G_\tau \mid \mathcal{F}_{n+1}] \geq \mathbb{E}[G_{\tau^{*}} \mid \mathcal{F}_{n+1}]$ para toda $\tau^{*} \in \mathcal{M}_{n+1}$, por lo tanto
	\begin{align*}
	\mathbb{E}[G_\tau \mid \mathcal{F}_n] & = G_n 1_{\{\tau = n\}} + 1_{\{\tau \geq n+1\}}  \mathbb{E}[ \mathbb{E} [G_{\bar{\tau}} \mid \mathcal{F}_{n+1}] \mid \mathcal{F}_n] \\
	& \leq G_n 1_{\{\tau = n\}} + 1_{\{\tau \geq n+1\}}  \mathbb{E}[ S_{n+1} \mid \mathcal{F}_n] \\ 
	& \leq \max (G_n, \mathbb{E}[S_{n+1} \mid \mathcal{F}_n]).
	\end{align*}
De modo que,  $\max (G_n, \mathbb{E}[S_{n+1} \mid \mathcal{F}_n])$ es siempre mayor a $\mathbb{E}[G_\tau \mid \mathcal{F}_n]$ para todo $\tau \in \mathcal{M}_n$, por lo que resulta ser una cota superior, entonces, de la propiedad del supremo esencial sabemos que éste siempre es menor a cualquier cota superior, verificando así (\ref{corrJC}). \\

Para corroborar la segunda desigualdad, tenemos que, si $S_n$ es mayor que $\max (G_n, \mathbb{E}[S_{n+1} \mid \mathcal{F}_n])$, entonces tiene que ser mayor a $G_n$ y al mismo tiempo, mayor a $\mathbb{E}[S_{n+1} \mid \mathcal{F}_n]$. Por definición sabemos que
	\begin{align}
	S_n \geq G_n, \hspace{0.5cm} \mathbb{P}\text{-c.s.}  \label{abh}
	\end{align}
Entonces, basta mostrar que 
	\begin{align*}
	S_n \geq \mathbb{E}[S_{n+1} \mid \mathcal{F}_n].
	\end{align*}
Veamos que, $\{ \mathbb{E}[G_\tau \mid \mathcal{F}_{n+1}] : \tau \in \mathcal{M}_{n+1} \}$ cumple con las condiciones del Corolario \ref{coroesssup}. Supongamos que $\alpha, \beta \in \mathcal{M}_{n+1}$, denotemos a $\bar{A}$ como el complemento del conjunto $A$ y definamos
	\begin{align*}
	\gamma = \alpha 1_A + \beta 1_{\bar{A}}, \hspace{0.3cm} \text{ donde } A = \{\mathbb{E}[G_\alpha \mid \mathcal{F}_{n+1}] \geq \mathbb{E}[G_\beta \mid \mathcal{F}_{n+1}]\}.
	\end{align*}
Tenemos que $\gamma$ es un tiempo de paro, pues la suma de tiempos de paro resulta ser un tiempo de paro y más aún $\gamma \in \mathcal{M}_{n+1}$, por lo tanto
	\begin{align*}
	\mathbb{E}[G_\gamma \mid \mathcal{F}_{n+1}] & = \mathbb{E}[G_\alpha 1_A \mid \mathcal{F}_{n+1}] + \mathbb{E}[G_\beta 1_{\bar{A}} \mid \mathcal{F}_{n+1}] \\
	& = 1_A \mathbb{E}[G_\alpha \mid \mathcal{F}_{n+1}] + 1_{\bar{A}} \mathbb{E}[G_\beta \mid \mathcal{F}_{n+1}] \\
	& = \max(\mathbb{E}[G_\alpha \mid \mathcal{F}_{n+1}], \mathbb{E}[G_\beta \mid \mathcal{F}_{n+1}]).
	\end{align*}
Del Corolario \ref{coroesssup} tenemos que existe una sucesión $\{\gamma_k : k \geq 1\} \in \mathcal{M}_{n+1}$ tal que 
	\begin{align*}
	S_{n+1} = \esssup_{\tau \in \mathcal{M}_{n+1}} \mathbb{E}[G_\tau \mid \mathcal{F}_{n+1}] = \lim_{k \rightarrow \infty} \mathbb{E}[G_{\gamma_k} \mid \mathcal{F}_{n+1}],
	\end{align*}
donde $\mathbb{E}[G_{\gamma_k} \mid \mathcal{F}_{n+1}] \leq \mathbb{E}[G_{\gamma_{k+1}} \mid \mathcal{F}_{n+1}]$, con $k \geq 1$, $\mathbb{P}$-c.s., haciendo uso del Teorema de Convergencia Monótona tenemos que
	\begin{align}
	\mathbb{E}[S_{n+1} \mid \mathcal{F}_n] & = \mathbb{E}\left[\lim_{k \rightarrow \infty} \mathbb{E}[G_{\gamma_k} \mid \mathcal{F}_{n+1}] \bigg | \mathcal{F}_n \right]  \nonumber \\
	& = \lim_{k \rightarrow \infty} \mathbb{E}[\mathbb{E}[G_{\gamma_k} \mid \mathcal{F}_{n+1}] \mid \mathcal{F}_n] \nonumber \\
	& = \lim_{k \rightarrow \infty} \mathbb{E}[G_{\gamma_k} \mid \mathcal{F}_n] \nonumber \\
	& \leq S_n ( = \esssup_{\tau \in \mathcal{M}_n} \mathbb{E}[G_\tau \mid \mathcal{F}_n]). \label{abi}
	\end{align}
% El proceso $(\mathbb{E}[G_\tau \mid \mathcal{F}_{n+1}] : \tau \in \mathcal{M}_{n+1})$ cumple con las condiciones del Corolario \ref{coroesssup}. 

% Para esto, tenemos que si $(\mathbb{E}[G_\tau \mid \mathcal{F}_{n+1}] : \tau \in \mathcal{M}_{n+1})$ cumple  las condiciones del Corolario \ref{coroesssup} 
De (\ref{abh}) y (\ref{abi}) sabemos que
\begin{align}
S_n & \geq \max (G_n, \mathbb{E}[S_{n+1} \mid \mathcal{F}_n]).
\end{align}

Por último, de las dos desigualdades (\ref{corrJC}) y (\ref{corrJC2}) tenemos que $S_n = \max \{G_n, \mathbb{E}[S_{n+1} \mid \mathcal{F}_n]\}$. \\

\noindent $(2)$ De la Propiedad $(1)$ tenemos que para toda $n$
	\begin{align*}
	S_n \geq \mathbb{E}[S_{n+1} \mid \mathcal{F}_n].
	\end{align*}
Por lo tanto, $S_n$ es una supermartingala. Veamos que si $n \leq k < \tau_n$, con $\tau_n$ como el primer momento en el que $S_\ell = G_\ell$ para algún tiempo $\ell \geq n$, tenemos de la propiedad $(1)$ que $S_k = \max ( G_k, \mathbb{E} [S_{k+1} \mid \mathcal{F}_k] )$, pero como $k < \tau_n$ entonces
	\begin{align}
	S_k = \mathbb{E}[S_{k+1} \mid \mathcal{F}_k], \hspace{0.3cm} \text{ c.s. } \label{abj}
	\end{align}
De la misma definición de $\tau_n$ sabemos que 
	\begin{align}
	S_{\tau_n} = G_{\tau_n}. \label{abk}
	\end{align}
De (\ref{abj}) y (\ref{abk}) tenemos que para toda $k \geq n$
	\begin{align*}
	\mathbb{E}[S_{(k+1) \wedge \tau_n} \mid \mathcal{F}_k] & = \mathbb{E}[S_{(k+1) \wedge \tau_n} 1_{\{\tau_n \leq k\}} \mid \mathcal{F}_k] + \mathbb{E}[S_{(k+1) \wedge \tau_n} 1_{\{k+1 \leq \tau_n\}} \mid \mathcal{F}_k] \\
	& = \mathbb{E}[S_{k \wedge \tau_n} 1_{\{\tau_n \leq k\}} \mid \mathcal{F}_k] + \mathbb{E}[S_{k+1} 1_{\{k+1 \leq \tau_n\}} \mid \mathcal{F}_k] \\
	& = S_{k \wedge \tau_n} 1_{\{\tau_n \leq k\}} + 1_{\{k+1 \leq \tau_n\}} \mathbb{E}[S_{k+1} \mid \mathcal{F}_k] \\
	& = S_{k \wedge \tau_n} 1_{\{\tau_n \leq k\}} + 1_{\{k+1 \leq \tau_n\}} S_k \\
	& = S_{k \wedge \tau_n}.
	\end{align*}
Por lo tanto, el proceso parado $(S_{k \wedge \tau_n} : k \geq n)$ es una martingala. \\

\noindent $(3)$ Para mostrar que $\mathbb{E}[S_n] = V_n$ observemos que se cumple al mismo tiempo las siguientes desigualdades
\begin{align*}
	\mathbb{E}[S_n] & \leq V_n, \\
    \mathbb{E}[S_n] & \geq V_n.
\end{align*}

Por la Propiedad $(2)$ tenemos, en particular para $N > n$, que
	\begin{align*}
	S_n = S_{n \wedge \tau_n} & = \mathbb{E}[S_{N \wedge \tau_n} \mid \mathcal{F}_n] \\
	& = \mathbb{E}[S_{N \wedge \tau_n} 1_{\{\tau_n < N\}} + S_{N \wedge \tau_n} 1_{\{\tau_n \geq N\}} \mid \mathcal{F}_n] \\
	& = \mathbb{E}[S_{\tau_n} 1_{\{\tau_n < N\}} + S_N 1_{\{\tau_n \geq N\}} \mid \mathcal{F}_n].
	\end{align*}
De (\ref{abk}) tenemos 
	\begin{align}
	S_n & = \mathbb{E}[S_{\tau_n} 1_{\{\tau_n < N\}} + S_N 1_{\{\tau_n \geq N\}} \mid \mathcal{F}_n] \nonumber \\
	& = \mathbb{E}[G_{\tau_n} 1_{\{\tau_n < N\}} + S_N 1_{\{\tau_n \geq N\}} \mid \mathcal{F}_n] \nonumber \\ 
	& = \mathbb{E}[G_{\tau_n} 1_{\{\tau_n < N\}} \mid \mathcal{F}_n] + \mathbb{E}[S_N 1_{\{\tau_n \geq N\}} \mid \mathcal{F}_n]. \label{abl}
	\end{align}
Para $n < N$ vemos que
	\begin{align*}
	\mathbb{E}[G_n \mid \mathcal{F}_N] \leq \mathbb{E} \left[ \sup_n G_n \bigg | \mathcal{F}_N \right]
	\end{align*}
Por lo que, de la definición del supremo esencial, sabemos que éste es la mínima cota superior, por lo que
	\begin{align*}
	S_N = \esssup_{\tau \in \mathcal{M}_N} \mathbb{E}[G_\tau \mid \mathcal{F}_N] \leq \mathbb{E}\left[\sup_n G_n \bigg | \mathcal{F}_N \right], \hspace{0.3cm} \text{ para } n < N.
	\end{align*}
Y por lo tanto usando el Teorema de Convergencia Monótona, además del supuesto de que $\mathbb{E}[\sup_n G_n] < \infty$ implica que $\mathbb{E}[\sup_n G_n \mid \mathcal{F}_n] < \infty$ c.s., 
	\begin{align}
	\lim_{N \uparrow \infty} \mathbb{E}[S_N 1_{\{\tau_n \geq N\}} \mid \mathcal{F}_n] & \leq \lim_{N \uparrow \infty} \mathbb{E} \left[ \mathbb{E} \left[ \sup_{n < N} G_n  \mid \mathcal{F}_N \right] 1_{\{\tau_n \geq N\}} \bigg | \mathcal{F}_n \right] \nonumber \\
	& \leq \lim_{N \uparrow \infty} \mathbb{E}\left[\sup_{n < N} G_n 1_{\{\tau_n \geq N\}} \bigg | \mathcal{F}_n\right] \\
	& = \mathbb{E} \left[ \lim_{N \uparrow \infty} \sup_{n < N} G_n 1_{\{\tau_n \geq N\}} \bigg | \mathcal{F}_n \right] = 0. \label{abm}
	\end{align}
La última igualdad se da por el supuesto de que $\mathbb{P}(\tau_n < \infty) = 1$, es decir, sabemos que $\tau_n = \inf \{k \geq n : S_k = G_k \}$ es finito casi seguramente, por lo que
	\begin{align*}
	1_{ \{ \tau_n \geq N \} } \rightarrow 0, \hspace{0.3cm} N \rightarrow \infty.
	\end{align*}
De (\ref{abl}) y (\ref{abm}) concluimos que si $N \rightarrow \infty$ entonces
	\begin{align*}
	S_n = \mathbb{E}[G_{\tau_n} \mid \mathcal{F}_n].
	\end{align*}
Por último, recordemos que $S_{\tau_n} = G_{\tau_n}$, por lo que
	\begin{align}
	\mathbb{E}[S_n] & = \mathbb{E}[S_{\tau_n}] = \mathbb{E}[G_{\tau_n}] \leq \sup_{\tau \in \mathcal{M}_n} \mathbb{E}[G_{\tau}] = V_n. \label{abn}
	\end{align}

Por otra parte, como $(S_n, n \geq 1)$ es una supermartingala y por el Teorema \ref{opcional} tenemos que
	\begin{align*}
	S_n \geq \mathbb{E}[S_\tau \mid \mathcal{F}_n], \hspace{0.3cm} \text{ para todo } \tau \in \mathcal{M}_n.
	\end{align*}
Luego, para todo $\tau \in \mathcal{M}_n$
	\begin{align*}
	\mathbb{E}[S_n] \geq \mathbb{E}[S_\tau],
	\end{align*}
En particular, 
	\begin{align*}
	\mathbb{E}[S_n] \geq \sup_{\tau \in \mathcal{M}_n} \mathbb{E}[S_\tau ].
	\end{align*}
% Sabemos además que $S_n \geq G_n$ para todo $n$ entonces
%	\begin{align*}
%	\mathbb{E}[S_n] \geq \mathbb{E}[S_\tau] \geq \mathbb{E}[G_\tau].
%	\end{align*}
De la definición de $S$ sabemos que $S_n \geq G_n$ $\mathbb{P}$-c.s., entonces
	\begin{align}
	\mathbb{E}[S_n] \geq \sup_{\tau \in \mathcal{M}_n}\mathbb{E}[S_\tau] \geq \sup_{\tau \in \mathcal{M}_n}\mathbb{E}[G_\tau] = V_n. \label{abo}
	\end{align}

De (\ref{abn}) y (\ref{abo}) tenemos que $\mathbb{E}[S_n] = V_n$ entonces
\begin{align*}
\mathbb{E}[S_n] = \mathbb{E}[G_{\tau_n}] = \sup_{\tau \in \mathcal{M}_n}\mathbb{E}[G_\tau] = V_n.
\end{align*}
Es decir, $\mathbb{E}[S_n]$ y el tiempo de paro $\tau_n$ son óptimos para el problema de paro óptimo definido en (\ref{abb}). \\

\noindent $(4)$ En la primera parte del Teorema, probamos que $(S_n, n \geq 1)$ es una supermartingala. Además, comprobamos que $S_n \geq G_n$ $\mathbb{P}$-c.s., para toda $n$, es decir, $(S_n, n \geq 1)$ domina al proceso $(G_n, n \geq 1)$. Para probar que $S$ es la mínima supermartingala que domina a $G$ supongamos que $(U_n, n \geq 1)$ es también una supermartingala que domina a $G$. De la demostración en $(3)$ y al $U$ dominar a $G$ tenemos
	\begin{align*}
	S_k = \mathbb{E}[G_{\tau_k} \mid \mathcal{F}_k] \leq \mathbb{E}[U_{\tau_k} \mid \mathcal{F}_k].
	\end{align*}
Del Teorema de Paro Opcional de Doob y el Lema de Fatou aplicado al proceso de paro $U_{m \wedge \tau_n}$ obtenemos
	\begin{align*}
	S_k & = \mathbb{E}[U_{\tau_k} \mid \mathcal{F}_k] \\
    & = \mathbb{E} \left[ \liminf_{m \rightarrow \infty} U_{m \wedge \tau_k } \bigg| \mathcal{F}_k \right] \\
    & \leq \liminf_{m \rightarrow \infty} \mathbb{E}[U_{m \wedge \tau_k } \mid \mathcal{F}_k] \\
    & \leq \mathbb{E}[U_{m \wedge \tau_k } \mid \mathcal{F}_k] \leq U_k.
	\end{align*}
La segunda igualdad, está justificada por el hecho de que $U_{m \wedge k}$ está acotado inferiormente por $- \sup_{n \leq \gamma \leq \tau_k} |G_{\gamma} |$. Por lo tanto, $(S_k : k \geq n)$ es la mínima supermartingala que domina a $(G_k : k \geq n)$.

\end{proof}

\section{Enfoque con Cadenas de Markov}
En esta sección presentaremos los resultados básicos de la teoría de paro óptimo, considerando tiempos discreto y procesos de Markov. Sea $X = (X_n, n \geq 1)$ una cadena de Markov definida en un espacio de probabilidad $(\Omega, \mathcal{F}, (\mathcal{F}_n)_{n \geq 1}, \mathbb{P}_x)$ y tomando valores en un espacio medible $(E, \mathcal{E})$. Asumiremos además que para todo $x \in E$, la cadena $X$ empieza en el estado $x$ bajo la medida $\mathbb{P}_x$. Por cuestiones de simplicidad  definiremos al espacio de probabilidad como $(\Omega, \mathcal{F}) = (E^{\mathbb{Z}_{+}}, \mathcal{E}^{\mathbb{Z}_{+}})$, así los operadores de traslación $\theta_n : \Omega \rightarrow \Omega$ pueden definirse como $f \circ \theta_n (\omega_{m}) = f(\omega_{m+n})$ para funcionales $f$ del proceso canónico $\omega_m$. \\

Recordemos que la propiedad de Markov nos dice que para toda función positiva, continua y uniformemente acotada $f$, tenemos que para toda $n \in \mathbb{N}$ y $x \in E$, condicionando a $X_n = x$, la distribución de $X_{n+1}$ es
	\begin{align*}
	\mathbb{P}_{xy} = \mathbb{P}(X_{n+1} = y \mid X_n = x), \hspace{0.3cm} x \in E
	\end{align*}
Además es independiente de $\mathcal{F}_1, \mathcal{F}_1, \ldots, \mathcal{F}_{n-1}$, entonces 
	\begin{align*}
	\mathbb{E}[f(X_{n+k}) \mid \mathcal{F}_n] = \mathbb{E}[f(X_{n+k}) \mid X_n] = \mathbb{E}_x [f(\tilde{X}_k)] \bigg|_{x = X_n}.
	\end{align*}
donde $\tilde{X}$ es una copia independiente de $X$. Consideremos una función medible $G : E \rightarrow \mathbb{R}$ que satisface la siguiente condición (con $G(X_N) = 0$ si $N = \infty$):
	\begin{align*}
	\mathbb{E}_x \left[ \sup_n |G(X_n)| \right] < \infty,
	\end{align*}
Para todo $x \in E$, consideramos el siguiente problema de paro óptimo
	\begin{align}
	V_n (x) = \sup_{\tau \in \mathcal{M}_n} \mathbb{E}_x [G(X_\tau)]. \label{abu}
	\end{align}
Se ha reemplazado el proceso general $G_n$  por un proceso dependiente solo del estado actual de la cadena de Markov, $G(X_n)$, usualmente $G$ es una función continua. El siguiente resultado muestra la relación entre nuestro problema (\ref{abu}) y la \emph{envoltura de Snell}.

\begin{lemma}
Supongamos que $\mathbb{P}_x(\tau_0 < \infty) = 1$ para toda $x \in E$, donde $\tau_0 = \inf \{k \geq 0 : S_k = G_k\}$. Sea $S_n$ la envoltura de Snell dada en (\ref{snell}), entonces tenemos
	\begin{align}
	S_n = V(X_n), \hspace{0.3cm} \mathbb{P}_x\text{-c.s.} \label{abt}
	\end{align}
Para toda $x \in E$, con $n \geq 1$.
\end{lemma}
\begin{proof}
Veamos en primer lugar que si $\tau \in \mathcal{M}_0$ entonces $\tau \circ \theta_n \in \mathcal{M}_n$. En particular, si $\tau_k^B := \inf \{n \geq k : X_n \in B\}$ para un $B \in \mathcal{E}$, tenemos que
	\begin{align*}
	\tau_0^B \circ \theta_k & = \inf \{m \geq 0 : X_{m+k} \in B\} \\
	& = \inf \{n \geq k : X_n \in B\} - k \\
	& = \tau_k^B - k.
	\end{align*}
Entonces
	\begin{align}
	S_n & =  \esssup_{\tau \in \mathcal{M}_n} E[G(X_\tau) \mid \mathcal{F}_n] \nonumber \\
	& \geq \esssup_{\tau \in \mathcal{M}_0} E[G(X_{n+ (\tau \circ \theta_n)}) \mid \mathcal{F}_n] \nonumber \\
	& =  \esssup_{\tau \in \mathcal{M}_0} \mathbb{E}_x [G(\tilde{X}_\tau)] \bigg|_{x = X_n} \nonumber \\
	& = V(X_n). \label{abp}
	\end{align}
Donde $\tilde{X}$ es una copia independiente de $X$. La desigualdad ocurre pues no todos los tiempos de paro en $\mathcal{M}_n$ pueden ser escritos de la forma $n+ (\tau \circ \theta_n)$ con $\tau \in \mathcal{M}_0$. \\

Por otro lado, como $\mathcal{M}_1 \subseteq \mathcal{M}_0$ y del hecho anterior
	\begin{align*}
	V(x) & \geq \sup_{\tau \in \mathcal{M}_1} \mathbb{E}_x [G(X_\tau)] \\
	& \geq \sup_{\tau \in \mathcal{M}_0} \mathbb{E}_x [G(X_{1+ (\tau \circ \theta_1)})].
	\end{align*}
Como $1+ (\tau \circ \theta_1) \in \mathcal{M}_1$ entonces podemos hacer uso de la esperanza condicional y aplicar la propiedad de Markov
	\begin{align*}
	\sup_{\tau \in \mathcal{M}_0} \mathbb{E}_x [G(X_{1+ (\tau \circ \theta_1)})] & = \sup_{\tau \in \mathcal{M}_0} \mathbb{E}_x[\mathbb{E}[G(X_{1+ (\tau \circ \theta_1)}) \mid \mathcal{F}_1]] \\
	& = \sup_{\tau \in \mathcal{M}_0} \mathbb{E}_x \left[ \mathbb{E}_x [G(\tilde{X}_\tau)] \bigg|_{x= X_1} \right].
	\end{align*}
Por último, de la definición de tiempo de paro
	\begin{align*}
	V (x) & \geq \sup_{\tau \in \mathcal{M}_0} \mathbb{E}_x \left[ \mathbb{E}_x [G(\tilde{X}_\tau)] \bigg|_{x= X_1} \right] \\
	& \geq \mathbb{E}_x \left[ \mathbb{E}_x [G(\tilde{X}_{\tau_0})]\bigg|_{x= X_1} \right] \\
	& = \mathbb{E}_x \left[ \sup_{\tau \in \mathcal{M}_0} \mathbb{E}_x[ G(\tilde{X}_\tau)]  \bigg|_{x= X_1} \right] \\
	& \geq \mathbb{E}_x [V(X_1)].
	\end{align*}
Donde la igualdad es correcta al ser $\tau_0$ óptimo en $V$, pues se tiene que, $\mathbb{E}[G_{\tau_0}] = \sup_{\tau \in \mathcal{M}_0} \mathbb{E}[G_{\tau}]$. Utilizando la propiedad de Markov en la desigualdad anterior, obtenemos 
	\begin{align*}
	V (x) \geq \mathbb{E}_x [V(X_1)] = \mathbb{E}_x [V(X_1) \mid \mathcal{F}_0].
	\end{align*}
De forma similar, tenemos que para $m \leq n$
	\begin{align*}
	V(X_m) \geq \mathbb{E}_x [V(X_n)] = \mathbb{E}_x [V(X_n) \mid \mathcal{F}_m].
	\end{align*}
Por lo tanto, $(V(X_n) : n \geq 1)$ es una supermartingala y entonces tenemos que
	\begin{align}
	E[V(X_{n+1}) \mid \mathcal{F}_n] = \mathbb{E}_x [V(\tilde{X}_1)] \bigg|_{x = X_n} \leq V(X_n). \label{abq}
	\end{align}
De la definición de $V(x)$ tenemos que para cualquier tiempo de paro en $\mathcal{M}_1$
	\begin{align*}
	V(x) \geq \mathbb{E}_x [G(X_\tau)].
	\end{align*}
En particular si definimos al tiempo de paro $\tau$ como el valor inicial del proceso podemos observar que para toda $x \in E$
	\begin{align}
	V(x) \geq G(x) \label{abr}
	\end{align}
De (\ref{abq}) y (\ref{abr}) podemos concluir que $V_n$ es una supermartingala que domina a $G_n$, entonces, utilizando (4) del Teorema \ref{probmartin} tenemos que
	\begin{align}
	V(X_n) \geq S_n \label{abs}
	\end{align}
Finalmente, al tener (\ref{abp}) y (\ref{abs}), podemos concluir la prueba.
\end{proof}

A continuación se expresa un resultado análogo al Teorema \ref{probmartin}. Dada la función $V$, definimos los siguientes conceptos. La región de continuación está dada por
	\begin{align*}
	C = \{x \in E : V(x) > G(x)\}
	\end{align*}
Y la región de paro
	\begin{align*}
	P = \{x \in E : V(x) = G(x)\}
	\end{align*}
El óptimo tiempo de paro entonces debería ser
	\begin{align*}
	\tau_P = \inf \{n \geq 1 : X_n \in P\}
	\end{align*}

\begin{theorem}
\label{paro_markov}
Supongamos que $\mathbb{P} (\tau_P < \infty) = 1$ para toda $x \in E$. Entonces
	\begin{enumerate}
	\item El proceso de paro $(V(X_{n \wedge \tau_P}) : n \geq 1)$ es una $\mathbb{P}_x$-martingala para cada $x \in E$.
	\item El tiempo de paro $\tau_P$ es óptimo para (\ref{abo}).
	\item El proceso $(V(X_n) : n \geq 1)$ es la mínima supermartingala que domina a $(G(X_n) : n \geq 1)$
	\end{enumerate}
\end{theorem}
\begin{proof}
El resultado se sigue directamente del Teorema \ref{probmartin} haciendo uso de (\ref{abt}).
\end{proof}

\chapter{Movimiento Browniano}
En el presente capítulo, enfocaremos nuestra atención al estudio del movimiento Browniano. Desde su construcción, por el método de Lévy; hasta sus propiedades trayectoriales y más importante aún, verificando que este proceso cumple con la Propiedad de Markov. \\

\section{Variables Aleatorias Gaussianas}
Conviene recordar algunas propiedades importantes de las variables aleatorias Gaussianas, las cuales estarán presentes en la construcción del movimiento Browniano. \\

Decimos que $X$ se distribuye como una variable aleatoria Gaussiana con media cero y varianza unitaria si su transformada de Fourier satisface la siguiente igualdad
	\begin{align*}
	\varphi_X (\lambda) := \mathbb{E}\left[ \exp \{ i \lambda X \} \right] = \exp \{ \lambda^2/2 \}, \hspace{0.5cm} \forall \lambda \in \mathbb{R}.
	\end{align*}
	
De igual manera, las siguientes desigualdades serán una valiosa herramienta en la construcción del movimiento Browniano.

\begin{proposition}
Sea $X$ una variable aleatoria Gaussiana centrada y con varianza unitaria, entonces para todo $x > 0$ se tienen las siguientes desigualdades.
	\begin{align*}
	\frac{1}{\sqrt{2 \pi}} \left( \frac{1}{x} - \frac{1}{x^3} \right) e^{-x^2 / 2} \leq \mathbb{P} (X & > x) \leq \frac{1}{\sqrt{2 \pi}} \frac{1}{x} e^{-x^2 / 2}, \hspace{0.3 cm}  x > 0, 	
	\end{align*}
\end{proposition}
\begin{proof}
Para mostrar el lado derecho de la desigualdad, recordemos que, si $X$ es una variable aleatoria Gaussiana centrada y de varianza unitaria entonces, tenemos que para toda $x > 0$
	\begin{align}
	\mathbb{P} (X > x) = \frac{1}{\sqrt{2 \pi}} \int_x^{\infty} e^{-t^2 / 2} dt. \label{abw}
	\end{align}
Consideremos, los valores $t$ tales que $0 < x \leq t$, entonces $1 \leq t/x$. De lo anterior y haciendo uso de \ref{abw} obtenemos
	\begin{align*}
	\mathbb{P} (X > x) & = \frac{1}{\sqrt{2 \pi}} \int_x^{\infty} e^{-t^2 / 2} dt \\
	& \leq \frac{1}{\sqrt{2 \pi}} \int_x^{\infty} \frac{t}{x} e^{-t^2 / 2} dt \\
	& = \frac{1}{\sqrt{2 \pi}} \frac{1}{x} e^{-x^2 / 2}.
	\end{align*}
Para probar el lado izquierdo, veamos que integrando por partes obtenemos que
\begin{align*}
	\mathbb{P} (X > x) = \int_{x}^{\infty} \frac{1}{\sqrt{2 \pi}} e^{- t^2 / 2} dt & = \int_{x}^{\infty} \frac{1}{t} \frac{1}{\sqrt{2 \pi}} t e^{- t^2 / 2} dt\\
    & = - \frac{1}{t} \frac{1}{\sqrt{2 \pi}} \exp\{- t^2 / 2 \} \bigg|_{x}^{\infty} \\
    & - \int_{x}^{\infty} \left( - \frac{1}{t^2} \right) \left( - \frac{1}{\sqrt{2 \pi}} \exp \{ - t^2 / 2 \} \right) dt \\
    & = \frac{1}{x} \frac{e^{- x^2 / 2}}{\sqrt{2 \pi}} - \int_{x}^{\infty} \frac{1}{t^2} \frac{e^{- t^2 / 2}}{\sqrt{2 \pi}} dt \\
    & > \frac{e^{-x^2 / 2}}{\sqrt{2 \pi}} \left( \frac{1}{x} - \frac{1}{x^3} \right) \hspace{0.5cm} \text{para } x > 0
\end{align*}
La desigualdad se obtiene al tener un valor positivo de la integral pues su integrando es siempre positivo.
\end{proof}

Tengamos presente el siguiente resultado que muestra la relación de convergencia en distribución y en probabilidad para variables aleatorias Gaussianas.

\begin{theorem}
Sea $(X_n, n \geq 1)$ una sucesión de variables aleatorias Gaussianas tales que para toda $n$, $X_n \sim N(m_n, \sigma_n^2)$, con $m_n, \sigma_n^2 \in \mathbb{R}$ y $\sigma_n^2 \geq 0$.
	\begin{enumerate}
	\item Si $X_n \rightarrow X$ en distribución, entonces $X \sim N(m, n)$ donde $m = \lim_{n \rightarrow \infty} m_n$ y $\sigma^2 = \lim_{n \rightarrow \infty} \sigma_n^2$.
	\item Si $X_n \rightarrow X$ en probabilidad entonces para toda $p \geq 1$, $X_n \rightarrow X$ en $\mathcal{L}^p$.
	\end{enumerate}
\end{theorem}

\begin{proof}
%  para cualquier función continua y acotada, se cumple lo siguiente
% 	\begin{align*}
% 	\mathbb{E}[f(X_n)] \rightarrow \mathbb{E}[f(X)], \hspace{0.5cm} n \rightarrow \infty.
% 	\end{align*}
1. Puesto que cada $X_n$ es una variable aleatoria Gaussiana con media $m_n$ y varianza $\sigma_n^2$, la función característica de éstas se define como
	\begin{align*}
	\varphi_{X_n} (\lambda) = \exp \{i m_n \lambda - \frac{\sigma_n^2}{2} \lambda^2\} = \mathbb{E}\left[ e^{i \lambda X_n} \right].
	\end{align*}
Como $X_n$ converge en distribución a $X$, tenemos del Teorema de Continuidad \cite[p.~320]{shiryaev} que $\varphi_{X_n} (t)$ converge a una función $\varphi_X (\lambda)$, con $\lambda \in \mathbb{R}$. Veamos ahora que, $\varphi_X(\lambda)$ es la función característica de una variable aleatoria Gaussiana con media $m$ y varianza $\sigma^2$, lo que implicaría que $X \sim N(m, \sigma^2)$. \\

Para corroborar lo anterior veamos que $m_n$ y $\sigma_n^2$ convergen, y entonces con $m := \lim_n m_n$ y $\sigma^2 := \lim_n \sigma_n^2$ se tendría que

\begin{align*}
	\varphi_X(\lambda) = \exp \{i m \lambda - \frac{\sigma^2}{2} \lambda^2\},
\end{align*}
en otras palabras, que $X$ es una variable aleatoria Gaussiana. \\

Para ver que $(m_n)$ y $(\sigma_n^2)$ convergen, primero debemos verificar que las sucesiones son acotadas. Al ser $(X_n, n \geq 1)$ una sucesión de variables Gaussianas que convergen en distribución a $X$ podemos tomar a $F_n$, $F$ como las funciones de distribución acumulada de $X_n$ y $X$ respectivamente. Tomemos $t$ tal que $F(t) > 1 - \epsilon$, por lo que $t$ es un punto de continuidad de $F$. Luego, de la convergencia en distribución tenemos que $F_n(t) > 1 - \epsilon$ para toda $n \geq N$. Entonces la sucesión es tensa en el sentido de que para todo $\epsilon > 0$ existe una $r > 0$ tal que
\begin{align}
	\sup_{n \geq 1} \mathbb{P} (|X_n| > r) < \epsilon. \label{tightness}
\end{align}

Supongamos que $\sup_n (|m_n| + \sigma_n^2) = \infty$ y veamos que llegamos a una contradicción. En el primer caso, supongamos que $\sup_n |m_n| = \infty$. Al ser $X_n$ una variable Gaussiana, luego, $m_n$ además de ser la media de la distribución, también representa la mediana de $X_n$, por lo tanto 
\begin{align*}
	\mathbb{P}(|X_n| > |m_n|) \geq 1/2.
\end{align*}

Se sigue entonces que, para todo $M > 0$, existe una $n$ tal que $|m_n| > M$ y para este valor de $n$ se tiene que $\mathbb{P}(|X_n| > M) \geq 1/2$. Por lo tanto, para toda $M$ tenemos
\begin{align*}
	\sup_n \mathbb{P}(|X_n| > M) \geq 1/2,
\end{align*}
lo cual contradice la propiedad de \ref{tightness}. \\

Por otra parte, si suponemos que $\sup_n \sigma_n^2 = \infty$ podemos notar que al ser $X_n$ una variable Gaussiana se tiene que la $\mathbb{P}(|x_n| \leq M$ está acotada por la constante $(\sqrt{2\sigma^2 \pi})^{-1}$. Por lo tanto, para toda $M > 0$ y para toda $n$
\begin{align*}
	\mathbb{P}(|X_n| \leq M) & \leq \int_{|M|} \frac{1}{\sigma_n \sqrt{ 2 \pi}} dx \\
    & = \frac{M}{\sigma_n} \sqrt{\frac{2}{\pi}}.
\end{align*}
Utilizando la desigualdad anterior y la suposición sobre $\sigma_n^2$ podemos ver que
\begin{align*}
	\sup_n \mathbb{P}(|X_n| > M) \geq \sup_n \left( 1 - \frac{M}{\sigma_n} \sqrt{\frac{2}{\pi}} \right) = 1,
\end{align*}
nuevamente, llegamos a una contradicción. Por lo tanto, tenemos que $(m_n)$ y $(\sigma_n^2)$ son sucesiones acotadas. \\

Por último, veamos que $(m_n)$ y $(\sigma_n^2)$ convergen. Consideremos el módulo de $\varphi_{X_n} (\lambda)$, recordando de la identidad de Euler que $e^{ix} = \cos(x) + i \sen(x)$ y observemos que
	\begin{align*}
	|\varphi_{X_n} (\lambda)| &=  \left \lvert \exp \left\{i m_n \lambda - \frac{\sigma_n^2}{2} \lambda^2 \right\} \right \rvert \\
	& = \bigg \lvert \exp \left\{i m_n \lambda \right\} \bigg \rvert  \left\lvert \exp \left\{ - \frac{\sigma_n^2}{2} \lambda^2 \right\} \right\rvert\\
	& = \exp \left\{- \frac{\sigma_n^2}{2} \lambda^2 \right\}.
	\end{align*}
Para toda $\lambda \in \mathbb{R}$. Del Teorema de Mapeo Continuo \cite[p.~21]{billingsley} tenemos que
	\begin{align*}
	|\varphi_{X_n} (\lambda)| = \exp \left\{- \frac{\sigma_n^2}{2} \lambda^2 \right\} \rightarrow |\varphi_X (\lambda)|, \hspace{0.3cm} \text{ para toda } \lambda \in \mathbb{R}.
	\end{align*}
Lo anterior, implica que  $\sigma_n^2 \rightarrow \sigma^2 \in [0, \infty)$ pues la convergencia no depende de $m_n$. \\

Consideremos dos valores, $m$ y $m'$, de adherencia en la sucesión mencionada. Entonces, para toda $\lambda \in \mathbb{R}$, 
	\begin{align*}
	e^{i m \lambda} = e^{i m' \lambda},
	\end{align*}
se tiene que $m$ y $m'$ son iguales. Si tomamos a $m = \lim_n m_n$ y a $\sigma^2 = \lim_n \sigma_n^2$ tenemos que la transformada de Fourier de $X$ es la función característica de una variable aleatoria Gaussiana, es decir, $X \sim N(m, \sigma^2)$. \\

2. La transformada de Laplace (también conocida como Función Generadora de Momentos) para una variable aleatoria $X_n$ Gaussiana satisface para toda $\theta \in \mathbb{R}$,
	\begin{align*}
	\mathbb{E} \left[ \exp \{\theta X_n\} \right] = \exp \left\{ m_n \theta +  \frac{\sigma_n^2}{2} \theta^2 \right\}. 
	\end{align*}
Además del hecho de que $e^{|x|} \leq e^x + e^{-x}$, se tiene 
	\begin{align*}
	\mathbb{E} \left[ \exp \{ \theta |X_n|\} \right] \leq \mathbb{E}[\exp \{ \theta X_n\}] + \mathbb{E}[\exp \{ - \theta X_n\}].
	\end{align*}
Por lo que, 
	\begin{align*}
	\sup_{n \geq 1} \mathbb{E} \left[ \exp \{ \theta |X_n|\} \right] < \infty,
	\end{align*}
y esto para toda $\theta \in \mathbb{R}$. Luego, para $q \geq 0$, tenemos que para una $|x|$ suficientemente grande se cumple que $|x^q| \leq e^{\theta |x|}$, de esta desigualdad deducimos que
	\begin{align*}
	\sup_{n \geq 1} \mathbb{E} \left[ |X_n|^q \right] < \infty,
	\end{align*}
y entonces
	\begin{align}
	\sup_{n \geq 1} \mathbb{E} \left[ |X_n - X|^q \right] < \infty. \label{aby}
	\end{align}

Consideremos entonces, $p \geq 1$. Por hipótesis, la sucesión $(|X_n - X|^p, n \geq 1)$ converge a 0 en probabilidad, y además si tomamos $q = 2p$ en (\ref{aby}) tenemos que la sucesión esta acotada en $\mathcal{L}^2$ y por lo tanto es uniformemente integrable. Ambas propiedades de la sucesión implican que existe una convergencia en $\mathcal{L}^{p}$ (véase \cite[p.~221]{gut}). Por lo tanto, 
	\begin{align*}
	\mathbb{E} \left[ |X_n - X|^p \right] \rightarrow 0, \hspace{0.5cm} n \rightarrow \infty.
	\end{align*}
\end{proof}

Recordemos un último resultado antes de presentar la construcción del movimiento Browniano.

\begin{definition}
Una familia $(X_t, t \geq 0)$ de variables aleatorias es un proceso Gaussiano si para toda $n$ y para toda $(t_1, t_2, \cdots, t_n) \in \mathbb{R}^n$, $(X_{t_1}, X_{t_2}, \cdots, X_{t_n})$ es un vector Gaussiano con valores en $\mathbb{R}^n$, es decir, cualquier combinación lineal de los elementos del vector es una variable aleatoria Gaussiana.
\end{definition}

\begin{proposition} \label{gaussiano1}
Sea $\bar{X} = (X_1, X_2, \ldots, X_n)$ un vector Gaussiano. Entonces $X_1, X_2, \ldots, X_n$ son independientes si y solamente si $COV(X_i, X_j) = 0$ para toda $i \neq j$.
\end{proposition}
\begin{proof}
Veamos en primer lugar el recíproco de la proposición. Sin perdida de generalidad supongamos que el vector $\bar{X}$ es centrado, es decir, para toda $\mathbb{E}[X_i] = 0$ con $1 \leq i \leq n$. \\

Como los componentes del vector no están correlacionados tenemos que la transformada de Fourier del vector $\bar{X}$ resulta un producto de transformada de Fourier, es decir, para toda $\lambda \in \mathbb{R}^n$ se tiene que
	\begin{align*}
	\mathbb{E} \left[ \exp \left\{ i \langle \bar{X}, \lambda \rangle \right\} \right] = \exp \left\{ - \frac{1}{2} \sum_{j = 1}^{n} \sigma_j^2 \lambda_j^2 \right\} = \prod_{j=1}^{n} \mathbb{E} \left[ \exp \left\{ i \lambda_j X_j \right\} \right],
	\end{align*}
por lo tanto, $X_1, X_2, \ldots, X_n$ son independientes. La condición necesaria de la proposición es clara basándose en el argumento anterior.
\end{proof}

\section{Construcción}
A continuación definiremos algunos conceptos referentes al movimiento Browniano.

\begin{definition}
La trayectoria de $(X_t, t \geq 0)$ se define como la función $t \rightarrow X_t(\omega)$ para cada $\omega \in \Omega$.
\end{definition}

De la última definición podemos decir que el resultado de un experimento aleatorio puede observarse de manera continua en el tiempo a través de un modelo matemático. Con los siguientes conceptos podemos caracterizar a los procesos por sus trayectorias.

\begin{definition}
Se dice que un proceso estocástico es continuo por la derecha (por la izquierda) si las trayectorias del proceso son continuas por la derecha (por la izquierda) casi seguramente.
\end{definition}

De la misma manera, se dice que un proceso estocástico tiene límite por la derecha (por la izquierda). \\

El movimiento Browniano está estrechamente relacionado con la distribución Gaussiana, por lo que la siguiente definición será de ayuda para los siguientes resultados.

\begin{definition}
Decimos que $B = (B_t, t \geq 0)$ es un movimiento Browniano real que empieza en 0 si $B$ es un proceso Gaussiano centrado tal que para toda $s, t \geq 0$
\begin{align}
\mathbb{E}[B_s B_t] = COV(B_s, B_t) = s \wedge t.
\end{align}

El proceso $B$ es conocido como movimiento Browniano estándar.
\end{definition}

\begin{proposition}
El proceso $X$ es un movimiento Browniano si y solamente si
\begin{enumerate}
\item $X_0 = 0$ c.s.
\item Para $n \geq 2$ y para toda $0 \leq t_1 \leq t_2 \leq \cdots \leq t_n$, los incrementos
  \begin{align*}
  	X_{t_1}, X_{t_2} - X_{t_1}, \ldots, X_{t_n} - X_{t_{n-1}},
  \end{align*}
son independientes.
\item Para toda $0 \leq s \leq t$, $X_t - X_s \sim N(0, t - s)$.
\end{enumerate}
\end{proposition}

\begin{proof}
Veamos primero que el recíproco se cumple. Si $1$, $2$ y $3$ se cumplen, mostremos que $X$ es un movimiento Browniano. Consideremos tiempos tales que $0 \leq t_1 \leq t_2 \leq \cdots \leq t_n$ y un vector Gaussiano $(X_{t_1}, X_{t_2} - X_{t_1}, \ldots, X_{t_n} - X_{t_{n-1}})$. \\

Entonces de $(3)$ tenemos que, como $(X_{t_1}, X_{t_2} - X_{t_1}, \ldots, X_{t_n} - X_{t_{n-1}})$ es un vector gaussiano, es claro que $(X_{t_1}, X_{t_2}, \ldots, X_{t_n})$ también lo es, además, resulta ser un vector gaussiano centrado ya que para toda $i$ se tiene que $X_{t_i} \sim N(0, t_i)$. \\

Ahora, supongamos que $s \leq t$, entonces de $1$ y $3$ tenemos que $X_s$ y $X_t - X_s$ son variables aleatorias independientes, lo que implica que
  \begin{align*}
  	\mathbb{E}[X_t X_s] &= \mathbb{E}[(X_t - X_s + X_s) X_s] \\
    &= \mathbb{E}[(X_t - X_s) X_s] + \mathbb{E}[X_s^2] \\
    &= \mathbb{E}[X_t - X_s] \mathbb{E}[X_s] + \mathbb{E}[X_s^2] \\
    &= \mathbb{E}[X_s^2] \\
    & = s.
  \end{align*}
En general, $\mathbb{E}[X_t X_s] = s \wedge t$ para toda $s, t \geq 0$. Por lo tanto, $X$ es un movimiento Browniano. \\

Supongamos ahora que $X$ es un movimiento Browniano. Primero observemos que $\mathbb{E} [X_0^2] = \mathbb{E} [X_0 X_0] = 0$, por lo tanto, tenemos que $X_0 = 0$ c.s. Por otra parte, para $0 \leq s \leq t$, tenemos que la variable $X_t - X_s$ es una variable aleatoria Gaussiano centrada, pues X es un proceso Gaussiano centrado. Además observamos que la varianza de la variable $X_t - X_s$ es
  \begin{align*}
  	\mathbb{E} [(X_t - X_s)^2] &= \mathbb{E} [X_t^2] + \mathbb{E} [X_s^2] - 2 \mathbb{E} [X_t X_s] \\
    & = t + s - 2 s \\
    &= t - s.
  \end{align*}
Por lo tanto, tenemos que $X_t - X_s \sim N(0, t - s)$. \\

Por último, veamos que los incrementos de $X$ son independientes. Consideremos tiempos tales que $0 \leq t_1 \leq t_2 \leq \cdots \leq t_n$, como el vector $(X_{t_1}, X_{t_2}, \ldots, X_{t_n})$ es Gaussiano entonces
  \begin{align*}
  	(X_{t_1}, X_{t_2} - X_{t_1}, \ldots, X_{t_n} - X_{t_{n-1}}),
  \end{align*}
también lo es y más aún, para toda $i < j$, tenemos
  \begin{align*}
  	COV(X_{t_{i + 1}} - X_{t_i}, X_{j_{i + 1}} - X_{t_j}) &= \mathbb{E} [(X_{t_{i + 1}} - X_{t_i})(X_{t_{j + 1}} - X_{t_j})] \\
    & = \mathbb{E} [X_{t_{i + 1}}X_{t_{j + 1}} - X_{t_i}X_{t_{j + 1}} - X_{t_{i + 1}}X_{t_j} + X_{t_i}X_{t_j}] \\
    & = \mathbb{E} [X_{t_{i + 1}}X_{t_{j + 1}}] - \mathbb{E}[X_{t_i}X_{t_{j + 1}}] \\
    & - \mathbb{E}[X_{t_{i + 1}}X_{t_j}] + \mathbb{E}[X_{t_i}X_{t_j}] \\
    & = t_{i+1} + t_i - t_{i+1} - t_i \\
    & = 0,
  \end{align*}
De la Proposición \ref{gaussiano1} vemos que, las variables $X_{t_1}, X_{t_2} - X_{t_1}, \ldots, X_{t_n} - X_{t_{n-1}}$ son independientes.
\end{proof}

% A continuación se da una prueba de que el movimiento Browniano existe


\begin{theorem}[Wiener, 1923]
El movimiento Browniano existe.
\end{theorem}
\begin{proof}
\textbf{(Lévy, 1948)} Para la prueba, consideremos la construcción del movimiento Browniano en el intervalo $[0, 1]$. La idea es construir el movimiento Browniano como límite uniforme de funciones continuas. \\

Primero vamos a construir al movimiento como un elemento aleatorio del espacio $\mathbb{C}[0, 1]$. Para ello, vamos a definir al conjunto de puntos diádicos en el intervalo $[0, 1]$, es decir, 
	\begin{align*}
		\mathcal{D}_n = \left\{ \frac{k}{2^n} : 0 \leq k \leq 2^k \right\}.
	\end{align*}
Vamos a interpolar linealmente a través de estos puntos y se verificará que el límite uniforme de estas funciones continuas existe y que éste cumpla con las propiedades del movimiento Browniano. Al final, se construye al movimiento Browniano en $\mathbb{R}_{+}$\\

Consideremos una familia de variables aleatorias $(\xi_{k, n}, 0 \leq k \leq 2^n, n \geq 1)$ Gaussianas centradas y con varianza unitaria. Definamos al proceso $(X_n (t), t \in [0, 1], n \geq 1)$ de manera recursiva, como sigue

	\begin{enumerate}
	\item Sea $X_0 (0) = 0$, $X_0 (1) = \xi_{0, 0}$ con $X_0$ lineal en $[0, 1]$.
	\item Sea 
		\begin{align*}
		X_1 (t) = 
			\begin{cases}
			0, & \text{ si } t = 0, \\
			X_0 (\frac{1}{2}) + \frac{\xi_{1, 1}}{2}, & \text{ si } t = \frac{1}{2}, \\
			X_0 (1) & \text{ si } t = 1.
			\end{cases}
		\end{align*}
con $X_1$ lineal en $[0, \frac{1}{2}]$ y en $(\frac{1}{2}, 1]$. 
	\item Considerando el caso para $n = 2$ tenemos que
		\begin{align*}
		X_2 (t) = 
			\begin{cases}
			0, & \text{ si } t = 0, \\
			X_1 (\frac{1}{4}) + \frac{\xi_{1, 2}}{2 \sqrt{2}}, & \text{ si } t = \frac{1}{4}, \\
			X_1 (\frac{1}{2}), & \text{ si } t = \frac{1}{2}, \\
			X_1 (\frac{3}{4}) + \frac{\xi_{3, 2}}{2 \sqrt{2}}, & \text{ si } t = \frac{3}{4}, \\
			X_1 (1), & \text{ si } t = 1.
			\end{cases}
		\end{align*}
con $X_2$ lineal en $[0, \frac{1}{4}]$, $(\frac{1}{4}, \frac{1}{2}]$, $(\frac{1}{2}, \frac{3}{4}]$ y $(\frac{3}{4}, 1]$.
	\end{enumerate}
En general, para toda $n \geq 0$, el mapeo $t \rightarrow X_n (t)$ es lineal en cada uno de los intervalos de la forma
	\begin{align*}
		\left[ \frac{k}{2^n}, \frac{k+1}{2^n} \right],
	\end{align*}
y además, 
	\begin{align*}
		X_n \left(\frac{2j}{2^n}\right) & := X_{n-1} \left(\frac{j}{2^{n-1}}\right), \\
		X_n \left(\frac{2j + 1}{2^n}\right) & := X_{n-1} \left(\frac{2j + 1}{2^n}\right) + \frac{\xi_{2j + 1, n}}{2^{n/2}}. \\
	\end{align*}
Veamos que, para toda $n \geq 0$, el vector $(X_n(k / 2^n), 0 \leq k \leq 2^n)$ es Gaussiano, centrado y de covarianza
	\begin{align}
		\mathbb{E} \left[ X_n \left( \frac{k}{2^n} \right) X_n \left( \frac{\ell}{2^n} \right) \right] = \frac{k}{2^n} \wedge \frac{\ell}{2^n}. \label{abv}
	\end{align}
Para $n = 0$ tenemos que $X_0$ es lineal y su extremo $X_0(1)$ es  una variable aleatoria gaussiana centrada, en otras palabras, cualquier $X_0(t)$ es combinación lineal de variables aleatorias gaussianas, más aún
	\begin{align*}
		\mathbb{E} \left[ X_0 (0) X_0 (1) \right] & = \mathbb{E} \left[ 0 \cdot X_0 (1) \right] \\
		& = 0 \\
		& = \frac{0}{2^n} \wedge \frac{1}{2^n}.
	\end{align*}
Supongamos valido el caso para $n - 1$. Observemos que $(X_n (k / 2^n), 0 \leq k \leq 2^n)$ es una combinación lineal del vector Gaussiano $(X_{n-1} (k / 2^{n-1}), 0 \leq k \leq 2^{n-1})$ y de la familia de variables aleatorias Gaussianas $(\xi_{k, n}, 0 \leq k \leq 2^n)$, donde ambos componentes son independientes, por lo que, $(X_n (k / 2^n), 0 \leq k \leq 2^n)$  es un vector Gaussiano. \\

Basta mostrar el caso en que $\ell = k$, pues si $\ell > k$ hacemos
	\begin{align*}
		\mathbb{E} \left[ X_n \left( \frac{k}{2^n} \right) X_n \left( \frac{\ell}{2^n} \right) \right] & = \mathbb{E} \left[ X_n \left( \frac{k}{2^n} \right) \left( X_n \left( \frac{\ell}{2^n} \right)  - X_n \left( \frac{k}{2^n} \right) + X_n \left( \frac{k}{2^n} \right) \right) \right] \\
        & = \mathbb{E} \left[ X_n \left( \frac{k}{2^n} \right) \left( X_n \left( \frac{\ell}{2^n} \right)  - X_n \left( \frac{k}{2^n} \right) \right) \right] + \mathbb{E} \left[ X^2_n \left( \frac{k}{2^n} \right) \right] \\
        & = \mathbb{E} \left[ X^2_n \left( \frac{k}{2^n} \right) \right].
	\end{align*}
Entonces, supongamos que $\ell = k$, y veamos que,
	\begin{align*}
		\mathbb{E} \left[ X_n^2 \left( \frac{k}{2^n} \right) \right] & = \frac{k}{2^n}.
	\end{align*}
Para el caso en el que $k$ es par, tenemos que $X_n (k / 2^n) = X_n (2j / 2^n) = X_{n-1} (j / 2^{n-1})$ entonces
	\begin{align*}
		\mathbb{E} \left[ X_n^2 \left( \frac{k}{2^n} \right) \right] & = \mathbb{E} \left[ X_n^2 \left( \frac{2j}{2^n} \right) \right] \\
		& = \mathbb{E} \left[ X_{n-1}^2 \left( \frac{j}{2^{n-1}} \right) \right] \\
		& = \frac{j}{2^{n-1}},
	\end{align*}
Pero, $k = 2j$, entonces $j = k/2$, por lo que 
	\begin{align*}
		\mathbb{E} \left[ X_n^2 \left( \frac{k}{2^n} \right) \right] = \frac{j}{2^{n-1}} = \frac{k}{2^n}.
	\end{align*}
Para el caso en el que $k$ es impar tenemos que
	\begin{align*}
		X_n \left( \frac{k}{2^n} \right) & = X_n \left( \frac{2j + 1}{2^n} \right) \\
		& = X_{n-1} \left( \frac{2j + 1}{2^n} \right) + \frac{\xi_{2j+1, n}}{2^{(n+1)/2}} \\
		& = X_{n-1} \left( \frac{2j}{2^n} \right) + X_{n-1} \left( \frac{2(j + 1)}{2^n} \right) + \frac{\xi_{2j+1, n}}{2^{(n+1)/2}} \\
		& = \frac{1}{2} \left( X_{n-1} \left( \frac{j}{2^{n-1}} \right) + X_{n-1} \left( \frac{j + 1}{2^{n-1}} \right) \right) + \frac{\xi_{2j+1, n}}{2^{(n+1)/2}}. \\
	\end{align*}
Por lo tanto, 
	\begin{align*}
		\mathbb{E} \left[ X_n^2 \left( \frac{k}{2^n} \right) \right] & = \mathbb{E} \left[ \left( \frac{1}{2} \left( X_{n-1} \left( \frac{j}{2^{n-1}} \right) + X_{n-1} \left( \frac{j + 1}{2^{n-1}} \right) \right) + \frac{\xi_{2j+1, n}}{2^{(n+1)/2}} \right)^2 \right] \\
		& = \mathbb{E} \left[ \frac{1}{4} \left( X_{n-1} \left( \frac{j}{2^{n-1}} \right) + X_{n-1} \left( \frac{j + 1}{2^{n-1}} \right) \right)^2 + \left( \frac{\xi_{2j+1, n}}{2^{(n+1)/2}} \right)^2 \right] \\
		& + \mathbb{E} \left[ \left(\frac{\xi_{2j+1, n}}{2^{(n+3)/2}}\right) \left( X_{n-1} \left( \frac{j}{2^{n-1}} \right) + X_{n-1} \left( \frac{j + 1}{2^{n-1}} \right) \right) \right] \\	
		& = \mathbb{E} \left[ \frac{1}{4} \left( X_{n-1} \left( \frac{j}{2^{n-1}} \right) + X_{n-1} \left( \frac{j + 1}{2^{n-1}} \right) \right)^2 \right] + \mathbb{E} \left[ \left( \frac{\xi_{2j+1, n}}{2^{(n+1)/2}} \right)^2 \right] \\
		& + \mathbb{E} \left[\frac{\xi_{2j+1, n}}{2^{(n+3)/2}} \right] \mathbb{E} \left[ X_{n-1} \left( \frac{j}{2^{n-1}} \right) + X_{n-1} \left( \frac{j + 1}{2^{n-1}} \right) \right],
	\end{align*}
Sabemos que $\xi$ es una variable Gaussiana independiente de $X_n$, además de tener media igual a $0$ y varianza unitaria. Entonces,
	\begin{align*}
		\mathbb{E} \left[ X_n^2 \left( \frac{k}{2^n} \right) \right] & = \frac{1}{4} \mathbb{E} \left[ \left( X_{n-1} \left( \frac{j}{2^{n-1}} \right) + X_{n-1} \left( \frac{j + 1}{2^{n-1}} \right) \right)^2 \right] + \frac{1}{2^{n+1}}.
	\end{align*} 
Desarrollando el cuadrado y utilizando nuestro supuesto de covarianza para el proceso $X_n$ tenemos que
	\begin{align*}
	\mathbb{E} \left[ X_n^2 \left( \frac{k}{2^n} \right) \right] & = \frac{1}{4} \left[ \mathbb{E} \left[ X_{n-1}^2 \left( \frac{j}{2^{n-1}}  \right) \right] +  \mathbb{E} \left[ X_{n-1}^2 \left( \frac{j+1}{2^{n-1}} \right) \right] \right. \\
	& \left. + 2\mathbb{E} \left[ X_{n-1}\left( \frac{j}{2^{n-1}}  \right) X_{n-1} \left( \frac{j+1}{2^{n-1}} \right) \right] \right] + \frac{1}{2^{n+1}} \\
	& = \frac{1}{4} \left( \frac{j}{2^{n-1}} \right) + \frac{1}{4} \left( \frac{j+1}{2^{n-1}} \right) + \frac{1}{2} \left( \frac{j}{2^{n-1}} \right) + \frac{1}{2^{n+1}} \\
	& = \frac{4j + 2}{2^{n+1}} = \frac{2j + 1}{2^n}.
	\end{align*}
Por lo tanto, para toda $n \geq 0$, $(X_n (t), t \in [0, 1])$ es un proceso Gaussiano centrado y de covarianza definida en (\ref{abv}). Ahora probemos que casi seguramente el proceso $(X_n (t), 0 \leq t \leq 1)$ converge uniformemente en el intervalo $[0, 1]$. Definamos los siguientes conjuntos, 
	\begin{align*}
		A_n = \left\{ \sup_{t \in [0, 1]} |X_n (t) - X_{n-1} (t)| \geq 2^{-n / 4} \right\}.
	\end{align*}
Notemos que, 
	\begin{align*}
		\mathbb{P}(A_n) & = \mathbb{P} \left( \bigcup_{j = 0}^{2n - 1} \left\{ \sup_{t \in [j/2^n, (j+1)/2^n]} |X_n (t) - X_{n-1} (t)| \geq 2^{-n / 4} \right\} \right) \\
		& = \mathbb{P} \left( \bigcup_{j = 0}^{2n - 1} \left\{ \bigg| \frac{\xi_{2j+1, n}}{2^{(n+1)/2}} \bigg| \geq 2^{-n / 4} \right\} \right) \\
		& \leq 2^n \mathbb{P}\left( |N(0, 1)| \geq 2^{(n+2)/4} \right) \\
		& = 2^{n+1} \mathbb{P}\left( N(0, 1) \geq 2^{(n+2)/4} \right) \\
		& \leq \frac{1}{\sqrt{\pi}} 2^{\frac{3n}{4}} e^{-2^{n/2}},
	\end{align*}
Lo cual implica que $\sum \mathbb{P}(A_n) < \infty$, y entonces, por la Ley de Borel - Cantelli tenemos que 
\begin{align*}
\mathbb{P} \left( \limsup_n A_n \right) = 0.
\end{align*}

Luego, tenemos que existe un conjunto $\tilde{\Omega} \in \mathcal{F}$ con $\mathbb{P}(\tilde{\Omega}) = 1$, tal que, para todo $\omega \in \tilde{\Omega}$ existe $N(\omega) < \infty$ donde
	\begin{align*}
		\sup_{t \in [0, 1]} |X_n (t) - X_{n-1} (t)| \leq 2^{-n / 4} \hspace{0.5cm} \forall n \geq N.
	\end{align*}
Es decir, casi seguramente el proceso $X_n (t)$ converge uniformemente con $t \in [0, 1]$. Al ser, $(X_n (t); n \geq 0, 0 \leq t \leq 1)$ uniformemente convergente en $\mathbb{C}([0, 1], \mathbb{R})$, dicho límite es una función continua, ya que $\mathbb{C}([0, 1], \mathbb{R})$ es completo bajo la topología uniforme. Sea 
	\begin{align*}
		X(t) := \lim_{n \rightarrow \infty} X_n (t), \hspace{0.5cm} t \in [0, 1], 
	\end{align*}
entonces, por construcción, el proceso $(X(t), 0 \leq t \leq 1)$ es un proceso Gaussiano que satisface
	\begin{align*}
		\mathbb{E} \left[ X(s) X(t) \right] & = \lim_{n \rightarrow \infty} \mathbb{E} [X_n (s_n) X_t (t_n)] \\
		& = \lim_{n \rightarrow \infty} s_n \wedge t_n \\
		& = s \wedge t, 
	\end{align*}
donde $(s_n)$ y $(t_n)$ son diádicos de la forma $j / 2^n$ y tales que $s_n \rightarrow s$, $t_n \rightarrow t$. Por lo tanto, $(X (t), 0 \leq t \leq 1)$ es un movimiento Browniano en el intervalo $[0, 1]$. \\

Para finalizar, consideremos $B_t^0, B_t^1, B_t^2, \ldots$ movimientos Brownianos independientes en el intervalo $[0, 1]$ y definamos
	\begin{align*}
		B_t = 
		\begin{cases}
		B_t^0, & \text{ si } t \in [0, 1] \\
		B_{t-n}^n + \sum_{m = 0}^{n-1} B_1^m, & \text{ si } t \in [n, n+1)
		\end{cases}
	\end{align*}
Entonces, por último veamos que $B_t$ es un movimiento Browniano en $\mathbb{R}_{+}$. \\

Recordemos que la combinación lineal de un proceso Gaussiano centrado es de nuevo, un proceso Gaussiano, por lo tanto, de la definición de $B_t$, tenemos que es un proceso Gaussiano centrado. 

Además, veamos que para todo $s, t \in \mathbb{R}_{+}$ se cumple que
\begin{align*}
\mathbb{E}[B_s B_t] = s \wedge t.
\end{align*}
Consideremos primero el caso en que, $s, t \in [n, n+1)$ para cualquier $n$, con $s < t$. Tenemos que
\begin{align*}
\mathbb{E}[B_s B_t] & = \mathbb{E} \left[ \left( B_{s-n}^n + \sum_{m=0}^{n-1} B_1^m \right) \left( B_{t-n}^n + \sum_{m=0}^{n-1} B_1^m \right) \right] \\
& = \mathbb{E}[B_{s-n}^n B_{t-n}^n] + \mathbb{E} \left[ \sum_{m=0}^{n-1} B_{s-n}^n B_1^m \right] + \mathbb{E} \left[ \sum_{m=0}^{n-1} B_{t-n}^n B_1^m \right] + \mathbb{E} \left[ \sum_{m=0}^{n-1} \left( B_1^m \right)^2 \right] \\
& = \mathbb{E}[B_{s-n}^n B_{t-n}^n] + \sum_{m=0}^{n-1} \mathbb{E} \left[ B_{s-n}^n B_1^m \right] + \sum_{m=0}^{n-1} \mathbb{E} \left[ B_{t-n}^n B_1^m \right] + \sum_{m=0}^{n-1} \mathbb{E} \left[ \left( B_1^m \right)^2 \right].
\end{align*}

De la independencia de los movimientos $B^0, B^1, B^2, \ldots, B^{n-1}, B^n$ y al ser procesos Gaussianos centrados; tenemos que $\mathbb{E}[B^m B^n] = 0$ para toda $m < n$. Por lo tanto
\begin{align*}
\mathbb{E}[B_s B_t] & = \mathbb{E}[B_{s-n}^n B_{t-n}^n] + \sum_{m=0}^{n-1} \mathbb{E} \left[ \left( B_1^m \right)^2 \right].
\end{align*}

Como $B^n_t$ es un proceso Gaussiano, sabemos que tiene una varianza igual a $1$, entonces
\begin{align*}
\mathbb{E}[B_s B_t] & = \mathbb{E}[B_{s-n}^n B_{t-n}^n] + \sum_{m=0}^{n-1} \mathbb{E} \left[ \left( B_1^m \right)^2 \right] \\
& = (s - n) \wedge (t - n) + n \\ 
& = (s \wedge t).
\end{align*}

Para el caso en que, los tiempos $s$ y $t$ no pertenezcan al mismo intervalo de tiempo, observemos que para $s \in [m, m+1)$ y $t \in [n, n+1)$ con $m < n$ se tiene
\begin{align*}
\mathbb{E}[B_s B_t] & = \mathbb{E} \left[ \left( B_{s-m}^m + \sum_{j=0}^{m-1} B_1^j \right) \left( B_{t-n}^n + \sum_{k=0}^{n-1} B_1^k \right) \right] \\
& = \mathbb{E}[B_{s-m}^m B_{t-n}^n] + \mathbb{E} \left[ \sum_{k=0}^{n-1} B_{s-m}^m B_1^k \right] + \mathbb{E} \left[ \sum_{j=0}^{m-1} B_{t-n}^n B_1^j \right] + \mathbb{E} \left[ \sum_{j=0}^{m-1} B_1^j \sum_{k=0}^{n-1} B_1^k \right] \\
& = \mathbb{E}[B_{s-m}^m B_{t-n}^n] + \sum_{k=0}^{n-1} \mathbb{E} \left[ B_{s-m}^m B_1^k \right] + \sum_{j=0}^{m-1} \mathbb{E} \left[ B_{t-n}^n B_1^j \right] + \mathbb{E} \left[ \sum_{j=0}^{m-1} B_1^j \sum_{k=0}^{n-1} B_1^k \right]
\end{align*}
Al ser $B^m$ y $B^n$ independientes para todo $m \neq n$ y procesos Gaussianos centrados, tenemos que $\mathbb{E}[B_{s-m}^m B_{t-n}^n] = 0$; lo mismo ocurre para la primera y segunda suma excepto cuando su índice superior coincide, de ahí tenemos que $\mathbb{E}[B^m_{s-m} B^m_1] = s-m$, al ser $B^m$ un movimiento Browniano en $[0, 1]$. Por lo tanto,
\begin{align}
\mathbb{E}[B_s B_t] & = \mathbb{E}[B_{s-m}^m B_{t-n}^n] + \sum_{k=0}^{n-1} \mathbb{E} \left[ B_{s-m}^m B_1^k \right] \nonumber \\
& + \sum_{j=0}^{m-1} \mathbb{E} \left[ B_{t-n}^n B_1^j \right] + \mathbb{E} \left[ \sum_{j=0}^{m-1} B_1^j \sum_{k=0}^{n-1} B_1^k \right] \nonumber \\
& = (s - m) + \mathbb{E} \left[ \sum_{j=0}^{m-1} B_1^j \sum_{k=0}^{n-1} B_1^k \right] \label{mbr1}.
\end{align}
Desarrollando la última expresión vemos que
\begin{align}
	\mathbb{E} \left[ \sum_{j=0}^{m-1} B_1^j \sum_{k=0}^{n-1} B_1^k \right]  & = \sum_{j=0}^{m-1} \sum_{k=0}^{n-1} \mathbb{E} \left[ B_1^j  B_1^k \right] \nonumber \\
    & = \sum_{j=0}^{m-1} 1 = m. \label{mbr2}
\end{align}
Utilizando (\ref{mbr2}) en (\ref{mbr1}) tenemos que
\begin{align*}
\mathbb{E}[B_s B_t] & = (s - m) + \mathbb{E} \left[ \sum_{j=0}^{m-1} B_1^j \sum_{k=0}^{n-1} B_1^k \right] \\
& = (s - m) + m  \\
& = s = s \wedge t.
\end{align*}
Por lo tanto $B_t$ es un movimiento Browniano en $\mathbb{R}_{+}$.
\end{proof}

\section{Trayectorias del movimiento Browniano}
En esta sección, estudiaremos las propiedades trayectoriales del movimiento Browniano, y más específicamente probaremos el Criterio de Continuidad de Kolmogorov, resultado que nos asegura existe una versión continua del movimiento Browniano.

\begin{definition}
Sean $X = (X_t, t \geq 0)$ y $\tilde{X} = (\tilde{X}_t, t \geq 0)$ dos procesos definidos en un conjunto $\mathbb{R}_{+}$. Decimos que $\tilde{X}$ es una modificación de $X$ si para toda $t \in \mathbb{R}_{+}$, $X_t = \tilde{X}_t$ c.s.
\end{definition}

Si $\tilde{X}$ es una modificación de $X$, para toda n y para toda $(t_1, t_2, \ldots, t_n) \in \mathbb{R}_{+}^n$, entonces
  \begin{align*}
	(X_{t_1}, X_{t_2}, \ldots, X_{t_n}) & = (\tilde{X}_{t_1}, \tilde{X}_{t_2}, \ldots, \tilde{X}_{t_n}) \hspace{0.5cm} \text{c.s.}
  \end{align*}
En particular, si alguno de los procesos es un movimiento Browniano, el otro también. Sin embargo, $X$ y $\tilde{X}$ pueden tener trayectorias completamente diferentes.

\begin{definition}
Decimos que $X$ y $\tilde{X}$ son indistinguibles si
  \begin{align*}
	\mathbb{P} \left(X_t = \tilde{X}_t, \hspace{0.2cm} \forall t \in \mathbb{R}_{+} \right) = 1 
  \end{align*}
Si $X$ y $\tilde{X}$ son indistinguibles entonces $\tilde{X}$ es una modificación de $X$, además, ambos procesos tienen las mismas trayectorias casi seguramente.
\end{definition}

Por otra parte, observemos que si $\tilde{X} = (\tilde{X}_t, t \in \mathbb{R}_{+})$ es una modificación de $X = (X_t, t \in \mathbb{R}_{+})$ y ambos tienen trayectorias continuas casi seguramente, entonces $X$ y $\tilde{X}$ son indistinguibles.

\begin{definition}
Sean dos procesos estocásticos $(X_t, t \in \mathbb{R}_{+})$ y $(\tilde{X}_t, t \in \mathbb{R}_{+})$ definidos en los espacios de probabilidad $(\Omega, \mathcal{F}, \mathbb{P})$ y $(\tilde{\Omega}, \tilde{\mathcal{F}}, \tilde{\mathbb{P}})$, respectivamente, y con el mismo espacio de estados $(E, \mathcal{E})$.   
  
Decimos que $X$ y $\tilde{X}$ son equivalente si para cualquier sucesión finita $t_1, t_2, \ldots, t_n$ donde $0 \leq t_1 < t_2 < \cdots < t_n < \infty$ y toda $A_i \in \mathcal{E}$ con $1 \leq i \leq n$,
	\begin{align*}
	\mathbb{P} \left( X_{t_1} \in A_1 , X_{t_2} \in A_2, \ldots, X_{t_n} \in A_n \right) = \mathbb{P} \left( \tilde{X}_{t_1} \in A_1, \tilde{X}_{t_2} \in A_2, \ldots, \tilde{X}_{t_n} \in A_n \right).
	\end{align*}
\end{definition}

\begin{theorem}[Criterio de Continuidad de Kolmogorov]
Sea $X = (X_t, t \in I)$ un proceso aleatorio indexado por un intervalo $I \subset \mathbb{R}$ con valores en $(E, d)$ un espacio métrico completo. Supongamos que existen $p, \epsilon, c > 0$ tal que
	\begin{align*}
	\mathbb{E} \left[\left( d(X_s, X_t)\right)^p\right] \leq c|t - s|^{1 + \epsilon}, \hspace{0.5cm} s, t \in I.
	\end{align*}
Entonces, existe una modificación $\tilde{X}$ de $X$, cuyas trayectorias son localmente Hölder de índice $\alpha$, para toda $\alpha \in (0, \epsilon / p)$, es decir, para toda $T \geq 0$,
	\begin{align*}
	d \left( \tilde{X}_s(\omega), \tilde{X}_t(\omega)\right) \leq K(\omega, T) |t - s|^{\alpha}, \hspace{0.5cm} s, t \leq T.
	\end{align*}
\end{theorem}
\begin{proof}
La demostración de este importante resultado se basa en tres problemas. 1. Comprobar que las trayectorias del proceso $\tilde{X}$ son $\alpha$-Hölder continuas en el conjunto de números diádicos del intervalo $[0,1]$. 2. ``Extender'' al proceso $\tilde{X}$ para un intervalo $[0, T]$. 3. Finalmente, mostrar que $\tilde{X}$ está bien definido y que resulta ser una modificación de nuestro proceso original. \\

Consideremos sin pérdida de generalidad que $I$ es un intervalo acotado y para simplificar la notación, definamos $I = [0, 1]$. De la desigualdad de Chebyshev veamos que para toda $a > 0$ y para toda $s, t \in [0, 1]$ tenemos que
	\begin{align}
	\mathbb{P} \left(d(X_s, X_t) > a \right) \leq \frac{\mathbb{E} \left[\left(d(X_s, X_t)^p\right)\right]}{a^p} \leq c \frac{|t-s|^{1+\epsilon}}{a^p}.\label{aca} 
	\end{align}
Si consideramos a $D_n$ como el conjunto de números diádicos del intervalo $[0, 1]$ tales que $D_n = \{k / 2^n : k = 0, 1, \ldots, 2^n\}$, entonces $D_0, D_1, \ldots $ es una sucesión de conjuntos crecientes, por lo que podemos definir al siguiente conjunto $D = \cup_n D_n = \lim_n D_n$. \\

Tomemos a $s = (i-1)/2^n$, $t = i/2^n$ y $a = 1/2^{n \alpha}$. De (\ref{aca}), con $\alpha$ fija en $(0, \epsilon / p)$ obtenemos
	\begin{align*}
	\mathbb{P} \left( d(X_{\frac{i-1}{2^n}}, X_{\frac{i}{2^n}}) > \frac{1}{2^{n \alpha}} \right) & \leq c \frac{| i/2^n - (i-1)/2^n|^{1+\epsilon}}{2^{- n \alpha p}} \\
	& = \frac{c}{2^{n(1 + \epsilon - \alpha p)}}.
	\end{align*}
Como $\alpha \in (0, \epsilon / p)$ entonces, $\epsilon - p \alpha > 0$, por lo tanto
	\begin{align*}
	\sum_{n \geq 1} \mathbb{P} \left( \max_{1 \leq i \leq 2^n} d(X_{\frac{i-1}{2^n}}, X_{\frac{i}{2^n}}) > \frac{1}{2^{n \alpha}} \right)  & \leq \sum_{n \geq 1} \sum_{i=1}^{2n} \mathbb{P} \left( d(X_{\frac{i-1}{2^n}}, X_{\frac{i}{2^n}}) > \frac{1}{2^{n \alpha}} \right) \\
	& = \sum_{n \geq 1} \frac{c2^n}{2^{n(1 + \epsilon - \alpha p)}} \\
	& = \sum_{n \geq 1} \frac{c}{2^{n(\epsilon - \alpha p)}} < \infty
	\end{align*}
En virtud del Lema de Borel-Cantelli sabemos que existe un conjunto $\tilde{\Omega} \subset \Omega$, $\tilde{\Omega} \in \mathcal{F}$, con $\mathbb{P} (\tilde{\Omega}) = 1$ tal que para cada $\omega \in \tilde{\Omega}$ existe una $n_0 := n_0 (\omega) < \infty$ tal que para toda $n \geq n_0$, 
	\begin{align}
	\max_{1 \leq i \leq 2^n} d \left(X_{\frac{i-1}{2^n}}, X_{\frac{i}{2^n}}\right) \leq 2^{-n \alpha}, \hspace{0.3cm} n \geq n_0. \label{acb}
	\end{align}
Por lo tanto, 
	\begin{align*}
	\sup_{n \geq n_0} \max_{1 \leq i \leq 2^n} \frac{d \left(X_{\frac{i-1}{2^n}}, X_{\frac{i}{2^n}}\right)}{2^{-n \alpha}} \leq 1.
	\end{align*}
Entonces, 
	\begin{align*}
	\sup_{n \geq 1} \max_{1 \leq i \leq 2^n} \frac{d \left(X_{\frac{i-1}{2^n}}, X_{\frac{i}{2^n}}\right)}{2^{-n \alpha}} \leq K(\omega).
	\end{align*}
Veamos que lo anterior implica la condición $\alpha$-Hölder que buscamos observar en las trayectorias del proceso $X$. \\

Recordemos que $t \in D$ si y solo si podemos escribir a $t$ como $\sum_{k=1}^N \zeta_k / 2^k$ donde $\zeta_k \in \{0, 1\}$. Consideremos dos elementos $s, t \in D$ tales que $s < t$. Sea $q \geq 1$ el entero más grande donde la desigualdad $t - s < 2^{-q}$ se cumple. Entonces, existen $0 \leq k \leq 2^{q}$ y enteros $l, m \geq 0$ tales que
	\begin{align*}
	s = \frac{k}{2^q} + \frac{\zeta_{q+1}}{2^{q+1}} + \frac{\zeta_{q+2}}{2^{q+2}} + \ldots + \frac{\zeta_{q+l}}{2^{q+l}}, \hspace{0.3cm} \zeta_j = 0 \text{ ó } 1; \\
	t = \frac{k}{2^q} + \frac{\zeta_{q+1}}{2^{q+1}} + \frac{\zeta_{q+2}}{2^{q+2}} + \ldots + \frac{\zeta_{q+m}}{2^{q+m}}, \hspace{0.3cm} \tilde{\zeta}_j = 0 \text{ ó } 1.
	\end{align*}
Donde $k = [2^q s]$, en el cual $[x]$ es tal que $[x] \leq x \leq [x+1]$. Notemos que $k \leq [2^q t] \leq [2^q (s + 2^{-q})] = k+1$. Definamos los siguiente números, 
	\begin{align*}
	s_i & = \frac{k}{2^q} + \frac{\zeta_{q+1}}{2^{q+1}} + \frac{\zeta_{q+2}}{2^{q+2}} + \ldots + \frac{\zeta_{q+i}}{2^{q+i}}, \hspace{0.3cm} i = 0, 1, \ldots, l \\
	t_j & = \frac{k}{2^q} + \frac{\zeta_{q+1}}{2^{q+1}} + \frac{\zeta_{q+2}}{2^{q+2}} + \ldots + \frac{\zeta_{q+j}}{2^{q+j}}, \hspace{0.3cm} j = 0, 1, \ldots, m.
	\end{align*}
Entonces, de la desigualdad del triángulo obtenemos
	\begin{align*}
	d(X_s, X_t) & = d(X_{s_l}, X_{t_m}) \\
	& \leq d(X_{s_0}, X_{t_0}) + \sum_{i=1}^l d(X_{s_i}, X_{s_{i-1}}) + \sum_{j=1}^m d(X_{t_j}, X_{t_{j-1}}) \\
	& \leq \sum_{i=1}^l K(\omega) 2^{-(q+i) \alpha} + \sum_{j=1}^m K(\omega) 2^{-(q+j) \alpha} \hspace{0.3cm} \text{ (haciendo uso de (\ref{acb})) } \\
	& = K(\omega) 2^{-q \alpha} \left[ \sum_{i=1}^l 2^{-i \alpha} + \sum_{j=1}^m 2^{-j \alpha} \right] \\
	& \leq 2K(\omega) 2^{-q \alpha} \sum_{i=1}^{\infty} 2^{-i \alpha} \\
	& = 2K(\omega) 2^{-q \alpha} \left[1 + \sum_{i=1}^{\infty} 2^{-i \alpha} \right]  \hspace{0.3cm} \text{ (serie geométrica) } \\
	& = 2K(\omega) 2^{-q \alpha} \left[1 + \frac{2^{- \alpha}}{1 - 2^{- \alpha}} \right] \\
	& = 2K(\omega) 2^{-q \alpha} \left[ \frac{1}{1 - 2^{- \alpha}} \right] = \frac{2K(\omega) 2^{-q \alpha}}{1 - 2^{- \alpha}}.
	\end{align*}
Por último, tenemos que $2^{-(q+1)} < t-s$, entonces
	\begin{align*}
	d(X_s, X_t) & \leq \frac{2K(\omega) 2^{-q \alpha}}{1 - 2^{- \alpha}} \\
	& \leq \frac{2^{\alpha + 1}K(\omega)}{1 - 2^{- \alpha}} (t - s)^{\alpha}.
	\end{align*}
Por lo tanto, para toda $\omega \in \tilde{\Omega}$, la trayectoria $t \rightarrow X_t (\omega)$ en el conjunto $D$ es Hölder continua de índice $\alpha$, y con mayor razón, uniformemente continua en D. \\

Definamos un nuevo proceso estocástico para todo $t \in I$, 
	\begin{align*}
	\tilde{X}_t (\omega) := 
		\begin{cases}
		\lim_{s \rightarrow t} X_s (\omega), & \text{ si } \omega \in \tilde{\Omega}, s \in D, \\
		x_0 \in E, & \text{ si } \tilde{\Omega}^c .
		\end{cases}
	\end{align*}
El límite existe gracias a la condición de continuidad uniforme y al criterio de Cauchy. Por definición, el proceso $\tilde{X}$ es Hölder continuo de índice $\alpha$. Basta ver que $\tilde{X}$ es una modificación del proceso $X$. \\

Veamos que para todo $\delta > 0$ se tiene que
	\begin{align*}
	\mathbb{P} \left(d(X_s, X_t) > \delta \right) & \leq \frac{\mathbb{E} [d(X_s, X_t)^p]}{\delta^p} \\
	& \leq \frac{c|t-s|^{1+ \epsilon}}{\delta^p} \xrightarrow{s \rightarrow t} 0.
	\end{align*}
Por lo que, para toda $t \in I$ se tiene que
	\begin{align}
	X_t = \lim_{s \rightarrow t} X_s, \hspace{0.3cm} \text{ en probabilidad. } \label{acc}
	\end{align}
Como $\tilde{X}_t = \lim_{s \rightarrow t} X_s$ casi seguramente, con $s \in D$; por (\ref{acc}) tenemos que $X_t = \lim_{s \rightarrow t} X_s$ en probabilidad, con $s \in D$. Entonces, $X_t = \tilde{X}_t$ casi seguramente.
\end{proof}

\begin{corollary}
Sea $B = (B_t, t \geq 0)$ un movimiento Browniano. El proceso $B$ admite una modificación cuyas trayectorias son localmente hölderianas de índice $\alpha = 1/2 - \epsilon$, para $\epsilon \in (0, 1/2)$. En particular, $B$ admite una modificación continua.
\end{corollary}
\begin{proof}
Recordemos que $B_t - B_s$ posee la misma distribución que $\sqrt{(t - s) N}$ donde N es una variable aleatoria normal estándar. Entonces, para $p > 0$ tenemos,
\begin{align*}
	\mathbb{E} \left[ |B_s - B_t|^p \right] = |t-s|^{p/2} \mathbb{E} \left[ |N(0, 1)|^p \right],
	\end{align*}
Luego, para una $\epsilon \in (0, 1/2)$ tenemos que 
	\begin{align*}
	\mathbb{E} \left[ |B_s - B_t|^p \right] = |t-s|^{p/2} \mathbb{E} \left[ |\tilde{B}_{1}|^p \right] = c(p) |t-s|^{1 + (p/2 -1)},
\end{align*}
Tomando $p > 2$, vemos que para $\alpha < (p/2 - 1)/p$, tomando $\epsilon = p/2 - 1$. Si tomamos a $p$ suficientemente grande, podemos aplicar el Criterio de Continuidad de Kolmogorov para probar nuestro enunciado.
\end{proof}

\section{Propiedad de Markov}
En esta última sección mostraremos la propiedad de Markov para el movimiento Browniano, la cual a grandes rasgos, nos dice que si un movimiento Browniano $B = (B_t, t \in \mathbb{R}_{+})$ comienza en un punto $x$, entonces el movimiento a tiempo $\tilde{B} = (B_{t+s} - B_s, t \in \mathbb{R}_{+})$ tiene la misma distribución como el proceso que comenzó en $s$. \\

Para poder ver este resultado, considere a $\mathcal{F}_t^0$ como la $\sigma$-álgebra generada por $\{ B_s, s \leq t \}$ y sea $\mathcal{F}_t$ la versión completa de $\mathcal{F}_t^0$, es decir, los conjuntos de $\mathcal{F}_t^0$ junto con los subconjuntos de conjuntos de medida cero. Se tiene que para $s < t$, entonces $\mathcal{F}_s \subset \mathcal{F}_t$. \\

\begin{theorem}[Propiedad de Markov]
Sea $s > 0$, el proceso $\tilde{B}_t = (B_{t+s} - B_s, t \geq 0)$ es un movimiento Browniano independiente de $\mathcal{F}_s$.
\end{theorem}
\begin{proof}
Notemos fácilmente que $\tilde{B}_t$ es un movimiento Browniano, pues $B_{t+s}$ y $B_s$ son procesos Gaussianos centrados, por lo que la diferencia entre ellos resulta ser un proceso Gaussiano centrado. Además, $B_{t+s}$ y $B_s$ son procesos continuos, entonces $B_{t+s} - B_s$ es continuo. Por último, veamos que
	\begin{align*}
	COV \left( \tilde{B}_u, \tilde{B}_v \right) & =  \mathbb{E} \left[ (B_{u+s} - B_s) (B_{v+s} - B_s) \right]  \\ 
	& = \mathbb{E} [B_{u+s} \ B_{v+s}] - \mathbb{E} [B_{u+s} \ B_s] - \mathbb{E}[B_s \ B_{v+s}] + \mathbb{E}[B_s \ B_s] \\
	& = (u + s \wedge v + s) - s - s + s  \\
	& = u  \wedge v. 
	\end{align*}
Por lo que $\tilde{B}$ es un movimiento Browniano. \\

Recordemos que si dos vectores (finitos) son independientes, entonces, lo son también para conjuntos de clases bajo intersecciones finitas (Teorema 10.1 de \cite[p.~65]{jacodprotter}) y del Teorema de Clases Monótonas \cite[p.~36]{jacodprotter}, entonces, basta ver que para toda $s_1 \leq s_2 \leq \ldots \leq s_n \leq s$ y $t_1 \leq t_2 \leq \ldots \leq t_m \leq t$, los vectores $(\tilde{B}_{t_1}, \ldots, \tilde{B}_{t_m})$ y $(B_{s_1}, \ldots, B_{s_n})$ son independientes. \\

Como $(\tilde{B}_{t_1}, \ldots, \tilde{B}_{t_m})$ y $(B_{s_1}, \ldots, B_{s_n})$ son vectores Gaussianos, de la Proposición \ref{gaussiano1} tenemos que para toda $i \neq j$
	\begin{align*}
	COV(\tilde{B}_{t_j} B_{s_i}) = \mathbb{E} \left[ \tilde{B}_{t_j} B_{s_i} \right] & = \mathbb{E} \left[ \left( B_{t_j + s} - B_s \right) B_{s_i} \right] \\
	& = \mathbb{E} \left[ B_{t_j + s}B_{s_i} \right] - \mathbb{E} \left[ B_s B_{s_i} \right] \\
	& = s_i - s_i = 0,
	\end{align*}
lo que nos asegura la independencia de los vectores y por el argumento previo, tenemos que $\tilde{B}_t$ es independiente de $\mathcal{F}_s$.
\end{proof}

Otra manera de ver la la propiedad de Markov es la siguiente.

\begin{proposition}
Sea $s > 0$. Al condicionar el proceso $\widehat{B} = (B_{t+s}, t \geq 0)$ con respecto a $\mathcal{F}_s$, resulta ser un movimiento Browniano que empieza en $B_s$.
\begin{proof}
Para toda $t \geq 0$ tenemos que $\widehat{B}_t = \tilde{B}_t + B_s$. Entonces, si $t_1 \leq t_2 \leq \ldots \leq t_n$ y toda función boreliana y acotada $ f : \mathbb{R}^n \rightarrow \mathbb{R}$, se tiene que
	\begin{align*}
	\mathbb{E} \left[ f \left( \widehat{B}_{t_1}, \ldots, \widehat{B}_{t_n} \right) \bigg| \mathcal{F}_s \right] & = \mathbb{E} \left[ f \left( \tilde{B}_{t_1} + B_s, \ldots, \tilde{B}_{t_n} + B_s \right) \bigg| \mathcal{F}_s \right] \\
	& = \mathbb{E} \left[ f \left( \tilde{B}_{t_1} + x, \ldots, \tilde{B}_{t_n} + x \right), \ x = B_s \right] \\
	& = \mathbb{E} \left[ f \left( B_{t_1} + x, \ldots, B_{t_n} + x \right) \right] \hspace{0.3cm} \text{ (prop. de Markov)} \\
	& = \mathbb{E}_{x} \left[ f \left( B_{t_1}, \ldots, B_{t_n} \right) \right],
	\end{align*}
donde $\mathbb{E}_x$ es la ley del movimiento Browniano que empieza en $x$. Por lo que, $\widehat{B}$ es un movimiento Browniano que comienza en el punto $B_s = x$.
\end{proof}
\end{proposition}

\subsection{Ley de Blumenthal}
Presentamos la Ley 0 - 1 de Blumenthal, la cual es una herramienta útil para estudiar ciertas propiedades de las trayectorias del movimiento Browniano, aunque no es el objetivo principal de este capítulo, se darán algunas observaciones. Se dice que una $\sigma$-álgebra  $\mathcal{F}$, es trivial si para toda $A \in \mathcal{F}$, se tiene que $\mathbb{P}(A) = 0$ o $1$.

\begin{theorem}[Ley 0 - 1 de Blumenthal]
\label{blumenthal}
Sea $\mathcal{F}_{t^{+}} = \cap_{u > t} \mathcal{F}_u$, entonces $\mathcal{F}_{0^{+}}$ es trivial.
\end{theorem}
\begin{proof}
Veamos primero que para toda $s \geq 0$, $(B_{t+s} - B_s, t \geq 0)$ es independiente de $\mathcal{F}_{s^{+}}$. Para obtener este resultado, basta ver que para todo evento $A$ en $\mathcal{F}_{s^{+}}$, con $0 \leq t_1 \leq \cdots \leq t_n$ se tiene que
	 \begin{align*}
	 \mathbb{E} \left[ 1_A f \left( \tilde{B}_{t_1}, \ldots, \tilde{B}_{t_n} \right) \right] = \mathbb{P} (A) \mathbb{E} \left[ f \left( B_{t_1}, \ldots, B_{t_n} \right) \right],
	 \end{align*}
donde $f : \mathbb{R}^n \rightarrow \mathbb{R}$ es continua y acotada, además $\tilde{B}_{t_i} = B_{t_i + s} - B_s$. \\

De la propiedad de Markov, observemos que para cualquier $\epsilon > 0$, con $A \in \mathcal{F}_{s^{+}} \subset \mathcal{F}_{s + \epsilon}$ se tiene que
	 \begin{align*}
	 \mathbb{E} \left[ 1_A f \left( \tilde{B}_{t_1}^{(\epsilon)}, \ldots, \tilde{B}_{t_n}^{(\epsilon)} \right) \right] = \mathbb{P} (A) \mathbb{E} \left[ f \left( B_{t_1}, \ldots, B_{t_n} \right) \right],
	 \end{align*}
donde $\tilde{B}_{t_i}^{(\epsilon)} = B_{t_i + s + \epsilon} - B_{s + \epsilon}$. Haciendo tender $\epsilon$ a $0$ y aplicando el Teorema de Convergencia Dominada de Lebesgue en ambos lados de la igualdad, obtenemos la igualdad deseada. \\

Por último, si tomamos $s = 0$, tenemos que $(B_t, t \geq 0)$ es independiente de $\mathcal{F}_{0^{+}}$. Se sigue entonces que $\sigma(B_t, t \geq 0)$ y $\mathcal{F}_{0^{+}}$ son independientes, sin embargo, tenemos que 
	\begin{align*}
	\mathcal{F}_{0^{+}} = \bigcap_{u > 0} \mathcal{F}_u \subset \mathcal{F}_{\infty},
	\end{align*}
donde $\mathcal{F}_{\infty}$ es la versión completa de $\sigma(B_t, t \geq 0)$. De estos hechos podemos deducir que todo conjunto $A \in \mathcal{F}_{0^{+}}$ es independiente de si mismo, es decir, $\mathbb{P} (A) = 0$ ó $1$.
\end{proof}

Una consecuencia de la Ley 0 - 1 de Blumenthal es la siguiente. 

\begin{proposition}
Consideremos un movimiento Browniano $B_t$ y definamos
	\begin{align*}
	\tau_0^{+} & = \inf \{ t \geq 0 : B_t > 0 \}; \\
	\tau_0^{-} & = \inf \{ t \geq 0 : B_t < 0 \}.
	\end{align*}
Entonces, $\tau_0^{+} = 0$ c. s. y $\tau_0^{-} = 0$ c. s. 
\end{proposition}
\begin{proof}
Es fácil ver que, si $B$ es un movimiento Browniano, entonces $-B$ resulta ser un movimiento Browniano. A esta propiedad se le conoce como simetría del movimiento Browniano, lo que nos permite deducir que si $\epsilon > 0$ entonces
	\begin{align*}
	\mathbb{P} (B_t \leq 0) = \mathbb{P} (B_t \geq 0) = 1/2, \hspace{0.3cm} \text{ para toda } t \in [0, \epsilon].
	\end{align*}
Así, para toda $\epsilon > 0$ se cumple que $\mathbb{P} (B_t \geq 0, t \in [0, \epsilon]) \leq 1/2$ y $\mathbb{P}(\tau_0^{-} \leq \epsilon) \geq 1/2$. Haciendo $\epsilon \rightarrow \infty$ tenemos que $\mathbb{P}(\tau_0^{-} = 0) \geq 1/2$. \\

Luego, podemos ver que $\{\tau_0^{-} = 0\} \in \mathcal{F}_{0^{+}}$ ya que 
	\begin{align*}
	\{\tau_0^{-} = 0\} = \bigcap_{n \geq 1} \left\{ 0 < t < \frac{1}{n}: \ B_t < 0 \right\}.
	\end{align*}

Entonces, $\tau_0^{-}$ debe tener probabilidad $0$ ó $1$, pero ya sabemos que $\mathbb{P}(\tau_0^{-} = 0) \geq 1/2$, entonces $\mathbb{P}(\tau_0^{-} = 0) = 1$, donde el mismo argumento aplica para $\tau_0^{+}$. 
\end{proof}

Lo cual nos dice casi seguramente que los tiempos $\{t : B_t = 0\}$ no son acotados, en otras palabras, vemos que el movimiento Browniano oscila alrededor de su punto de origen. \\

Además, el resultado anterior nos dice que casi seguramente, el movimiento Browniano entre inmediatamente a $(0, \infty)$, esta propiedad será de gran utilidad a la hora de analizar una característica peculiar en los problemas de tiempo de paro a tiempo continuo. \\

De acuerdo al último resultado, la trayectoria del movimiento Browniano oscila infinitamente alrededor de su punto de inicio.

\section{Propiedad Fuerte de Markov}
En la última parte de este capítulo veremos la extensión de la propiedad de Markov a tiempo no determinista. Consideremos un movimiento Browniano $B = (B_t, t \geq 0)$ definido en un espacio de probabilidad completo (si todos los subconjuntos de eventos nulos son eventos nulos) $(\Omega, \mathcal{F}, \mathbb{P})$ y las $\sigma$-álgebras como en la sección pasada.

\begin{theorem}[Propiedad Fuerte de Markov] \label{markov-fuerte}
Sea $\tau$ un tiempo de paro. Considere el proceso $(\tilde{B}_t := B_{t + \tau} - B_{\tau}, t \geq 0)$, si éste es condicionado con respecto al evento $\{ \tau < \infty \}$, entonces, $\tilde{B}_t$ es un movimiento Browniano independiente de $\mathcal{F}_{\tau}$. Donde $\mathcal{F}_{\tau}$ es descrita en la Definición \ref{algebraaleatoria}.
\end{theorem}
\begin{proof}
% Supongamos primero que nuestro tiempo de paro $\tau$ es finito c. s. 
Como en la prueba de la Propiedad de Markov, basta probar que para todo evento $A \in \mathcal{F}_{\tau}$, con $0 \leq t_1 \leq \cdots \leq t_n$ y una función $f : \mathbb{R}^n \rightarrow \mathbb{R}_{+}$ continua y acotada, 
	\begin{align}
	\mathbb{E} \left[ 1_A f \left( \tilde{B}_{t_1}, \ldots, \tilde{B}_{t_n} \right) \right] = \mathbb{P} (A) \mathbb{E} \left[ f \left( B_{t_1}, \ldots, B_{t_n} \right) \right], \label{acd}
	\end{align}
después utilizar el Lema de Clases Monótonas (\cite[p.~36]{jacodprotter}) para extender el resultado al proceso completo con respecto a la $\sigma$-álgebra $\mathcal{F}_{\tau}$. \\

Para ver que $\tilde{B}$ es un movimiento Browniano, basta con escoger $A$ como $\Omega$ en (\ref{acd}) y observar que las trayectorias de $\tilde{B}$ son siempre continuas. \\

Finalmente, veamos que se cumple la ecuación anterior. Consideremos números diádicos de tal manera que podamos aproximarnos a nuestro proceso, entonces tenemos que
	\begin{align*}
	\sum_{j = 1}^{\infty} 1_{ \{ \frac{j-1}{2^m} < \tau \leq \frac{j}{2^m} \} } f \left( B_{\frac{j}{2^m} + t_1} - B_{\frac{j}{2^m}}, \cdots, B_{\frac{j}{2^m} + t_n} - B_{\frac{j}{2^m}}  \right) \xrightarrow{m \rightarrow \infty} f \left( \tilde{B}_{t_1}, \ldots, \tilde{B}_{t_n} \right).
	\end{align*}
Del Teorema de Convergencia Monótona, 
	\begin{align}
	\mathbb{E} & \left[1_A f \left( \tilde{B}_{t_1}, \ldots, \tilde{B}_{t_n} \right) \right] \nonumber  \\
	& = \lim_{m \rightarrow \infty} \sum_{j = 1}^{\infty} \mathbb{E} \left[ 1_{ A \cap \{\frac{j-1}{2^m} < \tau \leq \frac{j}{2^m}\}} f \left( B_{\frac{j}{2^m} + t_1} - B_{\frac{j}{2^m}}, \cdots, B_{\frac{j}{2^m} + t_n} - B_{\frac{j}{2^m}}  \right) \right]. \label{ace}
	\end{align}
Para un $A \in \mathcal{F}_{\tau}$ se tiene que el evento $A \cap \{(j-1)/2^m < \tau \leq j/2^m\}$ es $\mathcal{F}_{j/2^m}$-medible, por lo que coincide casi seguramente con un evento de $\sigma(B_s, s \leq j/2^m)$. Al aplicar la propiedad de Markov, obtenemos
	\begin{align}
	\mathbb{E} & \left[ 1_{ A \cap \{\frac{j-1}{2^m} < \tau \leq \frac{j}{2^m}\}} f \left( B_{\frac{j}{2^m} + t_1} - B_{\frac{j}{2^m}}, \cdots, B_{\frac{j}{2^m} + t_n} - B_{\frac{j}{2^m}}  \right) \right] \nonumber \\
	& = \mathbb{E} \left[ 1_A 1_{\{\frac{j-1}{2^m} < \tau \leq \frac{j}{2^m}\}} f \left( B_{\frac{j}{2^m} + t_1} - B_{\frac{j}{2^m}}, \cdots, B_{\frac{j}{2^m} + t_n} - B_{\frac{j}{2^m}}  \right) \right] \nonumber \\
	& = \mathbb{P} \left( A \cap \left\{ \frac{j-1}{2^m} < \tau \leq \frac{j}{2^m} \right\} \right) \mathbb{E} \left[ f \left( B_{t_1}, \cdots, B_{t_n} \right) \right]. \label{acf}
	\end{align}
Por lo que, aplicando (\ref{acf}) en (\ref{ace}) se sigue que,
	\begin{align*}
	\mathbb{E} \left[ 1_A f \left( \tilde{B}_{t_1}, \ldots, \tilde{B}_{t_n} \right) \right] = \mathbb{P} (A) \mathbb{E} \left[ f \left( B_{t_1}, \ldots, B_{t_n} \right) \right].
	\end{align*}
\end{proof}







\chapter{Martingalas Continuas}
En el presente capítulo estudiamos las propiedades y resultados principales de tiempos de paro y martingalas a tiempo continuo. Muchos de los resultados se han sido vistos en el Capítulo 1, sin embargo, es necesario presentar nuevos conceptos para poder analizar los procesos estocásticos a tiempo continuo. Las herramientas que se presentan en este capítulo serán las necesarias para poder resolver el Problema de Paro Óptimo a tiempo continuo. \\

En principio, se presentan nuevos conceptos referentes a los procesos estocásticos a tiempo continuo, para después analizar los tiempos de paro y las martingalas. Al final del capítulo se dan algunas aplicaciones de las propiedades de martingalas para el movimiento Browniano. 

\section{Conceptos fundamentales}

\begin{definition}
	Considere un espacio de probabilidad $(\Omega, \mathcal{F}, \mathbb{P})$. Una filtración a tiempo continuo es una familia de sub-$\sigma$-álgebras $(\mathcal{F}_t)_{t \geq 0}$ cada una perteneciente a $\mathcal{F}$, de tal manera que $\mathcal{F}_s \subset \mathcal{F}_t$ si y solo si $s \leq t$. Al sistema $(\Omega, \mathcal{F}, (\mathcal{F}_t)_{t \geq 0}, \mathbb{P})$ se le conoce como espacio de probabilidad filtrado.
\end{definition}

Recordemos que la filtración canónica asociada al proceso estocástico $X = (X_t, t \geq 0)$ está definida como $\mathcal{F}_t = \sigma(X_s : s \leq t)$. Definamos los siguientes conceptos.

\begin{align*}
	\mathcal{F}_{t^{-}} = \sigma \left( \bigcup_{s < t} \mathcal{F}_s \right), \hspace{0.3cm} \mathcal{F}_{t^{+}} = \bigcap_{s > t} \mathcal{F}_s, \hspace{0.3cm} \text{ para toda } t \geq 0.
\end{align*}

De las definiciones anteriores es claro ver que $\mathcal{F}_{t^{-}} \subset \mathcal{F}_t \subset \mathcal{F}_{t^{+}}$. \\

Como se ha visto hasta el momento, el concepto de medibilidad, resulta claro cuando se fija una tiempo $t$, entonces se dice que $X_t$ es una función $\mathcal{F}$-medible, sin embargo, hemos observado que un proceso estocástico continuo depende de dos variables $(t, \omega)$, por lo que; si el evento $\omega$ es fijo, se contamos con ninguna propiedad que pueda decirnos si las trayectorias del proceso son medibles. Para ello, es conveniente contar con alguna característica de medibilidad conjunta. \\

\begin{definition}
	Un proceso estocástico $X = (X_t, t \geq 0)$ con espacio de estados $(E, \mathcal{E})$ se dice medible, si para toda $A \in \mathcal{E}$, el conjunto
	\begin{align*}
	\{ (t, \omega) : X_t (\omega) \in A \},
	\end{align*}
	Pertenece a la $\sigma$-álgebra producto $\mathcal{B} [0, \infty)] \times \mathcal{F}$, es decir,
	\begin{align*}
	(t, \omega) \rightarrow X_t(\omega) : ([0, \infty) \times \Omega, \ \mathcal{B}(0, \infty) \times \mathcal{F} ) \rightarrow (E, \mathcal{E}).
	\end{align*}
\end{definition}

Del Teorema de Fubuni sabemos que \\

Aquí va el Teorema de Tonell-Fubini. \\

En consecuencia, sabemos que las trayectorias de un proceso medible, son funciones $\mathcal{B}[0, \infty)$-medibles. El hecho de introducir un concepto de filtración continua, nos permitirá definir conceptos más interesantes y útiles que el de proceso medible.

\begin{definition}
	Considere un espacio de probabilidad filtrado  $(\Omega, \mathcal{F}, (\mathcal{F}_t)_{t \geq 0}, \mathbb{P})$ y un proceso estocástico $(X_t, t \geq 0)$ definido en  $(\Omega, \mathcal{F}, \mathbb{P})$. Se dice que el proceso es adaptado a la filtración $(\mathcal{F}_t)_{t \geq 0}$ si para toda $t \geq 0$ la variable aleatoria $X_t$ es $\mathcal{F}_t$-medible.
\end{definition}

La siguiente definición es fundamental para el estudios de los procesos a tiempo continuo.

\begin{definition}[Proceso Progresivo]
	Sea $(\Omega, \mathcal{F}, (\mathcal{F}_t)_{t \geq 0}, \mathbb{P})$ un espacio de probabilidad filtrado y sea $(X_t, t geq 0)$ un proceso estocástico definido en $(\Omega, \mathcal{F}, \mathbb{P})$, con un espacio de estados $(E, \mathcal{E})$. Decimos que el proceso es progresivamente medible, o simplemente, progresivo, si para toda $t \geq 0$
	\begin{align*}
		[0, t] \times \Omega \rightarrow E \\
		(s, \omega) \rightarrow X_s (\omega).
	\end{align*}
	es $\mathcal{B}[0, t] \times \mathcal{F}_t$-medible.
\end{definition}

De las definiciones anteriores podemos afirmar que todo proceso aleatorio es adaptado a su filtración canónica y de la misma manera, decir que un proceso progresivamente medible es siempre adaptado. Sin embargo, en general no se puede suponer cierto que un proceso adaptado es un proceso progresivo, a menos que este sea continuo por la derecha o por la izquierda, como se verá en el siguiente teorema. \\

\begin{theorem}
	Sea $(X = X_t, t \geq 0)$ un proceso estocástico con valores en un espacio métrico $(E, \mathcal{E})$, adaptado a la filtración $(\mathcal{F}_t)_{t \geq 0}$ y continuo por la derecha (o por la izquierda), entonces $X$ es progresivo.
\end{theorem}
\begin{proof}
	Demostraremos el caso en que $X$ es continuo por la derecha, pues cuando es continuo por la izquierda, se realiza una prueba similiar. Para verificar que un proceso es progresivo basta ver que para cada conjunto $A \in \mathcal{B}(E)$
	\begin{align}
	\{ (s, \omega) \in [0, t] \times \Omega : X_s(\omega) \in A \} \in \mathcal{B}[0, t] \times \mathcal{F}_t, \label{progre}
	\end{align}
	Si definimos al proceso continuo en función de un proceso discreto podremos verificar la condición deseada y por la continuidad de $X$, lograremos probar la afirmación. Consideremos entonces para cada $n \geq 1$, $k \in \{ 0, 1, \ldots, 2^n - 1 \}$ y $s \in [0, t]$ el siguiente proceso
	\begin{align*}
	X_s^n (\omega) = 
	\begin{cases}
	X_{\frac{(k+1)t}{2^n}} (\omega), & \text{ si } \frac{kt}{2^n} < s \leq \frac{(k+1)t}{2^n}; \\
	X_0 (\omega), & \text{ si } s = 0.
	\end{cases}
	\end{align*}
	De la continuidad de $X$ sabemos que $\lim_{n \rightarrow \infty} X_s^n (\omega) = X_s (\omega)$ para toda $(s, \omega) \in [0, t] \times \Omega$. \\
	
	Por último, veamos que (\ref{progre}) se cumple para $X_s^n$, al conjunto
	\begin{align*}
	\{ (s, \omega) \in [0, t] \times \Omega : X_s(\omega) \in A \},
	\end{align*}
	lo podemos escribir de tal manera que cada uno de sus elementos pertenezca a $\mathcal{B}[0, t] \times \mathcal{F}_t$, por ejemplo
	\begin{align*}
	\bigcup_{k = 0}^{2^n - 1} \left\{ \left( \frac{kt}{2^n}, \frac{(k+1)t}{2^n} \right] \times \left\{ X_{\frac{(k+1)t}{2^n} \in A} \right\}  \right\} \bigcup\left\{ \{0\} \times \{X_0(\omega) \in A\} \right\}.
	\end{align*}
	Del argumento anterior podemos afirmar que $X$ es un proceso progresivo.
\end{proof}

El siguiente concepto será la base para la definición de una $\sigma$-álgebra ligada a los proceso progresivamente medibles, además de ser un concepto útil en el trabajo posterior.

\begin{definition}
	Se dice que un conjunto $A \subset \mathbb{R}_{+} \times \Omega$ es progresivamente medible si el proceso asociado a $A$
	\begin{align*}
	X_t(\omega) = 1_A (t, \omega) =
	\begin{cases}
	1, & \text{ si } (t, \omega) \in A; \\
	0, & \text{ si } (t, \omega) \notin A
	\end{cases}
	\end{align*}
	es progresivamente medible.
\end{definition}

\begin{proposition}
	La familia de conjuntos progresivamente medibles forma un $\sigma$-álgebra conocida como $\sigma$-álgebra progresiva.
\end{proposition}
\begin{proof}
	Aquí va la demostración que no he terminado.
\end{proof}

La siguiente proposición muestra la relación entre la $\sigma$-álgebra progresiva y los procesos progresivamente medibles.

\begin{proposition}
	Sea $(X_t, t \geq 0)$ un proceso estocástico definido en $(\Omega, \mathcal{F}, \mathbb{P})$, con un espacio de estados $(E, \mathcal{E})$. El proceso $X$ es progresivamente medible si y solo si $(t, \omega) \rightarrow X_t(\omega)$ es medible con respecto a la $\sigma$-álgebra progresiva.
\end{proposition}
\begin{proof}
	Aquí va otra demostración que no he entendido gracias a la medibilidad con respecto a la $\sigma$-álgebra progresiva.
\end{proof}

Antes de entrar de lleno al estudio de los tiempos de paro a tiempo continuo, consideremos las siguientes definiciones.
\begin{definition}
	Una filtración $(\mathcal{F}_t)_{t \geq 0}$ se llama continua por la derecha si $\mathcal{F}_t = \mathcal{F}_{t^{+}} = \cap_{s > t} \mathcal{F}_s$, para toda $t \geq 0$.
\end{definition}

Consideremos a la filtración $(\mathcal{F}_{t^{+}})_{t \geq 0}$ y veamos que es continua por la derecha. Para facilitar el argumento, renombremos a $\mathcal{F}_{t_{+}}$ por $\mathcal{G}_t$, por lo que tenemos, por definición que $\mathcal{G}_t \subseteq \mathcal{G}_{t^{+}}$, más aún
	\begin{align*}
	\mathcal{G}_{t^{+}} = \bigcap_{s < t} \mathcal{G}_s = \bigcap_{s < t} \mathcal{F}_{s^{+}} = \bigcap_{s < t}  \left( \bigcap_{u < s} \mathcal{F}_u \right) \subseteq \bigcap_{s < t} \mathcal{F}_s = \mathcal{F}_{t^{+}} = \mathcal{G}_t
	\end{align*}
Por lo que $\mathcal{G}_t  = \mathcal{G}_{t^{+}}$, lo que quiere decir, que la filtración $(\mathcal{F}_{t_{+}})_{t \geq 0}$ es continua por la derecha. \\

\begin{definition}
	Sea $(\mathcal{F}_t )_{t \geq 0}$ una filtración. Si $\mathcal{F}_0$ contiene a todos los conjuntos $\mathbb{P}$-nulos entonces decimos que es una filtración completa.
\end{definition}

\begin{definition}
	Decimos que la filtración $(\mathcal{F}_t )_{t \geq 0}$ Cumple con las condiciones habituales si es continua por la derecha y es completa.
\end{definition}

\section{Tiempos de paro a tiempo continuo}
En esta sección introduciremos la noción de un tiempo de paro, pero esta vez a tiempo continuo.Las propiedades de los tiempos de paro discretos son similares a las que a continuación veremos, pero aún así es conveniente realizar definiciones detalladas de los conceptos.

\begin{definition}
	En un espacio de probabilidad filtrado $(\Omega, \mathcal{F}, (\mathcal{F}_t)_{t \geq 0}, \mathbb{P})$, una función $\tau : \Omega \rightarrow [0, \infty]$ se llama tiempo de paro con respecto a la filtración $(\mathcal{F}_t)_{t \geq 0}$ si
	\begin{enumerate}
		\item $\tau$ es $\mathcal{F}_{\infty}$-medible.
		\item El conjunto $\{ \tau \leq t \}$ pertenece a $\mathcal{F}_t$, para toda $t \geq 0$.
	\end{enumerate}
\end{definition}

Resulta claro ver que, si $\tau$ es una función constante, entonces es un tiempo de paro. Si consideramos a $k$, una constante positiva, y además $t \in [0, k)$, el conjunto $\{\tau + k \leq t\} = \emptyset$ el cual esta en $\mathcal{F}_t$, por otro lado, si $t \in [k, \infty)$ entonces
\begin{align*}
\{ \tau + k \leq t \} = \{ \tau \leq t - k \} \in \mathcal{F}_{t-k} \subseteq \mathcal{F}_{t}.
\end{align*}
Lo cual implica que $\tau + k$ es un tiempo de paro.

\begin{proposition}
	\label{paroequivalente}
	Si la filtración $(\mathcal{F}_t)_{t \geq 0}$ es continua por la derecha, entonces $\tau$ es un tiempo de paro si y solo si $\{\tau < t\}$ pertenece a $\mathcal{F}_t$, para toda $t \geq 0$.
\end{proposition}
\begin{proof}
	Suponiendo que $\tau$ es un tiempo de paro. para un tiempo fijo $s < t$ se tiene
	\begin{align*}
	\{ \tau \leq s \} \in \mathcal{F}_s,
	\end{align*}
	Pero como, $\mathcal{F}_t$ es filtración, entonces para toda $t \geq 0$ se tiene que $\mathcal{F}_s \subset \mathcal{F}_t$, por lo tanto, $\{ \tau \leq s \} \in \mathcal{F}_t$. Entonces 
	\begin{align*}
	\{ \tau < t \} = \bigcup_{s < t} \{ \tau \leq s \} \in \mathcal{F}_t, \hspace{0.3cm} \text{ para } t \geq 0.
	\end{align*}
	
	Por otra parte, si suponemos que $\{ \tau < t \} \in \mathcal{F}_t$, para toda $t \geq 0$, podemos ver que 
	\begin{align*}
	\{\tau \leq t \} = \bigcap_{s > t} \{\tau > s\} \in \mathcal{F}_{t^{+}}.
	\end{align*}
	Como $(\mathcal{F}_t)_{t \geq 0}$ es un filtración continua por la derecha sabemos que $\mathcal{F}_{t^{+}} = \mathcal{F}_t$. Se tiene entonces que $\tau$ es un tiempo de paro.
\end{proof}

La siguiente definición nos será útil al analizar cuando un proceso aleatorio entra por primera vez en un conjunto. 
\begin{definition}
	Se define al tiempo de entrada del proceso $(X_t, t \geq 0)$ al conjunto $A$, como
	\begin{align*}
	\tau_A (\omega) = 
	\begin{cases}
	\inf \{ t \geq 0 : X_t (\omega) \in A \}, & \text{ si } \{ t \geq 0 : X_t (\omega) \in A \} \neq \emptyset, \\
	\infty, & \text{ si } \{ t \geq 0 : X_t (\omega) \in A \} = \emptyset.
	\end{cases}
	\end{align*}
\end{definition}

Veamos ahora que, bajo ciertas condiciones dadas a la filtración, al proceso y al conjunto, el tiempo de entrada es un tiempo de paro.
\begin{proposition}
	Sea $X = (X_t, t \geq 0)$ un proceso estocástico adaptado a la filtración $(\mathcal{F}_t)_{t \geq 0}$, y con un espacio de estados $(E, \mathcal{B}(E))$, donde $E$ es un espacio métrico y sea $A \in\mathcal{B}(E)$.
	\begin{enumerate}
		\item Si $X$ es continuo por la derecha, $(\mathcal{F}_t)_{t \geq 0}$ es continua por la derecha y $A$ es un conjunto abierto, entonces $\tau_A$ es un tiempo de paro con respecto a $(\mathcal{F}_t)_{t \geq 0}$.
		\item Si $X$ es continuo y $A$ es un conjunto cerrado, entonces $\tau_A$ es un tiempo de paro con respecto a $(\mathcal{F}_t)_{t \geq 0}$.
	\end{enumerate}
\end{proposition}
\begin{proof}
	Consideremos los elementos $\omega \in \Omega_0$ tales que, la trayectoria del proceso $X$ es continua por la derecha, con $\mathbb{P}(\Omega-0) = 1$. Sea
	\begin{align*}
	C_t : = \bigcup_{s \in \mathbb{Q}_{+}} \left\lbrace X_s \in A : s < t \right\rbrace \hspace{0.3cm} \text{ para } t \geq 0 \text{ fija}, 
	\end{align*}
	donde $A$ s un conjunto abierto de $\mathcal{B}(E)$. Como $X$ es adaptado a $(\mathcal{F}_t)_{t \geq 0}$ tenemos que $X_t$ es $\mathcal{F}_t$-medible para toda $t \geq 0$, es decir, para cualquier $A \in \mathcal{B}(E)$
	\begin{align*}
	\{ \omega \in \Omega \mid X_t (\omega) \in A \} \in \mathcal{F}_t,
	\end{align*}
	por lo que $C_t \in \mathcal{F}_t$. Ya que la filtración es continua por la derecha y de la Proposición \ref{paroequivalente}, basta demostrar que $\{\tau_A < t\} = C_t$ para probar que $\tau_A$ es un tiempo de paro. \\
	
	Veamos que se cumplen las siguientes condiciones. Por un lado, si $\omega \in C_t$, de la definición misma de $C_t$ sabemos que existe $s \in \mathbb{Q}_{+}$ con $s < t$, tal que $X_s(\omega) \in A$. Por lo que, $\tau_A = \inf \{ t \geq 0 : X_t (\omega) \in A \} \leq s < t$, es decir, $\omega in \{\tau_A < t\}$. Por otro lado, si $\omega \in \{\tau_A < t\}$, entonces existe un tiempo $t_0 < t$ tal que $X_{t_0} (\omega) \in A$. 
	
	Como el proceso es continuo por la derecha y $A$ es abierto, existe $\epsilon > 0$ tal que
	\begin{align*}
	s \in \mathbb{Q}\cap [t_0, t_0 + \epsilon) \text{ y } X_s(\omega) \in A.
	\end{align*}
	De ambos argumentos, tenemos que 
	\begin{align*}
	\{ \tau_A < t \} = C_t \in \mathcal{F}_t.
	\end{align*}
	Lo que significa que, $\tau_A$ es un tiempo de paro respecto a la filtración $(\mathcal{F}_t)_{t \geq 0}$.
\end{proof}

\begin{definition}
	Sea $\tau$ un tiempo de paro con respecto a la filtración $(\mathcal{F}_t)_{t \geq 0}$. La $\sigma$-álgebra de eventos anteriores a $\tau$ está dada por 
	\begin{align*}
	\mathcal{F}_{\tau} = \left\{ A \in \mathcal{F}_{\infty} : \forall t \in \mathbb{R}_{+}, \ A \cap \{\tau \leq t\} \in \mathcal{F}_t \right\}.
	\end{align*}
\end{definition}
La prueba de que efectivamente, $\mathcal{F}_{\tau}$ es una $\sigma$-álgebra es análoga a la Proposición \ref{algebraaleatoria} del Capítulo 1. Veamos el siguiente resultado, relacionado a la $\sigma$-álgebra parada definida arriba.
\begin{proposition}
	Considere las siguientes definiciones
	\begin{align*}
		\mathcal{F}_{\tau^{+}} & = \left\{ A \in \mathcal{F}_{\infty} : \forall t \in \mathbb{R}_{+}, \ A \cap \{\tau < t\} \in \mathcal{F}_t \right\}, \\
		\mathcal{F}_{\tau^{-}} & = \sigma \left\{ A \cap \{\tau > t\} : t \geq 0, \ A \in \mathcal{F}_t \right\}.
	\end{align*}
	Entonces, 
	\begin{enumerate}
		\item $\mathcal{F}_{\tau^{+}}$ es una $\sigma$-álgebra.
		\item $\mathcal{F}_{\tau^{-}} \subset \mathcal{F}_{\tau} \subset \mathcal{F}_{\tau^{+}}$.
		\item Si $(\mathcal{F}_t)_{t \geq 0}$ es continua por la derecha, entonces $\mathcal{F}_{\tau} = \mathcal{F}_{\tau^{+}}$.
		\item Si $\tau = t$, entonces $\mathcal{F}_{\tau} = \mathcal{F}_t$ y $\mathcal{F}_{\tau^{+}} = \mathcal{F}_{t^{+}}$.
	\end{enumerate}
	\end{proposition}
\begin{proof}
1. Consideremos el conjunto $\Omega$ como la familia de conjuntos que pertenecen a $\mathcal{F}_{\tau^{+}}$. 

% Para ver que $\Omega$ es un elemento de $\mathcal{F}_{t^{+}}$ probemos que $\Omega \in \mathcal{F}_{\infty}$ y además, para todo $t \geq 0$ el conjunto $\Omega \cap \{ \tau < t \}$ pertenece a $\mathcal{F}_t$. \\

Como $\Omega$ está formado por conjuntos de $\mathcal{F}_{t^{+}}$, sabemos que cada uno de ellos pertenece a $\mathcal{F}_{\infty}$, por lo que $\Omega \in \mathcal{F}_{\infty}$. Además sabemos que cada uno de los conjuntos $A$ de $\Omega$ cumple para toda $t \geq 0$ que $A \cap \{\tau < t\} \in \mathcal{F}_t$, por lo que $\Omega \cap \{\tau < t\} \in \mathcal{F}_t$, de ambas condiciones, tenemos que $\Omega \in \mathcal{F}_{\tau^{+}}$. \\

Si $A \in \mathcal{F}_{\tau^{+}}$, tenemos que comprobar que el complemento de $A$ es un elemento de $\mathcal{F}_{\tau^{+}}$. Sabemos que $A \in \mathcal{F}_{\infty}$, entonces $A^{c} \in \mathcal{F}_{\infty}$ al ser una $\sigma$-álgebra. Además, tenemos que $A \cap \{ \tau < t \} \in \mathcal{F}_t$ para toda $t \geq 0$, es decir, $A \in \mathcal{F}_t$, por lo que $A^{c} \in \mathcal{F}_t$ al ser $\sigma$-álgebra. Entonces $A^{c} \cap \{\tau < t\} \in \mathcal{F}_t$. De ambas condiciones, sabemos que $A^{c} \in \mathcal{F}_{\tau^{+}}$. \\

Si $A_1, A_2, \ldots \in \mathcal{F}_{\tau^{+}}$, veamos que $\cup A_n \in \mathcal{F}_{\tau^{+}}$. Sabemos que si, $A_1, A_2, \ldots \in \mathcal{F}_{\tau^{+}}$ entonces cada $A_n \in \mathcal{F}_{\infty}$ para $n \geq 1$, por lo tanto $\cup A_n \in \mathcal{F}_{\infty}$.

Si $A_1, A_2, \ldots \in \mathcal{F}_{\tau^{+}}$, tenemos que para toda $n \geq 1$ y $t \geq 0$ se tiene que 
\begin{align*}
	A_1 \cap \{\tau < t\} \in \mathcal{F}_t, \hspace{0.3cm} A_2 \cap \{\tau < t\} \in \mathcal{F}_t, \ldots
\end{align*}
Entonces, para toda $t \geq 0$
\begin{align*}
	\left( \bigcup_{n \geq 1} A_n \right) \cap \{\tau < t\} = \bigcup_{n \geq 1} \left( A_n \cap \{\tau < t\} \right) \in \mathcal{F}_t.
\end{align*}
Se tiene entonces que $\cup A_n \in \mathcal{F}_{\tau^{+}}$. \\

2. Falta la parte en que pruebo que $\mathcal{F}_{\tau^{-}} \subset \mathcal{F}_{\tau}$. \\

Si $A \in \mathcal{F}_{\tau}$, entonces $A \in \mathcal{F}_{\infty}$, además de que, para toda $t \geq 0$ se tiene $A \cap \{ \tau \leq t \} \in \mathcal{F}_t$, en particular, para todo $t \geq 0$ el conjunto $A \cap \{ \tau < t \}$ pertenece a $\mathcal{F}_t$, es decir, $A \in \mathcal{F}_{\tau^{+}}$, por lo tanto, $\mathcal{F}_{\tau} \subset \mathcal{F}_{\tau^{+}}$. \\

3. De la parte anterior sabemos que $\mathcal{F}_{\tau} \subset \mathcal{F}_{\tau^{+}}$. Por lo que, suponiendo que $(\mathcal{F}_t)_{t \geq 0}$ es continua por la derecha basta demostrar que $\mathcal{F}_{\tau^{+}} \subset \mathcal{F}_{\tau}$ para comprobar que $\mathcal{F}_{\tau} = \mathcal{F}_{\tau^{+}}$. \\

4. Falta la cuarta demostración.
\end{proof}

\begin{proposition}
	Sea $\tau$ un tiempo de paro con respecto a la filtración $(\mathcal{F}_t)_{t \geq 0}$. Si consideramos un conjunto $A$ que pertenece a $\mathcal{F}{\infty}$ y definimos $\gamma = \tau 1_A + \infty 1_{A^{c}}$. Entonces, $\gamma$ es un tiempo de paro con respecto a la filtración $(\mathcal{F}_t)_{t \geq 0}$.
\end{proposition}

A continuación, se muestran pequeños resultados que nos daran útiles herramientas a la hora de trabajar con respecto a los tiempos de paro en afirmaciones posteriores.
\begin{lemma}
	\label{paromaxmin}
	Sean $\theta$ y $\tau$ dos tiempos de paro con respecto a la filtración $(\mathcal{F}_t)_{t \geq 0}$, entonces $\theta \wedge \tau$ y $\theta \vee \tau$ son tiempos de paro con respecto a la filtración $(\mathcal{F}_t)_{t \geq 0}$.
\end{lemma}
\begin{proof}
	Como $\theta$ y $\tau$ dos tiempos de paro entonces $\{\tau \leq t \} \in \mathcal{F}_t$ y $\{\theta \leq t \} \in \mathcal{F}_t$, podemos entonces ver al conjunto
	\begin{align*}
		\{\theta \wedge \tau > t \} & = \{ \theta > t \} \cap \{ \tau > t \} \\
		& = \{ \theta \leq t \}^{c} \cap \{ \tau \leq t \}^{c} \in \mathcal{F}_t.
	\end{align*}
	Por lo que $\{\theta \wedge \tau \leq t \} = \{\theta \wedge \tau > t \}^{c}$ es un conjunto de $\mathcal{F}_t$, es decir, $\theta \wedge \tau$ es un tiempo de paro respecto a la filtración $(\mathcal{F}_t)_{t \geq 0}$. \\

	Para mostrar la segunda afirmación, tenemos que
	\begin{align*}
		\{ \theta \vee \tau \leq t \} = \{ \theta \leq t \} \cap \{ \tau \leq t \} \in \mathcal{F}_t,
	\end{align*}
	lo cual muestra que $\theta \vee \tau$ es un tiempo de paro respecto a la filtración $(\mathcal{F}_t)_{t \geq 0}$.
\end{proof}

\begin{lemma}
	Sea $(\tau_n)_{n \geq 1}$ una sucesión de tiempos de paro con respecto a la filtración $(\mathcal{F}_t)_{t \geq 0}$, entonces $\sup_{n \geq 1} \tau_n$ es tiempo de paro con respecto a la misma filtración. Si además suponemos que la filtración es continua por la derecha entonces
	\begin{align*}
		\inf_{n \geq 1} \tau_n, \hspace{0.3cm} \liminf_{n \rightarrow \infty} \tau_n, \hspace{0.3cm} \limsup_{n \rightarrow \infty} \tau_n,
	\end{align*}
	son tiempos de paro.
\end{lemma}

\begin{proof}
	Basta mostrar que el $\sup_{n \geq 1} \tau_n$ e $\inf{n \geq 1} \tau_n$ son tiempos de paro, pues recordemos que de la definición de $\liminf$ y $\limsup$ tenemos que 
	\begin{align*}
		\limsup_{n \rightarrow \infty} \tau_n = \inf_{n \geq 1} \left\{ \sup_{m \geq n} \tau_m \right\} \text{   y   } \liminf_{n \rightarrow \infty}\tau_n = \sup_{n \geq 1} \left\{ \inf_{m \geq n} \tau_m \right\}.
	\end{align*}.
	Entonces, veamos que para demostrar que $\sup_{n \geq 1} \tau_n$ es un tiempo de paro con respecto a $(\mathcal{F}_t)_{t \geq 0}$, tenemos que comprobar que el conjunto $\{ \sup_{n \geq 1} \tau_n \leq t\}$ es un elemento de $\mathcal{F}_t$. Sin embargo, podemos ver que
	\begin{align*}
		\left\{ \sup_{n \geq 1} \tau_n \leq t \right\} = \bigcap_{n \geq 1} \{\tau_n \geq t\},
	\end{align*}
	donde cada elemento de la intersección pertenece a la filtración a tiempo $t$. De la misma manera podemos observar que
	\begin{align*}
		\left\{ \inf_{n \geq 1} \tau_n \leq t \right\} = \bigcup_{n \geq 1} \{\tau_n < t\}.
	\end{align*}
\end{proof}

\begin{lemma}
	\label{sigmaparadasubset}
	Sean $\tau$ y $\theta$ dos tiempos de paro con respecto a la filtración $(\mathcal{F}_t)_{t \geq 0}$, y un conjunto $A \in \mathcal{F}_{\theta}$ entonces $A \cap \{\theta \leq \tau\} \in \mathcal{F}_{\tau} \cap \mathcal{F}_{\theta}$. En particular, si $\theta \leq \tau$ entonces $\mathcal{F}_{\theta} \subset \mathcal{F}_{\tau}$.
\end{lemma}
\begin{proof}
	En principio, para comprobar que $A \cap \{\theta \leq \tau\}$ es un elemento de $\mathcal{F}_{\tau}$, recordemos que un conjunto $B$ pertenece a la $\sigma$-álgebra mencionada si y solo si
	\begin{align*}
		B \in \mathcal{F}_{\infty} \text{  y para todo  } t \geq 0 \ B \cap \{\tau \leq t\} \in \mathcal{F}_t.
	\end{align*}
	Sabemos que $A$ y $\{\theta \leq t\}$ son elementos de $\mathcal{F}_{\infty}$. Ahora, consideremos el conjunto
	\begin{align*}
		A \cap \{ \theta \leq \tau\} \cap \{\tau \leq t\} = A \cap \{\theta \leq t\} \cap \{\theta \wedge t \leq \tau \wedge t\} \cap \{\tau \leq t\}.
	\end{align*}
	Sabemos que $A \cap \{\theta \leq t\} \in \mathcal{F}_t$, pues $A \in \mathcal{F}_{\theta}$, como $\tau$ es un tiempo de paro, $\{\tau \leq t\} \in \mathcal{F}_t$, por último, $\theta \wedge t$ y $\tau \wedge t$ son variables aleatorias $\mathcal{F}_t$-medibles, por lo tanto
	\begin{align*}
		A \cap \{ \theta \leq \tau\} \cap \{\tau \leq t\} \in \mathcal{F}_t.
	\end{align*}
	Por lo que, $A \cap \{\theta \leq \tau\} \in \mathcal{F}_{\tau}$. \\

	Por otro lado, podemos ver que se puede descomponer el conjunto
	\begin{align}
		A \cap \{\theta \leq \tau\} \cap \{\theta \leq t\} & = \left( A \cap \{\theta \leq \tau\} \cap \{\tau \leq t\} \cap \{\theta \leq t\} \right) \nonumber \\
		& \cup \left( A \cap \{\theta \leq \tau\} \cap \{\tau > t\} \cap \{\theta \leq t\} \right), \label{acg}
	\end{align}
Pues ambos son disjuntos, ahora veamos que ambos son elementos de $\mathcal{F}_t$. De la primera parte de la demostración sabemos que el primer elemento de la unión en (\ref{acg}) pertenece a $\mathcal{F}_t$. Para la segunda parte en (\ref{acg}) vemos que
	\begin{align*}
		A \cap \{\theta \leq \tau\} \cap \{\tau > t\} \cap \{\theta \leq t\} = A \cap \{\tau \leq t\} \cap \{\theta \leq t\},
	\end{align*}
	donde cada elemento pertenece a $\mathcal{F}_t$, lo cual prueba que $A \cap \{\theta \leq \tau\} \in \mathcal{F}_{\theta}$. En particular, con $\theta \leq \tau$ tenemos que si $A \in \mathcal{F}_{\theta}$ entonces
	\begin{align*}
		A \cap \{ \tau \leq t \} = A \cap \{ \theta \leq \tau \} \cap \{ \tau \leq t \} \in \mathcal{F}_t,
	\end{align*}
	lo cual resulta cierto de los resultados anteriores.
\end{proof}

\begin{lemma}
	\label{nosequeponer}
	Sea $\tau$ un tiempo de paro con respecto a $(\mathcal{F}_t)_{t \geq 0}$ y $\theta$ una función que es $\mathcal{F}_{\tau}$-medible, de tal manera que $\theta \geq \tau$, entonces $\theta$ es un tiempo de paro con respecto a la filtración $\mathcal{F}_t)_{t \geq 0}$.
\end{lemma}
\begin{proof}
	Consideremos el conjunto $\{ \theta \leq t \}$, como $\theta \geq t$ tenemos que
	\begin{align*}
		\{\theta \leq t\} = \{\theta \leq t\} \cap \{\tau \leq t\},
	\end{align*}
	pero como $\theta$ es una función $\mathcal{F}_{\tau}$-medible sabemos que todo conjunto $B$ en $\mathcal{F}_{\tau}$ cumple $B \cap \{\tau \leq t\} \in \mathcal{F}_t$, lo cual muestra que $\theta$ es un tiempo de paro.
\end{proof}

\begin{lemma}
	Sean $\tau$ y $\theta$ dos dos tiempos de paro con respecto a la filtración $(\mathcal{F}_t)_{t \geq 0}$, entonces la suma de ambos es un tiempo de paro con respecto a la misma filtración.
\end{lemma}
\begin{proof}
	Recordemos del Lema \ref{paromaxmin} que si $\tau$ y $\theta$ son tiempos de paro respecto a la misma filtración entonces $\tau \vee \theta$ también lo es. Tenemos que $\tau \vee \theta \leq \tau + \theta$, con al ayuda del Lema \ref{sigmaparadasubset} podemos ver fácilmente que $\mathcal{F}_{\tau}, \mathcal{F}_{\theta} \subset \mathcal{F}_{\tau \vee \theta}$. 

	Por lo tanto, al ser $\tau$ una variable $\mathcal{F}_{\tau}$-medible y $\theta$ una variable $\mathcal{F}_{\theta}$-medible concluimos que $\tau + \theta \in \mathcal{F}_{\tau \vee \theta}$. Del Lema \ref{nosequeponer} concluimos que $\tau + \theta$ es un tiempo de paro.
\end{proof}

\begin{lemma}
	Considere un tiempo de paro $\tau$ con respecto a la filtración $(\mathcal{F}_t)_{t \geq 0}$ entonces existe una sucesión decreciente $(\tau_n)_{n \geq 1}$ de tiempo de paro discretos tal que $\lim_{n \rightarrow \infty} \tau_n = \tau$.
\end{lemma}
\begin{proof}
	
\end{proof}

\section{Martingalas a tiempo continuo}
\section{Aplicaciones al Movimiento Browniano}






\chapter{Problema de Paro Óptimo: Tiempo Continuo}
Para finalizar este trabajo, analizamos el caso a tiempo continuo de los problemas de paro óptimo. Consideremos los resultados vistos hasta ahora y tomemos el caso de una cadena de Markov. Recordemos del Capítulo 2, la definición del tiempo de paro candidato $\tau_P$ como
\begin{align*}
	\tau_P = \inf \{ n \geq 0 : X_n \in P \},
\end{align*}
donde $P$ es la región de paro, es decir, $P = \{ x \in E : V(x) = G(x) \}$. De la definición sabemos que se requiere cierto conocimiento \emph{a priori} de la función $V(x)$, la cual está definida por
\begin{align*}
	V(x) = \mathbb{E}_x [G(X_{\tau_P})].
\end{align*}
En otras palabras, debemos tener un conocimiento de $\tau_P$ para poder conocer el valor de $V(x)$. De ambas dependencias, resulta difícil dar una solución sistemática al problema de paro óptimo en este caso. Por lo tanto, se recurre a una técnica con un enfoque que requiere cierta intuición para proponer una solución a este tipo de problemas y después poder verificar que, efectivamente, la propuesta sea óptima. \\

Para ejemplificar este tipo de técnicas, consideremos problemas de paro óptimo de la siguiente forma
\begin{align}
	v(x) = \sup_{\tau \in \mathcal{T}} \mathbb{E}_x [e^{-q \tau} G(Y_{\tau})], \label{problema_paro}
\end{align}
donde $Y = \{ Y_t : t \geq 0 \}$, esta definido como
\begin{align}
	Y_t = at + \sigma B_t + \sum_{i = 1}^{N_t} X_i, \hspace{0.3cm} \text{para } t \geq 0, \label{levy_salto}
\end{align}
donde $(X_i)$ son variables aleatorias negativas independientes e identicamente distribuidas, $N_t$ es un proceso Poisson con parámetro $\lambda$, $B_t$ un movimiento Browniano, $\sigma \geq 0$ y $a \in \mathbb{R}$. Además, $G$ es una función no negativa medible, $q \geq 0$ y $\mathcal{T}$ es la familia de tiempos de paro con respecto a la filtración $\mathbb{F} = \{ \mathcal{F}_t : t \geq 0 \}$ donde $\mathcal{F}_t = \sigma\{ Y_u : u \leq t \}$. \\

El proceso $Y$ definido anteriormente pertenece a una familia de procesos estocásticos la cual es bastante importante para las aplicaciones, gracias a que se pueden calcular muchos de sus funcionales. Esta familia es conocida como procesos de Lévy.

\begin{definition}[Proceso de Lévy] 
  Un proceso $X = \{ X_t : t \geq 0 \}$ se dice que es un proceso de Lévy si tiene las siguientes propiedades
  \begin{enumerate}
  	\item Las trayectorias de $X$ son casi seguramente continuas por la derecha con límites por la izquierda.
    \item $X_0 = 0$ casi seguramente.
    \item Para los tiempos $0 \leq s \leq t$, los incrementos $X_t - X_s$ son estacionarios, es decir, $X_t - X_s$ tiene la misma distribución que $X_{t-s}$.
    \item Para los tiempos $0 \leq s \leq t$, los incrementos $X_t - X_s$ son independientes de $\sigma \{ X_u : u \leq s \}$.
  \end{enumerate}
\end{definition}

Vamos a denotar por $\mathbb{P}_y$ como ley del proceso $Y$ empezando en $y$, es decir, $\mathbb{P}(Y_0 = y) = 1$. \\

Podemos ver de (\ref{levy_salto}), que $Y_0 = 0$ c.s. De la misma manera, sabemos que el proceso es continuo con saltos negativos de intensidad $X_i$, por lo que el proceso es continuo por la derecha con límites por la izquierda. Los incrementos que presenta el proceso son de la forma
\begin{align}
	\tilde{Y}_t = Y_{t+s} - Y_s & = \left( a(t+s) + \sigma B_{t+s} + \sum_{i = 1}^{N_{t+s}} X_i \right) - \left( as + \sigma B_s + \sum_{i = 1}^{N_s} X_i \right) \nonumber \\
    & = at + \sigma (B_{t+s} - B_{s}) + \left( \sum_{i = 1}^{N_{t+s}} X_i - \sum_{i = 1}^{N_s} X_i \right) \nonumber \\
    & = at + \sigma (B_{t+s} - B_{s}) + \sum_{i = N_s}^{N_{t+s}} X_i  \nonumber \\
    & = at + \sigma (B_{t+s} - B_{s}) + \sum_{i = 0}^{N_{t+s} - N_s} X_{N_s + i}. \label{poisson}
\end{align}
Como $B_{t+s} - B_s$ y $N_{t+s} - N_s$ tienen la misma distribución que $B_t$ y $N_t$, para $t \in \mathbb{R}_{+}$, observamos que (\ref{poisson}) tiene la misma distribución que $Y_t$. Además como $B$ y $N$ tienen incrementos independientes y $X_i$ son independiente identicamente distribuidas, $Y_t - Y_s$ es independiente de $\mathcal{F}_s$. 

De los argumentos anteriores, tenemos que el proceso $Y$ es un proceso de Lévy. \\

De las propiedades de estacionariedad e independencia que presenta el proceso $Y$ en sus incrementos, podemos verificar que posee la propiedad de Markov. \\

Recordemos que la propiedad de Markov expresada en términos de $Y$ establece que para cualquier tiempo $s > 0$, el proceso $\widehat{Y}_t = (Y_{t+s}, \ t \geq 0)$ condicionado bajo $\mathcal{F}_s$, resulta ser un proceso de Lévy que comienza en $Y_s$.

\begin{proposition}
Consideremos al proceso $Y$ definido en (\ref{levy_salto}). Entonces el proceso $Y$ posee la propiedad de Markov.
\end{proposition}
\begin{proof}
Para mostrar que el proceso $Y$ cumple con la propiedad de Markov, definamos para una $s > 0$ fija al proceso $\widehat{Y}_t = (Y_{t+s}, \ t \geq 0)$ y al condicionar el proceso con respecto a $\mathcal{F}_s$ se tiene un proceso que comienza en $Y_s$. \\

Para toda $t \geq 0$ tenemos que $\widehat{Y}_t = Y_t + (Y_s - Y_s) = \tilde{Y}_t + Y_s$. Podemos considerar un número finito de tiempos pues al utilizar el Lema de Clases Monótonas se extiende el resultado a la $\sigma$-álgebra completa. Consideremos los siguientes tiempos tales que $t_1 \leq t_2 \leq \ldots \leq t_n$ y $f : \mathbb{R}^n \rightarrow \mathbb{R}$ tenemos que
\begin{align*}
\mathbb{E} \left[ f \left( \widehat{Y}_{t_1}, \ldots, \widehat{Y}_{t_n} \right) \bigg| \mathcal{F}_s \right] & = \mathbb{E} \left[ f \left( \tilde{Y}_{t_1} + Y_s, \ldots, \tilde{Y}_{t_n} + Y_s \right) \bigg| \mathcal{F}_s \right],
\end{align*}

Haciendo un abuso de notación y tomando $\{ Y_s = y \}$, vemos
\begin{align*}
\mathbb{E} \left[ f \left( \tilde{Y}_{t_1} + Y_s, \ldots, \tilde{Y}_{t_n} + Y_s \right) \bigg| \mathcal{F}_s \right] & = \mathbb{E} \left[ f \left( \tilde{Y}_{t_1} + y, \ldots, \tilde{Y}_{t_n} + y \right) \right] \hspace{0.5cm} \{ Y_s = y \},
\end{align*}

De las propiedades discutidas en (\ref{levy_salto}) sabemos que $\tilde{Y}_{t}$ se distribuye de la misma manera que $Y_t$, entonces
\begin{align*}
\mathbb{E} \left[ f \left( \widehat{Y}_{t_1}, \ldots, \widehat{Y}_{t_n} \right) \bigg| \mathcal{F}_s \right] & = \mathbb{E} \left[ f \left( \tilde{Y}_{t_1} + y, \ldots, \tilde{Y}_{t_n} + y \right) \right] \hspace{0.5cm} \{ Y_s = y \}, \\
& = \mathbb{E} \left[ f \left( Y_{t_1} + y, \ldots, Y_{t_n} + y \right) \right] \hspace{0.5cm} \{ Y_s = y \}, \\
& = \mathbb{E}_y \left[ f \left( Y_{t_1}, \ldots, Y_{t_n} \right) \right]
\end{align*}

Donde $\mathbb{E}_y$ representa el valor esperado del proceso respecto a la ley $\mathbb{P}_y$, y $Y$ es aleatorio con valor $Y_s$.
\end{proof}

La propiedad fuerte de Markov considera un tiempo de paro $\tau$ y $\tilde{Y}_{t} = Y_{t + \tau} - Y_{\tau}$ para toda $t \geq 0$. Al condicionar el proceso $\tilde{Y}_{t}$ con respecto al evento $\{ \tau < \infty \}$, entonces el proceso es independiente de $\mathcal{F}_{\tau}$.

\begin{proposition}
El proceso $Y$, definido en (\ref{levy_salto}) posee la propiedad fuerte de Markov.
\end{proposition}
\begin{proof}
La demostración de la propiedad fuerte de Markov básicamente es la misma que en el Teorema \ref{markov-fuerte}. La prueba se basa en considerar un evento $A \in \mathcal{F}_{\tau}$ con una sucesión de tiempos finitos tales que $t_1 \leq t_2 \leq \ldots \leq t_n$ y una función continua y acotada $f: \mathbb{R}^{+} \rightarrow \mathbb{R}$ tal que
\begin{align*}
\mathbb{E} \left[ 1_A f \left( Y_{t_1 + \tau} - Y_{\tau}, \ldots, Y_{t_n + \tau} - Y_{\tau} \right) \right] = \mathbb{P}(A) \mathbb{E} \left[ f \left( Y_{t_1}, \ldots, Y_{t_n} \right) \right],
\end{align*}
para después extender este resultado al proceso completo con respecto a la $\sigma$-álgebra $\mathcal{F}_{\tau}$ utilizando el Lema de Clases Monótonas. \\

Si se considera una sucesión de números como en el Teorema \ref{markov-fuerte} de tal forma que podamos aproximarnos a $ f\left( Y_{t_1 + \tau} - Y_{\tau}, \ldots, Y_{t_n + \tau} - Y_{\tau} \right)$, por la continuidad a la derecha y límites por la izquierda del proceso $Y$ tenemos que
\begin{align*}
\sum_{i = 1}^{\infty} 1_{ \{ \frac{i-1}{2^m} < \tau \leq \frac{i}{2^m} \} } f \left( Y_{\frac{i}{2^m} + t_1} - Y_{\frac{i}{2^m}}, \ldots, Y_{\frac{i}{2^m} + t_n} - Y_{\frac{i}{2^m}}  \right) \xrightarrow{m \rightarrow \infty} f \left( \tilde{Y}_{t_1}, \ldots, \tilde{Y}_{t_n} \right),
\end{align*}
donde $\tilde{Y}_{t_j} = Y_{t_j + \tau} - Y_{\tau}$. Luego, del Teorema de Convergencia Monótona, 
\begin{align}
	\mathbb{E} & \left[1_A f \left( \tilde{Y}_{t_1}, \ldots, \tilde{Y}_{t_n} \right) \right] \nonumber \\
	& = \lim_{m \rightarrow \infty} \sum_{i = 1}^{\infty} \mathbb{E} \left[ 1_{ A \cap \{\frac{i-1}{2^m} < \tau \leq \frac{i}{2^m}\}} f \left( Y_{\frac{i}{2^m} + t_1} - Y_{\frac{i}{2^m}}, \ldots, Y_{\frac{i}{2^m} + t_n} - Y_{\frac{i}{2^m}}  \right) \right]. \label{iphone}
\end{align}

Para poder aplicar la propiedad de Markov en la igualdad anterior, notemos que si $A \in \mathcal{F}_{\tau}$ entonces, casi seguramente $A \cap \{\frac{i-1}{2^m} < \tau \leq \frac{i}{2^m}\} \in \mathcal{F}_{i/2^m}$, entonces
\begin{align*}
	\mathbb{E} & \left[ 1_{ A \cap \{\frac{i-1}{2^m} < \tau \leq \frac{i}{2^m}\}} f \left( Y_{\frac{i}{2^m} + t_1} - Y_{\frac{i}{2^m}}, \ldots, Y_{\frac{i}{2^m} + t_n} - Y_{\frac{i}{2^m}}  \right) \right] \\
	& = \mathbb{E} \left[ 1_{ A \cap \{\frac{i-1}{2^m} < \tau \leq \frac{i}{2^m}\}} \mathbb{E} \left[ f \left( Y_{\frac{i}{2^m} + t_1} - Y_{\frac{i}{2^m}}, \ldots, Y_{\frac{i}{2^m} + t_n} - Y_{\frac{i}{2^m}}  \right) \bigg| \mathcal{F}_{ \frac{i}{2^m} } \right] \right] \\
	& = \mathbb{P} \left( A \cap \left\{ \frac{i-1}{2^m} < \tau \leq \frac{i}{2^m} \right\} \right) \mathbb{E} \left[ f \left( Y_{t_1}, \ldots, Y_{t_n} \right) \right].
	\end{align*}
Por lo tanto, sustituyendo en (\ref{iphone})
	\begin{align*}
	\mathbb{E} \left[ 1_A f \left( \tilde{Y}_{t_1}, \ldots, \tilde{Y}_{t_n} \right) \right] = \mathbb{P} (A) \mathbb{E} \left[ f \left( Y_{t_1}, \ldots, Y_{t_n} \right) \right], 
	\end{align*}
\end{proof}

\section{Condiciones Suficientes para la Optimización}
Ahora presentamos las condiciones suficientes para poder verificar que una solución propuesta es la óptima para resolver el problema de paro óptimo de la forma (\ref{problema_paro}), considerando a $G$ como una función sobre un proceso de Markov de dos dimensiones $(t, Y_t)$, es decir, una función de espacio y tiempo, donde $G$ es no negativa y medible.\\


\begin{lemma}
\label{condiciones_solucion}
	Considere el problema de paro óptimo definido en (\ref{problema_paro}), para $q \geq 0$, con el supuesto de que para toda $y \in \mathbb{R}$, existe casi seguramente $\lim_{t \rightarrow \infty} e^{-qt} G(Y_t)$ y además
    \begin{align}
		\mathbb{P}_y \left( \lim_{t \uparrow \infty} e^{-qt} G(Y_t) < \infty \right) = 1. \label{condicion1}
	\end{align}
    Si consideramos a $\tau^{*} \in \mathcal{T}$ como una estrategia candidata para el problema de paro óptimo (\ref{problema_paro}) y sea 
    \begin{align*}
		v^{*}(y) = \mathbb{E}_y [e^{-q \tau^{*}} G(Y_{\tau^{*}})].
	\end{align*}
Entonces, $(v^{*}, \tau^{*})$ es una solución si
  \begin{enumerate}
  	\item $v^{*}(y) \geq G(y)$ para toda $y \in \mathbb{R}$.
    \item El proceso $\{ e^{-qt} v^{*}(Y_t) : t \geq 0 \}$ es una supermartingala continua por la derecha.
  \end{enumerate}
\end{lemma}
\begin{proof}
Queremos observar que nuestra solución propuesta $(v^{*}, \tau^{*})$ es óptima, es decir, queremos mostrar que para toda $y \in \mathbb{R}$ la siguiente igualdad se cumple
\begin{align}
	v^{*}(y) = \sup_{\tau \in \mathcal{T}} \mathbb{E}_y \left[ e^{-q \tau} G(Y_{\tau}) \right]. \label{pd_condsuf}
\end{align}
De la definición de $v^{*}(y)$ se tiene que para toda $y \in \mathbb{R}$
\begin{align*}
	\sup_{\tau \in \mathcal{T}} \mathbb{E}_y \left[ e^{-q \tau} G(Y_{\tau}) \right] \geq v^{*}(y).
\end{align*}
Por otra parte, sabemos que el proceso $\{ e^{-qt} v^{*}(Y_t) : t \geq 0 \}$ es una supermartingala continua por la derecha, entonces utilizando el Teorema de Paro Opcional de Doob tenemos que para todo $t \geq 0$, $y \in \mathbb{R}$ y $\sigma \in \mathcal{T}$
\begin{align*}
	v^{*}(y) \geq \mathbb{E}_y \left[ e^{-q(t \wedge \sigma)} v^{*}(Y_{t \wedge \sigma}) \right],
\end{align*}
en particular tenemos que, 
\begin{align*}
	v^{*}(y) & \geq \liminf_{t \uparrow \infty} \mathbb{E}_y \left[ e^{-q(t \wedge \sigma)} v^{*}(Y_{t \wedge \sigma}) \right], 
\end{align*}
usando el hecho anterior y además que $v^{*}(y) \geq G(y)$ para toda $y$, el Lema de Fatou y la condición de no negativdad de la función $G$, vemos
\begin{align*}
	v^{*}(y) & \geq \liminf_{t \uparrow \infty} \mathbb{E}_y \left[ e^{-q(t \wedge \sigma)} v^{*}(Y_{t \wedge \sigma}) \right] \\
    & \geq \liminf_{t \uparrow \infty} \mathbb{E}_y \left[ e^{-q(t \wedge \sigma)} G(Y_{t \wedge \sigma}) \right] \\
    & \geq \mathbb{E}_y \left[ \liminf_{t \uparrow \infty} e^{-q(t \wedge \sigma)} G(Y_{t \wedge \sigma}) \right] \\
    & = \mathbb{E}_y \left[ e^{-q \sigma} G(Y_{\sigma}) \right],
\end{align*}
puesto que $\sigma \in \mathcal{T}$ es arbitraria, tenemos que para toda $y \in \mathbb{R}$
\begin{align*}
	 v^{*}(y) \geq \sup_{\tau \in \mathcal{T}} \mathbb{E}_y \left[ e^{-q \tau} G(Y_{\tau}) \right].
\end{align*}
De ambas desigualdades, vemos que se cumple (\ref{pd_condsuf}).
\end{proof}

Si consideramos las condiciones que se piden en el Lema \ref{condiciones_solucion}, podemos notar que para una función monótona creciente $G$ y $q > 0$, una clase de posibles soluciones son aquellas donde se supere algún umbral establecido. \\

Supongamos entonces que, $G$ es una función monótona creciente, si tomamos el tiempo $\tau$ en donde $Y_t$ se maximiza, entonces también ocurrirá para $G(Y_{\tau})$. Recordemos que la clase de problemas que estamos considerando, son aquellos donde la función $G$ tiene un factor de descuento exponencial, por lo tanto, no debería de transcurrir mucho tiempo para que el umbral que se estableció sea alcanzado. 

Ambas condiciones nos sugieren la existencia de un umbral, el cual posiblemente depende del tiempo, donde uno debería detenerse si se quiere maximizar el valor esperado de la ganancia. Si al tiempo $t > 0$ no se ha superado el umbral de paro, y el proceso $Y_t = y$, entonces cualquier tiempo de paro futuro dependerá solamente de la trayectoria del proceso a partir de este punto, teniendo un valor esperado de 
\begin{align}
	e^{-qt} \mathbb{E}_y \left[ e^{-q \tau} G(Y_{\tau}) \right]. \label{valor_esperado}
\end{align}
Optimizar (\ref{valor_esperado}) resulta ser un problema igual a (\ref{problema_paro}), además, no tiene ninguna utilidad considerar los tiempos de paro anteriores a $t$, gracias a la propiedad de Markov. \\

Los argumentos anteriores sugieren que el umbral no varía con respecto al tiempo, y entonces, las posibles soluciones son de la forma
\begin{align*}
	\tau_x^{*} = \inf \{ t > 0 : Y_t \in [x, \infty), x \in \mathbb{R} \}.
\end{align*}
El mismo razonamiento aplica en el caso donde $G$ es una función monótona decreciente. Considerar solamente a la función como monótona no asegura que la estrategia del umbral sea óptima, es por esta misma razón, lo que hace muy difícil poder hacer rigurosas las intuiciones anteriores. Las estrategias ``óptimas'' que se consideran para problemas de paro óptimo en particular, pueden variar enormemente si se cambia la naturaleza del problema. 

Se ha mostrado (véase \cite{avram}, \cite{kyprianou2}) que existen ciertas familias de problemas para las cuales, sus soluciones coinciden en varios aspectos, aunque resulta muy sencillo modificar el problema de tal manera que ninguna de las soluciones propuestas sea aplicable. Es por estas razones que vamos a considerar un problema de paro óptimo en particular. 

\section[Problema de McKean u Opción Americana]{Problema de Paro Óptimo de McKean u Opción Americana}
El problema de paro óptimo de McKean está dado por
\begin{align}
	v(y) = \sup_{\tau \in \mathcal{T}} \mathbb{E}_y \left[ e^{-q \tau} (K - e^{Y_{\tau}})^{+} \right], \label{mckean}
\end{align}
donde $q > 0$, $\mathcal{T}$ es la familia de tiempos de paro respecto a $\mathbb{F}$ y $(K - e^{Y_{\tau}})^{+} = \max(K - e^{Y_{\tau}}, 0)$. \\

El contexto de este problema puede ser visto como la venta de un activo con cierto riesgo a un valor específico. Donde $K$ es el precio fijo que se pacta para vender el activo, y el proceso $Y$, en este caso, sigue un movimiento Browniano con saltos negativos como se definió en (\ref{levy_salto}). El objetivo del problema es maximizar la ganancia que se puede esperar al momento de detenerse y obtener la mayor diferencia respecto al precio pactado $K$ con el precio del mercado $e^{Y_{\tau}}$. Este tipo de procesos son conocidos en finanzas como opciones americanas.  

Antes de entrar de lleno a la solución, veamos un resultado que será de utilidad a la hora de resolver el problema de McKean. Definamos para $y \in \mathbb{R}$ los tiempos de primera pasada, como
\begin{align*}
	\tau_{y}^{+} := \inf \{ t > 0 : Y_t > y \} \hspace{0.3cm} \text{y} \hspace{0.3cm} \tau_{y}^{-} := \inf \{ t > 0 : Y_t < y \}.
\end{align*}
Además, escribamos $\bar{Y}_t = \sup_{s \leq t} Y_s$.

\begin{lemma}
Para toda $q > 0$, $\beta \geq 0$ y $z \geq 0$, tenemos que
\begin{align}
	\mathbb{E} \left[ e^{-q \tau_{z}^{+} - \beta Y_{\tau_{z}^{+}}} 1_{ \{ \tau_{z}^{+} < \infty \} } \right] = \frac{\mathbb{E} \left[ e^{- \beta \bar{Y}_{e_q}} 1_{ \{ \bar{Y}_{e_q} > z \} } \right]}{\mathbb{E} \left[ e^{- \beta \bar{Y}_{e_q}} \right]}, \label{lema_mckean}
\end{align}
con $e_q$ como una variable aleatoria que es independiente de $Y$ y tiene una distribución exponencial.
\end{lemma}
\begin{proof}
Primero, supongamos que $q, \beta, z > 0$ y veamos que, si $\bar{Y}_{e_q} > z$ entonces el momento en que supero a $z$ ocurrió antes que $e_q$, es decir $\tau_z^{+} < e_q$, por lo tanto
\begin{align*}
	\mathbb{E} \left[ e^{- \beta \bar{Y}_{e_q}} 1_{ \{ \bar{Y}_{e_q} > y \}} \right]  & = \mathbb{E} \left[ e^{- \beta \bar{Y}_{e_q}} 1_{\{\tau_z^{+} < e_q\}} \right] \\
    & = \mathbb{E} \left[ e^{- \beta ( \bar{Y}_{e_q} + Y_{\tau_z^{+}} - Y_{\tau_z^{+}} )} 1_{\{\tau_z^{+} < e_q\}} \right] \\
    & = \mathbb{E} \left[ \mathbb{E} \left[ 1_{\{\tau_z^{+} < e_q\}} e^{- \beta Y_{\tau_z^{+}}} e^{ - \beta (\bar{Y}_{e_q} - Y_{\tau_z^{+}})} \bigg|  \mathcal{F}_{\tau_z^{+}} \right] \right] \\
    & = \mathbb{E} \left[ 1_{\{\tau_z^{+} < e_q\}} e^{- \beta Y_{\tau_z^{+}}} \mathbb{E} \left[ e^{- \beta (\bar{Y}_{e_q} - Y_{\tau_z^{+}})} \bigg|  \mathcal{F}_{\tau_y^{+}} \right] \right]
\end{align*}

De nuestra última expresión, consideremos la variable aleatoria $\bar{Y}_{e_q} - \bar{Y}_{\tau_z^{+}}$, vemos que 
\begin{align*}
	\bar{Y}_{e_q} - \bar{Y}_{\tau_z^{+}} & = \sup_{s \leq e_q} Y_s - Y_{\tau_z^{+}} \\
    & = \sup_{u \leq e_q - \tau_z^{+}} \left( Y_{u + \tau_z^{+}} - Y_{\tau_z^{+}} \right)
\end{align*}

Al ser $e_q$ una variable aleatoria que se distribuye exponencialmente, sabemos que posee la propiedad de pérdida de memoria, al estar condicionada sobre $\mathcal{F}_{\tau_y^{+}}$ y el evento $\{ \tau_z^{+} < e_q \}$ se tiene que $e_q - \tau_z^{+}$ se distribuye exponencialmente. Por otra parte de (\ref{poisson}) sabemos que $Y_{u + \tau_z^{+}} - Y_{\tau_z^{+}}$ tiene la misma distribución que $Y_{u}$. \\

Entonces, sabemos que $\bar{Y}_{e_q} - \bar{Y}_{\tau_z^{+}}$ tiene la misma distribución que $\bar{Y}_{e_q}$. Por lo tanto, tenemos que
\begin{align}
	\mathbb{E} \left[ e^{- \beta \bar{Y}_{e_q}} 1_{\{\bar{Y}_{e_q} > y\}} \right]  & = \mathbb{E} \left[ 1_{\{\tau_z^{+} < e_q\}} e^{- \beta Y_{\tau_z^{+}}} \mathbb{E} \left[ e^{- \beta (\bar{Y}_{e_q} - Y_{\tau_z^{+}})} \bigg| \mathcal{F}_{\tau_z^{+}} \right] \right] \nonumber \\
    & = \mathbb{E} \left[ 1_{\{\tau_z^{+} < e_q\}} e^{- \beta Y_{\tau_z^{+}}} \mathbb{E} \left[ e^{- \beta \bar{Y}_{e_q}} \right] \right] \nonumber \\
    & = \mathbb{E} \left[ 1_{\{\tau_z^{+} < e_q\}} e^{- \beta Y_{\tau_z^{+}}}  \right] \mathbb{E} \left[ e^{- \beta \bar{Y}_{e_q}} \right]. \label{lema_1}
\end{align}
Utilizando las propiedades de la esperanza condicional, tenemos que
\begin{align}
	\mathbb{E} \left[ 1_{\{\tau_z^{+} < e_q\}} e^{- \beta Y_{\tau_z^{+}}}  \right] & = \mathbb{E} \left[ e^{- \beta Y_{\tau_z^{+}}} \mathbb{E} \left[ 1_{ \{ \tau_z^{+} < e_q \}} \bigg| \mathcal{F}_{\tau_z^{+}} \right] \right] \nonumber \\
    & = \mathbb{E} \left[ e^{- \beta Y_{\tau_z^{+}}} \mathbb{P} \left( \tau_z^{+} < e_q \bigg| \mathcal{F}_{\tau_z^{+}} \right) \right] \nonumber \\
    & = \mathbb{E} \left[ e^{- \beta Y_{\tau_z^{+}}} \int_{\tau_z^{+}}^{\infty} q e^{-qs} ds \right] \nonumber \\
    & = \mathbb{E} \left[ e^{- \beta Y_{\tau_z^{+}}} e^{-q \tau_z^{+}} \right]. \label{lema_2}
\end{align}
Sustituyendo (\ref{lema_2}) en (\ref{lema_1}) tenemos
\begin{align*}
	\mathbb{E} \left[ e^{- \beta \bar{Y}_{e_q}} 1_{\{\bar{Y}_{e_q} > z\}} \right]  & = \mathbb{E} \left[ e^{- q \tau_z^{+} - \beta Y_{\tau_z^{+}}} \right] \mathbb{E} \left[ e^{- \beta \bar{Y}_{e_q}} \right],
\end{align*}
lo cual muestra la igualdad (\ref{lema_mckean}).
\end{proof}

Con el anterior Lema, procedamos a mostrar la solución al problema de McKean. Definimos a $\ubar{Y}_t = \inf_{s \leq t} Y_s$.

\begin{theorem}[Solución al problema de McKean]
\label{solu_mckean}
La solución a (\ref{mckean}) bajo los supuestos establecidos está dada por
\begin{align*}
v(y) = \frac{ \mathbb{E}  \left[ \left( K \mathbb{E}\left[ e^{\ubar{Y}_{e_q}} \right] - e^{y + \ubar{Y}_{e_q}}  \right)^{+} \right] }{ \mathbb{E} \left[ e^{\ubar{Y}_{e_q}} \right] },
\end{align*}
y el tiempo de paro óptimo está dado por
\begin{align*}
\tau^{*} = \inf \{ t > 0 : Y_t < y^{*} \},
\end{align*}
donde, 
\begin{align*}
y^{*} = \log \left( K \mathbb{E} \left[ e^{\ubar{Y}_{e_q}} \right] \right).
\end{align*}
Recordemos que $e_q$ es una variable aleatoria independiente de $Y$, con una distribución exponencial de intensidad $q$.
\end{theorem}
\begin{proof}
Recordemos que para poder utilizar el Lema \ref{condiciones_solucion} se tiene que verificar la condición (\ref{condicion1}). Esto resulta fácil de ver puesto que $G(Y_t) = (K - e^{Y_t})^{+}$. Siguiendo las condiciones que menciona el mismo Lema \ref{condiciones_solucion}, definamos las funciones de acotamiento
\begin{align}
v_{y'}(y) = \mathbb{E}_y \left[ e^{-q \tau_{y'}^{-}} \left( K - e^{Y_{\tau_{y'}^{-}}} \right)^{+} \right]. \label{f_acotada}
\end{align}
Veamos que utilizando el Lema \ref{condiciones_solucion} la solución al Problema de McKean está dado de la forma (\ref{f_acotada}), para una elección adecuada del valor de $y'$, es decir, $y' < \log(K)$. \\

Utilizando el Lema \ref{lema_mckean}, sustituimos $Y$ por $-Y$ y obtenemos el resultado análogo para el primer tiempo de entrada a un umbral negativo.
\begin{align}
	\mathbb{E}_y \left[ e^{-q \tau_{y'}^{-} + \beta Y_{\tau_{y'}^{-}}} 1_{\{\tau_{y'}^{-} < \infty\}} \right] = \frac{\mathbb{E} \left[ e^{\beta (\ubar{Y}_{e_q} + y)} 1_{\{ - \ubar{Y}_{e_q} > y - y'\}} \right]}{\mathbb{E} \left[ e^{\beta \ubar{Y}_{e_q}} \right]}, \label{mckean_eq1}
\end{align}
para $q, \beta \geq 0$ y $y-y' \geq 0$, entonces se sigue que
\begin{align}
v_{y'}(y) & = \mathbb{E}_y \left[ K e^{-q \tau_{y'}^{-}}  - e^{-q \tau_{y'}^{-} + Y_{\tau_{y'}^{-}}} \right] \nonumber \\
	& = \mathbb{E}_y \left[ K e^{-q \tau_{y'}^{-}} \right] - \mathbb{E}_y \left[ e^{-q \tau_{y'}^{-} + Y_{\tau_{y'}^{-}}} \right] \nonumber \\
	& = \mathbb{E} \left[ K 1_{\{ - \ubar{Y}_{e_q} > y - y'\}} \right] - \frac{\mathbb{E} \left[ e^{\ubar{Y}_{e_q} + y} 1_{\{ - \ubar{Y}_{e_q} > y - y'\}} \right]}{\mathbb{E} \left[ e^{\ubar{Y}_{e_q}} \right]} \nonumber \\
	& = \frac{ \mathbb{E} \left[K 1_{\{\ubar{Y}_{e_q} > y - y'\}} \mathbb{E} \left[ e^{\ubar{Y}_{e_q}} \right] \right] }{\mathbb{E} \left[ e^{\ubar{Y}_{e_q}} \right]} - \frac{\mathbb{E} \left[ e^{\ubar{Y}_{e_q} + y} 1_{\{ - \ubar{Y}_{e_q} > y - y'\}} \right]}{\mathbb{E} \left[ e^{\ubar{Y}_{e_q}} \right]} \nonumber \\
	& = \frac{ \mathbb{E} \left[ \left( K  \mathbb{E} \left[ e^{\ubar{Y}_{e_q}} \right] - e^{\ubar{Y}_{e_q} + y} \right) 1_{\{ - \ubar{Y}_{e_q} > y - y'\}} \right] }{ \mathbb{E} \left[ e^{\ubar{Y}_{e_q}} \right] }. \label{mckean_eq2}
\end{align}

La parte positiva puede ser excluida pues solo consideramos valores $y' < \log(K)$. Para asegurar que la solución efectivamente es de la forma (\ref{f_acotada}), verifiquemos los dos supuestos que se mencionan en el Lema \ref{condiciones_solucion}. \\

\textit{1. Cota inferior.} Comprobemos que $v_y(x) \geq (K - e^{x})^{+}$. Para ésto, primero notemos que de la misma definición de $v_y(x)$ en (\ref{f_acotada}) podemos afirmar que $v_y(x) \geq 0$. Por otra parte, podemos manipular la expresión (\ref{mckean_eq2}) para modificar la condición en la función indicadora y así obtener
\begin{align}
v_{y'}(y) & = \frac{ \mathbb{E} \left[ \left( K  \mathbb{E} \left[ e^{\ubar{Y}_{e_q}} \right] - e^{\ubar{Y}_{e_q} + y} \right) \left( 1 - 1_{\{ - \ubar{Y}_{e_q} \leq y - y'\}} \right) \right] }{ \mathbb{E} \left[ e^{\ubar{Y}_{e_q}} \right] }. \nonumber \\
	& = \frac{ \left( K \mathbb{E} \left[ e^{\ubar{Y}_{e_q}} \right] - e^{\ubar{Y}_{e_q} + y} \right) + \mathbb{E} \left[ \left( e^{\ubar{Y}_{e_q} + y} - K  \mathbb{E} \left[ e^{\ubar{Y}_{e_q}} \right] \right)  1_{\{ - \ubar{Y}_{e_q} \leq y - y'\}}  \right] }{ \mathbb{E} \left[ e^{\ubar{Y}_{e_q}} \right] }. \nonumber \\
    & = \frac{ \mathbb{E} \left[ e^{\ubar{Y}_{e_q}} (K - e^{y}) \right] }{ \mathbb{E} \left[ e^{\ubar{Y}_{e_q}} \right] } + \frac{ \mathbb{E} \left[ \left( e^{\ubar{Y}_{e_q} + y} - K  \mathbb{E} \left[ e^{\ubar{Y}_{e_q}} \right] \right) 1_{\{ - \ubar{Y}_{e_q} \leq y - y'\}} \right] }{ \mathbb{E} \left[ e^{\ubar{Y}_{e_q}} \right] }. \nonumber \\
	& = (K - e^{y}) + \frac{ \mathbb{E} \left[ \left( e^{\ubar{Y}_{e_q} + y} - K  \mathbb{E} \left[ e^{\ubar{Y}_{e_q}} \right] \right) 1_{\{ - \ubar{Y}_{e_q} \leq y - y' \}} \right] }{ \mathbb{E} \left[ e^{\ubar{Y}_{e_q}} \right] }. \label{cont_mckean}
\end{align}
De la ecuación anterior tenemos que una condición suficiente para que $v_{y'}(y) \geq (K - e^{y})$ es
\begin{align}
e^{y'} \geq K \mathbb{E} \left[ e^{\ubar{Y}_{e_q}} \right]. \label{mckean_eq3}
\end{align}

\textit{2. Condición de supermartingala.} Consideremos una variable aleatoria $I$ con la misma distribución que $\ubar{Y}_{e_q}$. En el evento $\{ t < e_q \}$ se puede describir a $\ubar{Y}_{e_q}$ desde el tiempo $t$. Si el valor mínimo del proceso ocurre antes del tiempo $t$ entonces $\ubar{Y}_{e_q} = \ubar{Y}_{t}$. 

Por otro lado, si el mínimo del proceso ocurre después del tiempo $t$ entonces podemos representar ese mínimo y manipularlo de la siguiente manera usando la propiedad de Markov del proceso $Y$ y recordando que el remanente en una diferencia de tiempos exponenciales es un tiempo exponencial
\begin{align*}
	\inf_{s \in (t, e_q]} Y_s & = \inf_{s \in (t, e_q]} (Y_s + Y_t - Y_t) \\
	& = \inf_{u \in (0, e_q - t]} (Y_{u+t} - Y_t + Y_t) \\
    & = \inf_{u \in (0, \bar{e}_q]} (\bar{Y}_u + Y_t) \\
    & = Y_t + \inf_{u \in (0, \bar{e}_q]} \bar{Y}_u\\
    & = Y_t + I.
\end{align*}
donde $\bar{e}_q = e_q - t$ es un tiempo exponencial y $\tilde{Y}_u = Y_{u+t} - Y_t$. Por lo que, tomando en mínimo para considerar ambos casos se tiene que
\begin{align*}
	\ubar{Y}_{e_q} = \ubar{Y}_t \wedge (Y_t + I).
\end{align*}
En particular se sigue que, en $\{ t < e_q \}$, $\ubar{Y}_{e_q} \leq Y_t + I$. Ahora supongamos que 
\begin{align}
e^y \leq K \mathbb{E} \left[ e^{\ubar{Y}_{e_q}} \right], \label{mckean_eq4}
\end{align}
entonces para toda $y \in \mathbb{R}$ y utilizando la igualdad (\ref{cont_mckean}) tenemos
\begin{align*}
v_{y'}(y) & = (K - e^{y}) + \frac{ \mathbb{E} \left[ \left( e^{\ubar{Y}_{e_q} + y} - K \mathbb{E} \left[ e^{\ubar{Y}_{e_q}} \right] \right) 1_{\{ - \ubar{Y}_{e_q} \leq y - y'\}} \right] }{ \mathbb{E} \left[ e^{\ubar{Y}_{e_q}} \right] } \\
	& = (K - e^{y}) + \frac{ \mathbb{E} \left[ 1_{\{ t < e_q \}} \mathbb{E} \left[ \left( e^{\ubar{Y}_{e_q} + y} - K \mathbb{E} \left[ e^{\ubar{Y}_{e_q}} \right] \right) 1_{\{ - \ubar{Y}_{e_q} \leq y - y'\}} \big| \mathcal{F}_t \right] \right] }{ \mathbb{E} \left[ e^{\ubar{Y}_{e_q}} \right] } \\
& + \frac{ \mathbb{E} \left[ 1_{\{ t \geq e_q \}} \mathbb{E} \left[ \left( e^{\ubar{Y}_{e_q} + y} - K \mathbb{E} \left[ e^{\ubar{Y}_{e_q}} \right] \right) 1_{\{ - \ubar{Y}_{e_q} \leq y - y'\}} \big| \mathcal{F}_t \right] \right] }{ \mathbb{E} \left[ e^{\ubar{Y}_{e_q}} \right] } \\
& \geq \frac{\mathbb{E} \left[ 1_{ \{t < e_q \}} \mathbb{E} \left[ \left( K  \mathbb{E} \left[ e^{\ubar{Y}_{e_q}} \right] - e^{Y_t + I + y} \right) 1_{\{ - (Y_t + I) > y - y'\}} \big| \mathcal{
F}_t \right] \right]}{ \mathbb{E} \left[ e^{\ubar{Y}_{e_q}} \right] }.
\end{align*}

Notemos que cuando $\{t < e_q\}$ se tiene que $\ubar{Y}_{e_q} \leq Y_t + I$, además, obtenemos la primera desigualdad al quitar los términos que son positivos. Luego, ocupando la propiedad de Markov del proceso $Y$ obtenemos,
\begin{align*}
	v_{y'}(y) & \geq \frac{\mathbb{E} \left[ 1_{ \{t < e_q \}} \mathbb{E}_{Y_t} \left[ \left( K  \mathbb{E} \left[ e^{\ubar{Y}_{e_q}} \right] - e^{I + y} \right) 1_{\{ - I > y - y'\}} \right] \right]}{ \mathbb{E} \left[ e^{\ubar{Y}_{e_q}} \right] } \\
	& \geq \mathbb{E} \left[ e^{- q t} v_{y'}(Y_t + y) \right] = \mathbb{E}_y \left[ e^{- q t} v_{y'}(Y_t) \right].
\end{align*}

La última desigualdad corresponde a la definición de $v_{y'}(y)$ y al hecho de que $e_q$ es una variable aleatoria exponencial independiente de $Y$. \\

Utilizando la propiedad de Markov, así como la última desigualdad y las propiedades de incrementos estacionarios e independientes del proceso $Y$ para tiempos $0 \leq s \leq t < \infty$ llegamos a que
\begin{align*}
\mathbb{E} \left[ e^{-qt} v_{y'}(Y_t) \big| \mathcal{F}_s \right] & = e^{-qs} \mathbb{E}_{Y_s} \left[ e^{-q(t-s)} v_{y'}(Y_{t-s}) \right] \\
& \leq e^{-qs} v_{y'}(Y_s)
\end{align*}
mostrando que $\{ e^{-qt} v_{y'}(Y_t) : t \geq 0 \}$ es una supermartingala. La continuidad por la derecha de las trayectorias se sigue de la continuidad de las trayectorias de $Y$ y la continuidad de $v_{y'}$ se puede concluir fácilmente de (\ref{cont_mckean}). \\

Notemos por último que, las condiciones \textit{1} y \textit{2} se satisfacen siempre y cuando se cumplan (\ref{mckean_eq3}) y (\ref{mckean_eq4}), es decir, cuando 
\begin{align*}
y' = \log \left( K \mathbb{E} \left[ e^{\ubar{Y}_{e_q}} \right] \right).
\end{align*}
\end{proof}

\section{Smooth Fit contra Continuous Fit}
De la condición $1$ del Lema \ref{condiciones_solucion} sabemos que la solución al problema (\ref{mckean}) está acotada inferiormente por la función de ganancia $G$, y más aún es igual a la función de ganancia en aquellos puntos donde la distribución de $Y_{\tau^{*}}$ se concentra. \\

Recurriendo a las propiedades de las trayectorias del proceso $Y$ se pueden verificar varias maneras en las que la función $v$ se aproxima a la función de ganancia $G$, nos referiremos a esta aproximación diciendo que $v$ se \textit{ajusta} a $G$. Resulta que el problema de paro óptimo de McKean es un excelente ejemplo para exponer un conflicto que ocurre cuando una solución se trata de \textit{ajustar} a la función de ganancia. \\

Decimos que existe una condición de \textit{continuous fit} al punto $y^{*}$ si los límites (puntuales) por la derecha y por la izquierda de $v$ en $y^{*}$ existen y además son iguales. Por otro lado, si las derivadas derechas e izquierdas de $v$ en el límite $y^{*}$ son iguales, decimos que existe una condición de \textit{smooth fit} en el punto $Y^{*}$. Ahora explicaremos más a detalle este problema que surge en (\ref{mckean}). \\

\begin{theorem}
La función $v(\log(y))$ es convexa para $y > 0$ y en particular existe \textit{continuous fit} de $v$ en $y^{*}$. La derivada derecha en $y^{*}$ está dada por 
\begin{align*}
v'(y^{*} +) = - e^{y^{*}} + K \mathbb{P} \left( \ubar{Y}_{e_q} = 0 \right).
\end{align*}
Por lo tanto, la solución al problema de paro óptimo muestra la existencia de \textit{smooth fit} en $y^{*}$ si y solo si $0$ entra inmediatamente a $(- \infty, 0)$.
\end{theorem}
\begin{proof}
Podemos ver que para dos números positivos $a, b$, la función $y \mapsto (a - by)^{+}$ es convexa, por lo que para la función de valor óptimo $v$ se tiene que es convexa con $y' = e^{y}$. Además, tomar el supremo es una operación subaditiva, entonces $v(\log(y))$ es una función convexa en $y$. Como toda función convexa es continua, $v$ es continua. \\

Ahora, estudiaremos la condición de \textit{smooth fit}, donde estableceremos las condiciones necesarias y suficientes para esta propiedad. Consideremos aquellos valores $y < y^{*} = \log \left( K \mathbb{E} \left[ \exp \{ \ubar{Y}_{e_q} \} \right]  \right)$, por lo tanto
\begin{align*}
v(y) = K - e^{y},
\end{align*}
y por ende, la derivada por la izquierda resulta ser
\begin{align*}
v'(y^{*} - ) = - e^{y^{*}},
\end{align*}
para establecer la condición de \textit{smooth fit} es necesario mostrar que $v'(y^{*} - ) = v'(y^{*} + ) = - e^{y^{*}}$. Recordemos que $y^{*} = \log (K \mathbb{E} [ \exp \{ \ubar{Y}_{e_q} \} ])$, por lo tanto tenemos que $e^{y^{*}} = K \mathbb{E} [ e^{ \ubar{Y}_{e_q} } ]$, del Teorema \ref{solu_mckean} tenemos que
\begin{align*}
v(y) & = \frac{ \mathbb{E} \left[ \left( e^{y^{*}} - e^{y + \ubar{Y}_{e_q}}  \right) 1_{\{ - \ubar{Y}_{e_q} > y - y^{*}\}} \right] }{ e^{y^{*}} / K } \\
	& = K e^{ - y^{*}} \mathbb{E} \left[ \left( e^{y^{*}} - e^{y + \ubar{Y}_{e_q}}  \right) 1_{\{ - \ubar{Y}_{e_q} > y - y^{*}\}} \right] \\
	& = K \mathbb{E} \left[ \left( 1 - e^{y + \ubar{Y}_{e_q} - y^{*}}  \right) 1_{\{ - \ubar{Y}_{e_q} > y - y^{*}\}} \right] \\
	& = - K \mathbb{E} \left[ \left( e^{y + \ubar{Y}_{e_q} - y^{*}} - 1  \right) 1_{\{ - \ubar{Y}_{e_q} > y - y^{*}\}} \right].
\end{align*}
Luego, al añadir un cero y desarrollar obtenemos
\begin{align*}
v(y) & = - K \mathbb{E} \left[ \left( e^{y + \ubar{Y}_{e_q} - y^{*}} + \left( e^{\ubar{Y}_{e_q}} - e^{\ubar{Y}_{e_q}} \right) - 1  \right) 1_{\{ - \ubar{Y}_{e_q} > y - y^{*}\}} \right] \\
	& = - K \mathbb{E} \left[ \left( e^{y + \ubar{Y}_{e_q} - y^{*}} - e^{\ubar{Y}_{e_q}}  \right) 1_{\{ - \ubar{Y}_{e_q} > y - y^{*}\}} \right] - K \mathbb{E} \left[ \left( e^{\ubar{Y}_{e_q}} - 1  \right) 1_{\{ - \ubar{Y}_{e_q} > y - y^{*}\}} \right] \\
	& = - K \left( e^{y - y^{*}} - 1 \right) \mathbb{E} \left[ e^{\ubar{Y}_{e_q}} 1_{\{ - \ubar{Y}_{e_q} > y - y^{*}\}} \right] - K \mathbb{E} \left[ \left( e^{\ubar{Y}_{e_q}} - 1  \right) 1_{\{ - \ubar{Y}_{e_q} > y - y^{*}\}} \right].
\end{align*}
Recordemos que $e^{y^{*}} = K \mathbb{E}[e^{\ubar{Y}_{e_q}}]$, entonces de la igualdad anterior podemos analizar la derivada por la derecha de $v$ como
\begin{align}
\frac{v(y) - (K - e^{y^{*}})}{y - y^{*}} & = \frac{v(y) + K(\mathbb{E}[e^{\ubar{Y}_{e_q}}] - 1)}{y - y^{*}} \nonumber \\
	& = - K \frac{\left( e^{y - y^{*}} - 1 \right)}{y - y^{*}} \mathbb{E} \left[ e^{\ubar{Y}_{e_q}} 1_{\{ - \ubar{Y}_{e_q} > y - y^{*}\}} \right] \nonumber \\ 
	& + K \frac{\mathbb{E} \left[ \left( e^{\ubar{Y}_{e_q}} - 1  \right) 1_{\{ - \ubar{Y}_{e_q} \leq y - y^{*}\}} \right]}{y - y^{*}}. \label{smooth_1}
\end{align}
Para poder estudiar en base al límite la anterior expresión, consideremos simplificar la notación llamando $A_y$ al primer término de la suma y $B_y$ al segundo. Es claro entonces que, 
\begin{align*}
\lim_{y \downarrow y^{*}} A_y = - K \mathbb{E} \left[ e^{\ubar{Y}_{e_q}} 1_{\{ - \ubar{Y}_{e_q} > 0\}} \right].
\end{align*}
Por otra parte, consideremos remover del valor esperado en $B$ la posibilidad de que la distribución de $\ubar{Y}_{e_q}$ tome un valor igual a $0$, asumiendo que $\exp \{ \ubar{Y}_{e_q} \} -  1 = 0$ en $\{ \ubar{Y}_{e_q} = 0 \}$. Se tiene entonces
\begin{align*}
\lim_{y \downarrow y^{*}} B_y & = K \frac{\mathbb{E} \left[ \left( e^{\ubar{Y}_{e_q}} - 1  \right) 1_{\{ 0 < - \ubar{Y}_{e_q} \leq y - y^{*}\}} \right]}{y - y^{*}} \\
	& = K \int_{ (0, y - y^{*} ] } \frac{e^{-z} - 1}{y - y^{*}} \mathbb{P} \left( \ubar{Y}_{e_q} \in \text{d}z \right) \\
	& = K \frac{e^{y - y^{*}} - 1}{y - y^{*}} \mathbb{P} \left( 0 < - \ubar{Y}_{e_q} \leq y - y^{*} \right) \\
	& + \frac{K}{y - y^{*}} \int_0^{y - y^{*}} e^{-z} \mathbb{P} \left( 0 < - \ubar{Y}_{e_q} \leq z \right) \text{ d}z,
\end{align*}
lo que nos conduce a que $\lim_{y \downarrow y^{*}} B_y = 0$. De (\ref{smooth_1}) y recordando que podemos escribir a $\mathbb{E}[e^{\ubar{Y}_{e_q}}]$ como $e^{y^{*}}$, podemos ver que la derivada por la derecha está definida como
\begin{align*}
v'(y^{*} + ) = - e^{y^{*}} + K \mathbb{P} \left( - \ubar{Y}_{e_q} = 0 \right).
\end{align*}
Finalmente, para que exista la igualdad en ambas derivadas, y por ende, la condición de \textit{smooth fit} es necesario y suficiente que $\mathbb{P} \left( - \ubar{Y}_{e_q} = 0 \right) = 0$, en otras palabras, un proceso $Y = (Y_t, t \geq 0)$ posee \textit{smooth fit} si y solo si el proceso entra inmediatamente en $(- \infty, 0)$.
\end{proof}

Para finalizar el contenido de este capítulo, analicemos brevemente la condición que se concluyó en el Teorema anterior. El resultado previo nos indica que no puede existir la posibilidad de tener \textit{smooth fit} si la distribución de $-\ubar{Y}_{e_q}$ toma en cuenta al evento $\{ -\ubar{Y}_{e_q} = 0 \}$. \\

A continuación damos un ejemplo en donde este evento este presente. De hecho, podemos considerar un proceso Poisson compuesto con tendencia positiva y saltos negativos
\begin{align*}
Y_t = ct + \sum_{i=1}^{N_t} \xi_i,
\end{align*}
donde $c > 0$, $(N_t, t \geq 0)$ es una proceso Poisson y $(\xi_i, i \geq 1)$ son variables aleatorias negativas, independientes e identicamente distribuidas. Podemos considerar a $T_1$ como el primer salto del proceso, y entonces, el evento $\{ T_1 > t \}$ es equivalente a decir que antes del tiempo $t$ no ha ocurrido ningún salto, por lo que
\begin{align*}
\mathbb{P} (-\ubar{Y}_{e_q} = 0) \geq \mathbb{P} (T_1 > e_q) = e^{-\lambda e_q} > 0,
\end{align*}
donde $\lambda$ es el parámetro de intensidad de los intervalos para los saltos del proceso. \\

Por otro lado, cuando el proceso no entra inmediatamente en $(- \infty, 0)$ es equivalente a
\begin{align*}
\mathbb{P} (\tau^{-}_{0} > 0) = \mathbb{P} (\inf \{ t > 0 : Y_t < 0 \} > 0) = 1.
\end{align*}

Finalmente, analicemos el caso en que tenemos presencia del movimiento Browniano, es decir, cuando $Y_t$ es un proceso de Lévy como se definió en (\ref{levy_salto}). Veamos que $\mathbb{P}( - \ubar{Y}_{e_q} = 0) = 0$ con $e_q$ como una variable aleatoria independiente de $Y$ y que tiene una distribución exponencial. \\

Al ser $e_q$ una variable independiente de $Y$ para todo $t \geq 0$, observemos que podemos escribir 
\begin{align*}
	\mathbb{P} \left( \underline{Y}_{e_q} = 0 \right) = \int_0^{\infty} ds \lambda e^{- q s} \mathbb{P} \left( \inf_{u \in (0, s]} Y_u = 0 \right).
\end{align*}

Por lo que, si determinamos que $\mathbb{P}( \underline{Y}_s = 0) = 0$ para toda $s$, habremos probado que el proceso $Y$ posee la propiedad de \textit{smooth fit}. Entonces, veamos que podemos considerar dos casos en la probabilidad anterior
\begin{align}
	\mathbb{P} \left(\inf_{u \in (0, s]} Y_u = 0\right) & = \mathbb{P}\left( \inf_{u \in (0, s]} Y_u = 0, s \leq T_1 \right) \nonumber \\
    & + \mathbb{P}\left( \inf_{u \in (0, s]} Y_u = 0, s > T_1 \right). \label{correcJCsmooth}
\end{align}

Veamos que el lado de derecho de la identidad (\ref{correcJCsmooth}) es igual a cero. Notemos que $\mathbb{P}(\tau_0^{-} > T_1)$ es mayor igual que el segundo miembro de la parte derecha de la identidad (\ref{correcJCsmooth}), donde $\tau_0^{-}$ es el primer momento en que $Y < 0$. Por lo tanto, queremos observar que $\mathbb{P}(\tau_0^{-} > T_1) = 0$, tenemos entonces que
\begin{align*}
	\mathbb{P}(\tau_0^{-} > T_1) & = \mathbb{P}\left( \inf_{s \in (0, T_1)} Y_s = 0 \right).
\end{align*}

Al considerar solo el intervalo hasta justo antes de que ocurra el primer salto negativo, tenemos que
\begin{align*}
	\mathbb{P}(\tau_0^{-} > T_1) & = \mathbb{P}\left( \inf_{s \in (0, T_1)} as + \sigma B_s = 0 \right) \\
    & = \int_0^{\infty} ds \lambda e^{- \lambda s} \mathbb{P} \left( \inf_{u \in (0, s)} au + \sigma B_u = 0 \right).
\end{align*}

La última igualdad ocurre bajo el supuesto de que $T_1$ es una variable que se distribuye exponencialmente con parámetro $\lambda$ y es independiente del proceso $Y$. Por lo que basta ver que
\begin{align}
	\mathbb{P} \left( \inf_{u \in (0, s)} au + \sigma B_u = 0 \right) = 0. \label{antesdeT12}
\end{align}

Recordemos de la Proposición \ref{martin_continuas} que 
\begin{align*}
	M_t^{(\lambda)} = \exp \left\{ \lambda B_t - \frac{\lambda^2}{2} t \right\} \hspace{0.3cm} \text{ para } \lambda \in \mathbb{R},
\end{align*}
es una martingala y además $\mathbb{E}[M_t^{(\lambda)}] = 1$. Esto nos permite definir una nueva medida de probabilidad
\begin{align*}
	\mathbb{P}^{(\lambda)} \left(\Lambda \right) = \mathbb{E} \left[ M_t^{(\lambda)} 1_{\Lambda} \right] \hspace{0.3cm} \text{ para todo } \Lambda \in \mathcal{F}_t.
\end{align*}

Analicemos al proceso $(\sigma B_s - as, s \geq 0)$ bajo la medida $\mathbb{P}^{(\lambda)}$ y observemos que
\begin{align*}
	\mathbb{E}^{(\lambda)} \left[ e^{\theta (\sigma B_s - as)} \right] & = \mathbb{E} \left[ \exp \left\{ \theta \sigma B_s - \theta a s +  \lambda B_s - \frac{\lambda^2}{2} s \right\} \right] \\
    & = \exp \left\{ - \theta a s - \frac{\lambda^2}{2} s \right\} \mathbb{E} \left[ e^{ (\theta \sigma +  \lambda) B_s } \right].
\end{align*}

De (\ref{asdfghjk}) sabemos que $\mathbb{E} \left[ e^{ \theta B_t } \right] = e^{t\theta^2 / 2}$, entonces
\begin{align*}
	\mathbb{E}^{(\lambda)} \left[ e^{\theta (\sigma B_s - as)} \right] & = \exp \left\{ - \theta a s - \frac{\lambda^2}{2} s \right\} \exp \left\{ \frac{(\theta \sigma +  \lambda)^2}{2} s \right\} \\
    & = \exp \left\{ - \theta a s + \frac{ \theta^2 }{2} \sigma^2 s + \theta \sigma \lambda s \right\}.
\end{align*}

Luego, tomando $\lambda = a/\sigma$ tenemos que
\begin{align*}
	\mathbb{E}^{(a/\sigma)} \left[ e^{\theta (\sigma B_s - as)} \right] = e^{ \frac{ (\theta \sigma)^2 }{2} s}.
\end{align*}

Por lo que $(\sigma B_s - as, s \geq 0)$ bajo la medida $\mathbb{P}^{(a/\sigma)}$ es un movimiento Browniano con varianza $\sigma^2$. En otras palabras, $(\sigma B_s, s \geq 0)$ bajo $\mathbb{P}^{(a/\sigma)}$ es un movimiento Browniano con varianza $\sigma^2$ y deriva $a$. Entonces, podemos concluir que 
\begin{align*}
	\mathbb{E}^{a/\sigma} \left[ 1_{ \{ \inf_{u \in (0, s)} \sigma B_u = 0 \}} \right] & = \mathbb{E} \left[ 1_{\{ \inf_{u \in (0, s)} \sigma B_u = 0 \}} M_t^{a/\sigma} \right] \\
    & = 0.
\end{align*}
La última desigualdad se da por el hecho de que el movimiento Browniano entra inmediatamente a $(- \infty, 0)$, por lo que la función indicadora es igual a $0$. Lo anterior implica que $\mathbb{P}(\tau_0^{-} > T_1) = 0$ y entonces el segundo miembro de la parte derecha de la identidad (\ref{correcJCsmooth}) es igual a $0$. Finalmente notemos que el primer miembro de la parte derecha de la identidad (\ref{correcJCsmooth}) es menor o igual que la probabilidad definida en (\ref{antesdeT12}), la cual es igual a $0$. Por lo tanto, 
\begin{align*}
	\mathbb{P} \left(-\inf_{u \in (0, s]} Y_u = 0\right) = 0, 
\end{align*}
es decir, se tiene la condición de \textit{smooth fit} para el proceso 
\begin{align*}
	Y_t = at + \sigma B_t + \sum_{i=1}^{N_t} X_i.
\end{align*}


\begin{thebibliography}{9}
\addtolength{\leftmargin}{0.2in}
\setlength{\itemindent}{-0.2in}

\bibitem[1]{apostol} Apostol, Tom M. \emph{Calculus}. Reverté Ediciones. México. 1999.
\bibitem[2]{bartle} Bartle, Robert G. \emph{The Elements of Real Analysis}. John Wiley \& Sons. Londres. 1964.
\bibitem[2]{billingsley} Billingsley, Patrick. \emph{Convergence of Probability Measures}. John Wiley \& Sons. 2da edición. Londres. 1999
\bibitem[3]{doob} Doob, J. L., \emph{Stochastic Processes}, Wiley, Ser. Probab. Math. Stat. New York. 1953.
\bibitem[4]{jacodprotter} Jacod, Jean; Protter, Philip E. \emph{Probability essentials}. Universitext, Springer. 2da edición. New York. 2000.
\bibitem[5]{kyprianou} Kyprianou, Andreas E. \emph{A Hands-on Approach to Optimal Stopping}, 6th School of Probability and Stochastic Process. México. 2009.
\bibitem[6]{marczyg} Marcinkiewicz, J.; Zygmund A. \emph{Sur les fonctions indépendantes}, Fundam. Math. 29 (1937), 309–335.
\bibitem[7]{mortersperes} Mörters, P.; Peres Y. \emph{Brownian Motion}, Cambridge University Press. 2da edición. Cambridge. 2010.
\bibitem[8]{peskirshiryaev} Peskir, Goran; Shiryaev, Albert. \emph{Optimal Stopping and Free-Boundary Problems}. Lectures in Mathematics. ETH Zürich, Springer Science \& Business Media. 2006.
\bibitem[9]{shiryaev} Shiryaev, Albert. \emph{Probability}. Graduate Text in Mathematics Vol. 95. Springer. 2da edición. New York. 1995.

\end{thebibliography}

\end{document}