Los objetivos de este trabajo consisten en analizar los problemas de paro óptimo a tiempo discreto y continuo, estudiar las posibles estrategias para su solución y las características de la función de ganancia que surgen en este tipo de problemas. \\

Los problemas de paro óptimo consisten en determinar el mejor momento para tomar una decisión y reaccionar ante una secuencia de observaciones que provienen de variables aleatorias, tratando de maximizar las ganancias esperadas o minimizar los costos esperados. \\ 

Dichos problemas surgen naturalmente en muchas áreas de las matemáticas aplicadas, como lo son: la Estadística (Inferencial o Bayesiana), las Matemáticas Financieras o la Investigación de Operaciones. Este trabajo es un referente introductorio a este tipo de problemas y está basado en el artículo de Andreas Kyprianou: \textit{A Hands-On Approach to Optimal Stopping} realizado durante la 6ta Escuela de Probabilidad y Procesos Estocásticos en el Centro de Investigaciones en Matemáticas, Guanajuato, Guanajuato. \\

Esta tesis es autocontenida, salvo posibles ocasiones en las que se requiera mencionar un resultado importante y no sea posible realizar una demostración apropiada debido al alcance del mismo, entonces se hará referencia a la bibliografía. \\

La estructura del trabajo consta de cinco capítulos. En el primer capítulo se estudia una de las herramientas básicas para el desarrollo de esta tesis, las martingalas. Introducimos los conceptos de martingala, tiempo de paro, así como importantes desigualdades que nos llevan a resultados de convergencia de martingalas. \\

En el segundo capítulo presentamos una definición rigurosa del problema de paro óptimo a tiempo discreto. Iniciamos con el tiempo discreto para desarrollar la idea general y así analizar fácilmente el problema a tiempo continuo. Se define el supremo esencial, que es de suma importancia para obtener la solución del problema. Para finalizar el capítulo, además de utilizar las martingalas para resolver el problema de paro óptimo, se presenta un enfoque con cadenas de Markov. \\

El movimiento Browniano se estudia en el tercer capítulo, pues es un componente importante del problema de paro que se analiza en esta tesis. Presentamos una construcción del movimiento Browniano para observar propiedades importantes de sus trayectorias y examinar la propiedad fuerte de Markov que éste posee. \\

El cuarto capítulo está dedicado a estudiar las martingalas continuas y la generalización de los resultados vistos en el primer capítulo. Aquí es donde vemos las características que adquieren los tiempos de paro cuando se trata de un tiempo continuo. Analizamos las principales desigualdades maximales en martingalas que dan paso a revisar el Teorema de Paro de Doob resultado que nos guiará a encontrar una estrategia óptima para resolver el problema a considerar en el capítulo final. Por último, presentamos aplicaciones de las martingalas al movimiento Browniano, las cuales nos ayudarán durante el análisis de los resultados finales. \\

Finalmente, consideramos en  el problema de paro óptimo a tiempo continuo un proceso basado en tres componentes: una parte lineal positiva, un movimiento Browniano y un proceso Poisson. Esta familia de procesos estocásticos es conocida como procesos de Lévy. Vemos las condiciones que se requieren para que se cumpla la optimización requerida por problema de paro. Analizamos al Problema de McKean, un problema de paro que modela al instrumento financiero conocido como Opción Americana y concluimos estudiando una propiedad que se presenta para este tipo de problema de paro, la condición de \textit{Smooth Fit}. \\ 

A lo largo del trabajo se tiene el supuesto de que el lector posee las nociones básicas en teoría de medida, para lo cual se puede consultar \cite{jacodprotter} o \cite{shiryaev}. 